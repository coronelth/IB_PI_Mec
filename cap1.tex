\chapter{Introducción}
\graphicspath{{figs/cap1/}}
\label{cap1}



En el estudio de la Mecánica de los Fluidos, son de importancia los problemas de transferencia de calor de flujos multi-fásicos con cambios de fase. Uno de los más relevantes por su importancia en la aplicación industrial y que contínuamente está en investigación debido a su complejidad; es la interacción entre el fluido de un reactor nuclear con sus barras de elemento combustible.

El punto crítico de calor (\textit{Critical Heat Flux} o CHF ) es aquél dónde la curva de Nukiyama de un material aumenta en varios órdenes de magnitud se temperatura si se incremente infinitesimalmente su potencia. Conocer el CHF de un reactor nuclear determina el punto de operación del mismo; debido a que el ente regulatorio nuclear que avala el funcionamiento de un reactor, determina la potencia de operación del mismo en base a la alcanzada en el CHF.

Nukiyama estudió, en la curva que lleva su nombre, cómo es la variación de la temperatura según la potencia que se le esté aplicandoa un material inmerso en un fluido y describió con detalle las diferentes etapas del proceso de ebullición. Actualmente la única forma de validar el CHF de las barras de elemento combustible de un reactor es mediante ensayos experimentales, los cuáles insumen mucho tiempo de espera (años) para realizar su prueba, porque son escasos los lugares que cuentan con la tecnología para brindar el servicio, además de ser costosos. 

Es de importancia contar con un código numérico que se encuentre validado, reproduzca la fenomenología de forma efectiva, sea fácil de utilizar cambiando la geometría y/o parámetros del fluido a utilizar. Los costos de fabricación disminuirían considerablemente al no ser necesaria una gran iteración de prototipos (en el problema mencionado).

Actualmente el dilema para reproducir numéricamente los tipos de problemas mencionados es que debido a la escala del fluido (\textit{mesoscópica}) con las técnicas convencionales de resolución de Mecánica de Fluidos Computacional(\textit{Computational Fluid Dynamicso} CFD) no se obtienen resultados en un tiempo razonable y que también tenga en cuenta todos las interacciones físicas que ocurren. En parte es debido a que los métodos tradicionales resuelven ecuaciones diferenciales, donde su tiempo de cálculo varía cuadráticamente con el número de elementos de malla utilizada para discretizar el problema físico.

Uno de los métodos numéricos que se están desarrollando para que el costo computacional sea cada vez menor es el método de lattice Boltzmann. Éstos métodos resuelven las Ecuaciones Diferenciales Ordinarias (EDO) lineales por medio de una ecuación lineal más sencilla, lo cuál reduce drásticamente el tiempo de cálculo. Las variables que se obtienen cumplen con la EDO de interés. Debido a que están basados en realizar operaciones elementales en los elementos de la malla discretizada y a su vez no dependen del total de nodos de la malla, lo hace altamente paralelizable. 

\section{Descripción de las escalas de los fluidos}

Los fluidos como el aire y el agua son conocidos en nuestra vida diaria. Físicamente los fluidos están compuestos de un gran conjunto de átomos o moléculas que chocan unas con otras moviéndose  aleatoriamente. Usualmente las interacciones de las moléculas de un fluido son más débiles que las de un sólido, por ello mediante la aplicación de un pequeño esfuerzo al fluido, éste puede ser deformado de manera continua.


Usualmente la dinámica microscópica de las moléculas del fluido son muy complicadas y demuestran una fuerte inhomogeneidad y fluctuaciones.
Por otro lado la dinámica macroscópica del fluido, el cual es el resultado medio del movimiento de las moléculas en un medio homogéneo y continuo.
Mediante modelos matemáticos puede explicarse la dinámica de los fluidos , según el fluido observado y su fuerte dependencia del tamaño de su escala.

La mecánica de los fluidos se puede describir en tres niveles: macroscópica, mesoscópica y microscópica.
Generalmente el movimiento de un fluido puede ser descripto por tres tipos de modelos matemáticos acuerdo a lo que se observan en las distintas escalas, por ejemplo microscópico en modelos de escala molecular, teorías cinéticas en la escala mesoscópica y modelos continuos para escalas macroscópicas.


Los modelos matemáticos de los flujos de fluidos son :

\begin{itemize}
	\item ecuación de Newton para una cantidad elevada de moléculas (escala microscópica)
	\item ecuación de Boltzman para la función de distribución simple (escala mesoscópica)
	\item ecuaciónes de aproximación de la mecánica del contínuo, como ser la ecuación de Navier-Stokes para macro-variaciones de flujos (escala macroscópica).
\end{itemize}

que resultan extremadamente difíciles de resolver analíticamente de no ser imposibles.

 La precisión de los modelos numéricos sin embargo han provisto de manera satisfactoria soluciones aproximadas de dichas ecuaciones.
Particularmente con la rapidez del desarrollo del software y hardware computacional y la tecnología, las simulaciones numéricas han comenzado a ser una importante metodología para la dinámica de fluidos.

Una de las vertientes de  investigación y aplicación más popular de simulación de fluidos es la técnica Mecánica de Fluidos Computacional (\textit{Computational Fluid Dynamics} o CFD) , el cual está diseñado para resolver ecuaciones hidrodinámicas basadas en supociones de continuidad. Siendo sus técnicas más difundidas la de elementos finitos, volúmenes finitos entre otras.
En CFD el dominio del flujo está compuesto según la técnica \textit{meshless} utilizada, como  diferencia finitas; los cuales resultan en un sistema algebraico de sistemas de ecuaciones para las variables discretas del fluido asociadas a la malla computacional. 
Computacionalmente son llevados a cabo para encontrar una solución aproximada resolviendo el problema algebraico de sistemas de ecuaciones usando un algoritmo secuencial o paralelo.
 
La Simulación de Dinámica Molecular (\textit{Molecular Dynamic Simulation} o MDS) es una técnica en la cual el movimiento individual de átomos o moléculas del fluido son registrados para resolver las ecuaciones de Newton.
La principal ventaja de MDS es el aspecto macroscópico del fluido puede ser directamente conectado con el comportamiento molecular, en donde la estructura molecular y las interacciones microscópicas pueden ser descriptas de una manera directa.

El método de lattice Boltzmann ( \textit{lattice Boltzman Method} o LBM) es una aproximación del nivel mesoscópica. LBM estudia la micro-dinámica de partículas ficticias utilizando modelos cinéticos simplificados, lo cual provee un camino alternativo de simular la mecánica de fluidos. La naturaleza de la cinética brinda distintas características de LBM tales que es claro el panorama de los procesos de advección y colisión de partículas de fluidos simuladas. La estructura del algoritmo es simple y de fácil implementación en las condiciones de contorno, además presenta un paralelismo natural. Todos éstos interesantes atributos hacen que LBM sea una potente herramienta numérica para la simulación de sistemas de fluidos envueltos en problemas físicos complejos\cite{guo2013lattice}. 


\section{Descripción general de LBM}

Para resolver un problema de mecánica de fluidos de forma numérica, se ha de discretizar el espacio físico mediante alguna técnica \textit{meshless}. Tradicionalmente los problemas se resueltos mediante la suposición del contínuo, y son llevados a cabo mediante las ecuaciónes de Navier-Stokes, que implican resolver un sistema algebraico de ecuaciones por ser EDO lineales. Según la cantidad de elementos que posea la malla, se regirá el tiempo de cálculo del problema; por ir aumentando de forma cuadrática el tiempo de cálculo, según la cantidad de elementos.

Uno de los métodos utilizados para resolver de manera indirecta las EDO lineales es el método de lattice Boltzmann. El cual consiste en resolver de forma sencilla una ecuación, cuyas variables obtenidas se pueden relacionar con la solución de la EDO lineal.

Los modelos de lattice Boltzmann provienen de la ecuación de Boltzmann, ésta se encuentra discretizada en espacios para resolver los problemas según: espacio físico, espacio de velocidades y espacio temporal. 

La discretización espacial surge de realizar un mallado a la región donde se plantea resolver un dado problema físico, y, cada nodo de la malla tiene asignado un espacio de velocidades. Todos los modelos se encuentran clasificados según el tipo \textit{DdQq} (\textit{d}-dimensiones  \textit{q}-velocidades), siendo \textit{q} la discretización realizada en el espacio de velocidades. La figura \ref{fig:D1Q3_D2Q9} muestra como es el conjunto de velocidades para el modelo D1Q3 y D2Q9.

El espacio de velocidades indica cómo es la propagación de las propiedades que poseen los nodos de la grilla. La velocidad de grilla del nodo i-ésimo se denota $\mathbf{e}_{i}$ y posee \textit{q} componentes. Para el modelo D2Q9 la figura \ref{fig:D1Q3_D2Q9} muestra un esquema de las velocidades de grilla del nodo i - ésimo y la Ec. (\ref{eq:velgrilla}) el valor adoptado.

\begin{figure}[h!]
	\centering
	\includegraphics[width=.8\textwidth]{figs/cap1/D1Q3_D2Q9}
	\caption{Conjunto de velocidades de los modelos D1Q3 y D2Q9. \cite{kruger2017lattice}}
	\label{fig:D1Q3_D2Q9}	
\end{figure}


\begin{equation}
{\mathbf{e}}_{i} =  
\left( \begin{array}{c} 
e_{i0} \\ e_{i1}\\ e_{i2}\\ e_{i3}\\ e_{i4}\\ e_{i5}\\
e_{i6}\\ e_{i7}\\ e_{i8}\\
\end{array}
\right) =
\left( \begin{array}{c} 
(0,0,0) \\ (1,0,0) \\ (0,1,0) \\(-1,0,0) \\ (0,-1,0) \\ (1,1,0) \\
(-1,1,0) \\ (-1,-1,0) \\ (1,-1,0)\\ 
\end{array}
\right) 
\label{eq:velgrilla}
\end{equation}




En el presente trabajo se resolverán los problemas descriptos en Cap. (\ref{cap4}), los cuáles son de 2 dimensiones y se adoptó el tipo de modelo D2Q9. 

Para los problemas de flujos multifásicos existen cuatro modelos generales de LB , siendo ellos:

\qquad \qquad color gradient, free energy, phase field y pseudopotential.

El modelo que se utilizará en el presente trabajo es el pseudopotencial, el cuál está basado en proponer un potencial de interacción entre las partículas del fluido. El potencial se utiliza para calcular la fuerza de interacción entre las partículas del fluido y está dado según una Ecuación de estado. 

El modelo posee un campo de distribución de probabilidades \textit{f} que tiene asociado \textit{q} componentes, siendo dependiente de la fuerza de interacción. Por medio de \textit{f} se pueden recuperar las variables macroscópicas asociadas del problema a resolver.

A continuación se datallará un ejemplo sencillo de un modelo \textit{D2Q9}. La Figura \ref{fig:grilla_D2Q9} muestra un esquema de las direcciones de las velocidades de grilla del nodo i-ésimo. En la misma figura se esquematiza como  es el proceso de colisión y advección (\textit{streaming}) que se realiza para actualizar los estados de los nodos cuando transcurre un paso de tiempo de la discretización temporal del problema. 


\begin{figure}[h!]
	\centering
	\includegraphics[width=8cm]{grilla_stre_colli_intro.png}
	\caption{Colisión y streaming de un modelo D2Q9 de LBM.}
	\label{fig:grilla_D2Q9}
\end{figure}


\begin{align}
	\mathbf{f}_{\alpha} (\mathbf{x} + \mathbf{e}_{\alpha} \mathbf{\delta}_{t}, t + \mathbf{\delta}_{t})  = \mathbf{f}_{\alpha} (\mathbf{x}, t) - \frac{1}{\tau} (\mathbf{f}_{\alpha} - {\mathbf{f}_{\alpha}}^{eq})
	\label{eq:field_intro} 
\end{align}

El término derecho de la Ec. (\ref{eq:field_intro}) contiene el proceso de \textit{colisión} y el izquierdo al \textit{streaming}; $\alpha$ corresponde al  $\alpha$-ésimo componente de los \textit{q} componentes del modelo, en éste ejemplo $\alpha = 1, 2, ...,9$. $f^{eq}$ es la función de distribución en estado de equilibrio. $\tau$ es un parámetro de relajación del modelo que depende fuertemente de las propiedades macroscópicas; como ser la densidad $\rho$, viscosidad $\nu$ entre otras.

Por medio de la \textit{colisión} y luego del \textit{streaming} se obtienen los parámetros macroscópicos el problema. Para éste caso sencillo se obtiene $\rho$ y $\mathbf{u}$ de las Ec. (\ref{eq: rho.1}) y (\ref{eq: u.1}) respectivamente.

\begin{align}
	\rho = \sum_{\alpha} \mathbf{f}_{\alpha}
	\label{eq: rho.1}
\end{align}

\begin{align}
	\rho \mathbf{u}= \sum_{\alpha} \mathbf{e}_{\alpha} \mathbf{f}_{\alpha}
	\label{eq: u.1}
\end{align}

El método numérico en éste caso tiene los siguiente forma de resolución:
\newline
{\scriptsize

\xymatrix{\>\ar@{->}[r]&\textbf{Colisión}\ar@^{->}[r]&\textbf{Streaming}\ar@{->}[r]&\textbf{C. de Contorno}\ar@{->}[r]&\textbf{Cálculo $\rho$, \textit{F} y \textit{U}\ar@{->}[r]}&\textbf{Actualización$\rho$, \textit{F} y \textit{U}}\ar@{->}[r]&\> \ar@/^{7mm}/[llllll]_{SIGUIENTE \> PASO \> DE \> TIEMPO}}
}
.
\newline 
\newline 

Por medio de las variables macroscópicas que se obtienen, se recupera la solución de la ecuación diferencial de interés

Debido a que cada nodo de la grilla debe realizar la misma operación de colisión de manera independiente del resto, el modelo es altamente paralelizable. Y además las operaciones matemáticas que deben ejecutarse en el operador de colisión son sencillas, no implicando un gran costo computacional.


\section{Descripción GPU}

Una Unidad de Procesamiento Gráfico (\textit{Graphics Processing Unit} o GPU) es un  circuito electrónico diseñado para realizar operaciones con punto flotante para renderizar píxeles en una pantalla. Están optimizadas para actuar en paralelo de forma simultánea instrucciones simples.

La GPU trabaja en conjunto con una Unidad Central de Procesamiento (\textit{Central Processing Unit} o CPU), debido a ello no presenta circuitos de control en su arquitectura. Dicho espacio que se encuentra ocupado en las CPU por el circuito de control las GPU lo disponen para incrementar el espacio en su chip de Unidades Aritmético Lógicas (\textit{Arithmetic Logic Unit} o ALU).

El esquema de la Figura \ref{fig:cpu_gpu_transis} se muestra cuanlitativamente la cantidade de transistores dedicados a diferentes tareas en la CPU comparado con la GPU.

\begin{figure}[h!]
	\centering
	\includegraphics[width=10cm]{cpu_gpu.png}
	\caption{Comparación cualitatica del uso de transistores entre CPU y GPU \cite{rinaldi2011modelos}.}
	\label{fig:cpu_gpu_transis}
\end{figure}

Las GPUs estan acondicionadas para llevar a cabo los cálculos que presentan los métodos numéricos de LB. Los LBM consisten en que a cada uno de los nodos de la grilla discretizada se le realicen operaciones elementales como los de la ecuación \ref{eq:field_intro}. Las GPU se encuentran optimizadas para ejecutar cálculos sobre los píxeles de un monitor, los cuales pueden considerarse como una grilla de nodos. A través de la paralelización la performance de las GPUs es alta, siendo capaces de procesar múltiples vértices y píxeles simultáneamente. Las simulaciones numéricas se producen en un menor tiempo en las GPU que sobre las CPU debido al alto paralelismo alcanzado por la GPU. \cite{rinaldi2011modelos}.


El lenguaje de programación que se utiliza en las placas gráficas es CUDA y fue desarrollado mediante la empresea NVIDIA. Se basa en el lenguaje de programación \textbf{C} con ciertas modificaciones para que los procesos sean en paralelo. Dichos procesos se especifican para que puedan ser lanzados en un número de \textit{blocks} y \textit{threads} de ejecución.






\section{Descripción y alcance del proyecto integrador}

Mediante el método LBM desarrollado por Fogliatto en \cite{fogliatto2018modelado}, \cite{fogliatto2019simulation} y \colorbox{green}{cita de ec energia} se realizó la implementación de un código numérico desarrollado en los lenguajes de programación \textsc{C} y \textsc{CUDA C}; para resolver problemas de transferencia de calor en flujos multifásicos con cambio de fase usando GPU de la forma más robusta posible.

La compilación del código se hizo mediante la herramienta multiplataforma \textsc{CMake}, por ser un proyecto complejo. Las bibliotecas que se generaron mediante \textsc{CMake} pueden ser utilizadas mediante el lenguaje \textsc{Python}. La utilización de \textsc{Python} es debida a la compatibilidad de aplicar  el código en las distintos sistemas operativos , como \textit{Linux} y \textit{Windows}. Además de que existe mucho soporte en éste lenguaje que cada vez se incrementa su número de usuarios por su facilidad de uso.

El proyecto se encuentra en \textbf{Git Hub} pudiendo ser descargado en \url{ https://github.com/efogliatto/LBCUDA_Test}.


La validación del código se hizo mediante los siguientes tres problemas:
\begin{itemize}
	\item \textit{construcción de Maxwell}
	\item \textit{estratificación de un fluido VdW con temperatura no uniforme}
	\item \textit{generación de burbujas sobre una superficie horizontal calefaccionada}
\end{itemize} 

las cuáles se encuentran explicadas en el Capítulo (\ref{cap4}).

Para los dos primeros problemas, se realizó una comparación en cuanto difieren los resultados de simple precisión y doble precisión del resultado analítico deseado. 

Se realizó un Speed Up, comparando el código en \textsc{C} y en \textsc{CUDA C}, ésto para ambas precisiones. Luego se realizó un Speed Up de las precisiones, tomando los códigos de \textsc{C} y \textsc{CUDA C} por separado.

Se implementó en \textsc{Python} parte del código realizado en \textsc{CUDA C}, exportando su biblioteca compilada mediante la biblioteca \textsc{PyCuda} de \textsc{Python}. También se realizó un Speed Up, ésta vez comparanco los resultados obtenidos en \textsc{CUDA C} con los de \textsc{Python}.


%%% Local Variables: 
%%% mode: latex
%%% TeX-master: "template"
%%% End: 
