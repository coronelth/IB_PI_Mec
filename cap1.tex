\chapter{Introducción}
\graphicspath{{figs/cap1/}}
\label{cap1}


En el estudio de la Mecánica de los Fluidos, es de importancia los problemas de transferencia de calor de flujos multi-fásicos con cambios de fase. 
Los problemas presentan la particularidad que se han realizado ensayos experimentales que estudian la fenomenología del mismo.
Para los tipos de problemas mencionados no se posee actualmente algún método numérico que concuerde con los ensayos realizados y que el calcule en un bajo tiempo .

Un ejemplo de aplicacion industrial del tipo de problema mencionado recide en el estudio de un reactor nuclear.
La transferencia de calor producida en las barras de elementos combustibles del núcleo provoca ebullición nucleada a partir de una dada potencia.
Se conoce bien cómo es la fenomenología a través del estudio realizado por la curva de \textit{Nukiyama}. 
EL problema actual para realizarlo numéricamente es que debido a la escala del fluido (\textit{mesoscópica}) con las técnicas convencionales de resolución de Mecánica de Fluidos Computacional(\textit{Computational Fluid Dynamicso} CFD)
no se obtienen resulados en un tiempo razonable.

\section{Descripción de las escalas de los fluidos}

Los fluidos como el aire y el agua son conocidos en nuestra vida diaria. Físicamente los fluidos están compuestos de un gran conjunto de átomos o moléculas que chocan unas con otras moviéndose  aleatoriamente. Usualmente las interacciones de las moléculas de un fluido son más débiles que las de un sólido, por ello mediante la aplicación de un pequeño esfuerzo al fluido, éste puede ser deformado de manera continua.


Usualmente la dinámica microscópica de las moléculas del fluido son muy complicadas y demuestran una fuerte inhomogeneidad y fluctuaciones.
Por otro lado la dinámica macroscópica del fluido, el cual es el resultado medio del movimiento de las moléculas en un medio homogéneo y continuo.
Mediante modelos matemáticos puede explicarse la dinámica de los fluidos , según el fluido observado y su fuerte dependencia del tamaño de su escala.

La mecánica de los fluidos se puede describir en tres niveles: macroscópica, mesoscópica y microscópica.
Generalmente el movimiento de un fluido puede ser descripto por tres tipos de modelos matemáticos acuerdo a lo que se observan en las distintas escalas, por ejemplo microscópico en modelos de escala molecular, teorías cinéticas en la escala mesoscópica y modelos continuos para escalas macroscópicas.


Los modelos matemáticos de los flujos de fluidos son :

\begin{itemize}
	\item ecuación de Newton para un basto número de moléculas (escala microscópica)
	\item ecuación de Boltzman para la función de distribución simple (escala mesoscópica)
	\item ecuación de Navier-Stokes para macro-variaciones de flujos (escala macroscópica)
\end{itemize}

que resultan extremadamente difíciles de resolver analíticamente de no ser imposibles.

La precisión de los modelos numéricos sin embargo han provisto de manera satisfactoria soluciones aproximadas de dichas ecuaciones.
Particularmente con la rapidez del desarrollo del software y hardware computacional y la tecnología, las simulaciones numéricas han comenzado a ser una importante metodología para la dinámica de fluidos.

El método más exitoso y popular de simulación de fluidos es la técnica Mecánica de Fluidos Computacional (\textit{Computational Fluid Dynamics} o CFD) , el cual está diseñado para resolver ecuaciones hidrodinámicas basadas en supociones de continuidad.
En CFD el dominio del flujo está compuesto en en conjunto de de sub-dominios con una malla computacional, las ecuaciones matemáticas son discretizadas usando algunos esquemas de discretización numérica como elementos finitos, volúmenes finitos o diferencia finitas; los cuales resultan en un sistema algebraico de sistemas de ecuaciones para las variables discretas del fluido asociadas a la malla computacional. 
Computacionalmente son llevados a cabo para encontrar una solución aproximada resolviendo el problema algebraico de sistemas de ecuaciones usando un algoritmo secuencial o paralelo.
 
La Simulación de Dinámica Molecular (\textit{Molecular Dynamic Simulation} o MDS) es una técnica en la cual el movimiento individual de átomos o moléculas del fluido son registrados para resolver las ecuaciones de Newton.
La principal ventaja de MDS es el aspecto macroscópico del fluido puede ser directamente conectado con el comportamiento molecular, en donde la estructura molecular y las interacciones microscópicas pueden ser descriptas de una manera directa.

El método de lattice Boltzmann ( \textit{lattice Boltzman Method} o LBM) es una aproximación del nivel mesoscópica. LBM estudia la micro-dinámica de partículas ficticias utilizando modelos cinéticos simplificados. La cual provee un camino alternativo de simular la mecánica de fluidos. La naturaleza de la cinética brinda distintas características de LBM tales que es claro el panorama de los procesos de advección y colisión de partículas de fluidos simuladas. La estructura del algoritmo es simple y de fácil implementación en las condiciones de contorno, además presenta un paralelismo natural. Todos éstos interesantes atributos hacen que LBM sea una potente herramienta numérica para la simulación de sistemas de fluidos envueltos en problemas físicos complejos.


%%% Local Variables: 
%%% mode: latex
%%% TeX-master: "template"
%%% End: 
