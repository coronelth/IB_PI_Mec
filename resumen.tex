\begin{resumen}%

Se implementó un código numérico para resolver problemas de mecánica de fluidos que posean transferencia de calor en flujos multifásicos con cambio de fase utilizando Unidades de Procesamiento Gráfico (GPU). Se utiliza la arquitectura de las GPU por su diseño, la cual realiza las ejecuciones de forma paralela y mediante ello reducir los costos computacionales de los algoritmos implementados.

El modelo de lattice Boltzmann (LBM) pseudo-potencial es el utilizado para la resolución de los problemas descriptos, el cual resuelve de manera indirecta Ecuaciones Diferenciales Ordinarias (EDO) no lineales por medio de ecuaciones lineales más sencilla, y tiene la ventaja de ser altamente paralelizables. En particular, para el presente trabajo se utilizó un LBM con dos ecuaciones pseudo-potencial con operador MRT.

El código realizado se implementó en los lenguajes de programación \textsc{C}, \textsc{Cuda C} y \textsc{Python}, en donde la compilación se llevo a cabo mediante el compilador \textsc{CMake} con la posibilidad de seleccionar su tipo de variables (\textit{float} o \textit{double}). El proyecto se planificó para que através de la compilación de los \textit{kernels} de \textsc{Cuda C} puedan ser implementados en un \textit{script} de \textsc{Python} con su módulo \textsc{PyCuda}.

La validación del código se realizó en CPU y GPU para diferentes problemas numéricos. Se realizó una comparación entre los tiempos de cálculo del código en los diferentes lenguajes utilizados, en donde se obtuvo que la implementación en \textsc{Cuda C} mejora de uno a dos ordenes de magnitud a la implementación equivalente en \textsc{C}. Por otro lado, al realizar la comparación del código implementadoo en \textsc{PyCuda} con respecto al equivalente en \textsc{Cuda C}, se obtuvo que el primero es alrededor de un 10 \% más lento que el segundo. En cuanto a la comparación de la conveniencia de utilización de tipos de variable \textit{float} o \textit{double}, se obtuvo que la diferencia porcentual entre las  precisiones es cercana al 0,003 \%, y que el tiempo de cálculo en variables tipo \textit{double} es aproximadamente un 25 \% mayor que en \textit{float}.



\end{resumen}

\begin{abstract}%

A numerical code was implemented to solve fluid mechanics problems that have heat transfer in multiphase flows with phase change using Graphics Processing Units (GPU). The GPU architecture is used by its design, which performs executions in parallel and thereby reduce the computational costs of the algorithms implemented. 

The pseudo-potential Lattice Boltzmann model is the one used to solve the problems described, which indirectly solves nonlinear Ordinary Differential Equations (EDO) by other  linear equations more simpler, and has the advantage of being highly parallel. In particular, for this work an LBM with two pseudo-potential equations was used with MRT operator.

Code made was implemented on the programming languages \textsc{C}, \textsc{Cuda C} and \textsc{Python}, where the compilation was carried out using the compiler \textsc{CMake} with the posibility to select the variable type between float and double. The project was planned to go through the compilation of the \textsc{Cuda C} kernels can be implemented in a
\textsc{Python} script using his library \textsc{PyCuda}. 

Code validation was performed on CPU and GPU for different numeric issues. A comparison was made between the calculation times of the code on the different programming languages implemented, where it was obtained that the implementation on \textsc{Cuda C} improvement from one to two orders of magnitude to the equivalent implementation on \textsc{C}. On the other hand, when comparing the code implemented on PyCuda with respect to the equivalent on Cuda C, the former was found to be about 10\% slower than the latter. As for comparing the desirability of using float or double variable types, the percentage difference between the accuracy was obtained is close to 0.003\%, and calculation time in variable type double is approximately 25\% greater than float.

\end{abstract}


%%% Local Variables: 
%%% mode: latex
%%% TeX-master: "template"
%%% End: 
