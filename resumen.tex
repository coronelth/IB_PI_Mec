\begin{resumen}%

En este trabajo se implementó un código numérico para resolver problemas de mecánica de fluidos con transferencia de calor en flujos multifásicos con cambio de fase utilizando Unidades de Procesamiento Gráfico (GPU). Se utilizó la arquitectura de las GPU por su diseño, el cual permite realizar la ejecución de instrucciones con elevado nivel de paralelismo, y así reducir los costos computacionales de los algoritmos involucrados.

El modelo de lattice Boltzmann (LBM) pseudopotencial es el utilizado para abordar los problemas de interés, el cual resuelve de manera indirecta ecuaciones diferenciales no lineales por medio de ecuaciones lineales más sencilla, y además tiene la ventaja de resultar altamente paralelizable. En particular, para el presente trabajo se utilizó un LBM con dos ecuaciones pseudopotencial con operador MRT.

El código realizado se implementó en los lenguajes de programación \textsc{C}, \textsc{Cuda C} y \textsc{Python}, en dónde el desarrollo del proyecto se llevó a cabo mediante la herramienta \textsc{CMake} con la posibilidad de seleccionar su tipo de variables (\textit{float} o \textit{double}). El proyecto se planificó para que los \textit{kernels} de \textsc{Cuda C} compilados puedan ser implementa- dos en un \textit{script} de \textsc{Python} con su módulo \textsc{PyCuda}.

La validación del código se realizó en CPU y GPU para diferentes problemas numéricos. Se realizó una comparación entre los tiempos de cálculo del código en los diferentes lenguajes utilizados, en donde se obtuvo que la implementación en \textsc{Cuda C} llega a ser 23 veces más rápida que el equivalente en \textsc{C} para un único proceso, en las GPU y CPU evaluadas durante este Proyecto Integrador. Por otro lado, al realizar la comparación del código implementado en \textsc{PyCuda} con respecto al equivalente en \textsc{Cuda C}, se obtuvo que el primero es alrededor de un 10 \% más lento que el segundo. En cuanto a la comparación de la conveniencia de utilización de tipos de variable \textit{float} o \textit{double}, se obtuvo que la diferencia porcentual entre las  precisiones es cercana al 0,003 \%, y que el tiempo de cálculo en variables tipo \textit{double} es aproximadamente un 25 \% mayor que en \textit{float}.



\end{resumen}

\begin{abstract}%

In this work, a numerical code for multiphase flow with phase change and heat transfer was implemented in Graphics Processing Units (GPU). The GPU architecture is used by its design, which performs executions in parallel and thereby reduce the computational costs of the algorithms implemented. 

A pseudopotential lattice Boltzmann model (LBM) was used, which indirectly recovers the solution of nonlinear differential equations from the solution of a much simpler equation with a highly paralellizable algorithm. In particular, a pseudopotential LBM with MRT operator was implemented.

Code was implemented under \textsc{CMake} on the programming languajes \textsc{C}, \textsc{CUDA C} and \textsc{Python}, with the posibility to select variable types at compilation time. \textsc{CUDA C} kernels were also compiled to be used within \textsc{Python} scripts by means of the \textsc{PyCuda} module.

Code validation was performed on CPU and GPU for different numerical benchmarks. A comparison was made between the calculation times of the code on the different programming languages, where it was obtained that the implementation on \textsc{Cuda C} improvement from one to two orders of magnitude to the equivalent implementation on \textsc{C}. On the other hand, when comparing the code implemented on PyCuda with respect to the equivalent on Cuda C, the former was found to be about 10\% slower. Numerical solutions computed with different variable types (i.e. float or double) showed relative differences lower than 0.003\%,  while the computational time is increased by 25\% between each case.

\end{abstract}


%%% Local Variables: 
%%% mode: latex
%%% TeX-master: "template"
%%% End: 
