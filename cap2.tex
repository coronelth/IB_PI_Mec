\chapter{Modelo de lattice-Boltzmann}
\graphicspath{{figs/cap2/}}
\label{cap3}

\section{LBM Multifásicos}

En éste capítulo se presentara cuál es el modelo de LB para la resolución de problemas de transferencia de calor con flujos multifásicos y cambio de fase.

Generalmente es difícil resolver las ecuaciones de la mecánica de los fluidos. Las soluciones analíticas de los problemas que pueden ser halladas son escasas, como el caso de flujos \textit{Couette} o \textit{Poisueuille}. Problemas que contengan una geometría más compleja u otras condiciones de contorno; poseen una gran dificultad para encontrar la solución de las ecuaciones de la mecánica de fluidos, si es que el problema la posee. Debido a ello las soluciónes se obtienen numéricamente \cite{kruger2017lattice}(sec 3.1). Es de importancia el desarrollo de métodos numéricos que resuelvan los problemas de forma paralela para así reducir el tiempo de cálculo.

Debido a que los problemas que se plantean resolver en el presente trabajo son de transferencia de calor en flujos multifásicos con cambio de fase, la escala de del fluido que se decide adoptar es la mesoscópica. Considerando la escala y siendo un problema multifásico, se opta por resolver numéricamente mediante LBM. 

Los LBM multifásicos tienen la particularidad de que la frontera entre las fases no son resueltas con exactitud \textbf{No todos. Esto ocurre en los pseudopotenciales}. Dicha interfaz es representada de forma difusa con un cierto tamaño en la grilla, siendo una importante ventaja para el cálculo puesto que la interfaz no debe ser reconstruida.\cite{parrill2019reviews}.

Los mayoris de los modelos para resolver los flujos multifásicos son tres, clasificados en cuatro categorias generales : \textit{color gradient}, \textit{Shan Chen model} o \textit{pseudopotential} y \textit{Free-energy}. \textbf{Falta phase-field}. 

\begin{itemize}
	
	\item \textit{Color gradient} fue el primer modelo de LBME para flujos multifásicos siendo desarrollado por Gunstensen \cite{gunstensen1991lattice}. Las fases y las interacciones entre las partículas son denotadas mediante diferentes colores. Por medio del modelado local del gradiente de color que se encuentra asociado a la diferencia de las densidades de las dos fases, se conoce como es la segregación y separación de las fases.
	
	\item \textit{Shan Chen model} o \textit{pseudopotential} surge del modelo de colores donde la redistribución de las particulas de fluido está basada en \textit{color gradient}. \textbf{No entendi esta primer oracion}La fuerza de interacción proviene de la diferencia entre las fuerzas promedio del modelo molecular  entre ambos lados de la interfaz. Shan and Chen \cite{shan1993lattice} presentaron un modelo de LBE (referenciado como modelo SC) que podría representar la interacción entre partículas fluidas de forma más precisa y directa introduciendo un pseudo potencial. 
	
	\item \textit{Free-energy} es un modelo un tipo alternativo de LBE desarrollado por Swift \cite{swift1995lattice} para modelos multifásicos/multicomponentes basado en la teoría de energía libre (\textit{free-energy}). Debido a que la fenomenología representada en los modelos de LBE de color y pseudo potenciales son la misma. \textbf{se corta la oracion}La idea básica del nuevo método es realizar una función de distribución de equilibrio basada en funciones de energía libre, en las cuales se incorpora el tensor de presión termodinámico.\cite{guo2013lattice}(sec 7) .
	
\end{itemize}

\section{Ecuación de estado y fluido Van Der Waals}

La Ley de gases ideales \ref{eq:gas_ideal} es una Ecuación de estado (Ecuation of State o EOS). Donde P es la presión (atm) , V es el volúmen, n es el número de moles, R la constante de los gases ideales y T la temperatura. La Ley se encuentra caracterizada para densidades bajas brindando una relación entre la densidad, presión y temperatura que posee un fluido.

\begin{align}

    P V = n R T
    \label{eq:gas_ideal}
\end{align}




\section{Modelo pseudopotencial}

El modelo pseudopotencial es el adoptado en el presente trabajo para abordar los problemas descriptos en \ref{cap4}, por ello es descripto con detalle en la siguiente sección.


La obtención de la ecuación de lattice-Boltzmann (LBE) a través de la ecuación de Boltzmann se encuentra descripta en \cite{kruger2017lattice} (sec 3.4 y 3.5 ). Primero se explica la discretización en el espacio de velocidades, en el segundo la del espacio físico, temporal e integración de las características.


Para entender el LBM, hay que saber que el mismo deriva de discretizar la ecuación de Boltzmann en los espacios de velocidad, físico y temporal.  

\textbf{Acordate que ya vas a decir algo de esto en la Introduccion}
Un LBM es el \textit{DdQq} (\textit{d} - dimensional , \textit{q} velocidades). La figura \ref{fig:D1Q3_D2Q9} muestra como es el conjunto de velocidades del modelo D1Q3 y D2Q9.

En el presente trabajo el LBM es $D2Q9$. 

\begin{figure}[h]
	\centering
    \includegraphics[width=.8\textwidth]{figs/cap2/D1Q3_D2Q9}
	\caption{Conjunto de velocidades de los modelos D1Q3 y D2Q9. \cite{kruger2017lattice}}
	\label{fig:D1Q3_D2Q9}	
\end{figure}

El problema físico a resolver cuenta con una región, la cuál se le realizará un mallado para discretizar el espacio. El nodo i - ésimo de la malla posee las coordenadas ${\bar{X}}_{i} = (x,y,z)$, a su vez densidad $\rho_{i}$ y temperatura $T_{i}$. La velocidad en el nodo tiene las componentes ${\bar{U}}_{i} = ({U}_{ix},{U}_{iy},{U}_{iz})$. El espacio de velocidades indica como es la propagación de las propiedades en la grilla. Dicha velocidad de grilla $\mathbf{e}_{i}$ posee $\alpha$ componentes donde $\alpha = q $. 

Para el modelo D2Q9 la figura \ref{fig:D1Q3_D2Q9} muestra un esquema de las velocidades de grilla del nodo i - ésimo y la Ec. (\ref{eq:velgrilla}) el valor adoptado.


\begin{equation}
    {\mathbf{e}}_{i} =  
    \left( \begin{array}{c} 
                e_{i0} \\ e_{i1}\\ e_{i2}\\ e_{i3}\\ e_{i4}\\ e_{i5}\\
                e_{i6}\\ e_{i7}\\ e_{i8}\\
            \end{array}
    \right) =
    \left( \begin{array}{c} 
        (0,0,0) \\ (1,0,0) \\ (0,1,0) \\(-1,0,0) \\ (0,-1,0) \\ (1,1,0) \\
        (-1,1,0) \\ (-1,-1,0) \\ (1,-1,0)\\ 
    \end{array}
    \right) 
    \label{eq:velgrilla}
\end{equation}

A su vez cada uno de los nodos de la grilla posee un campo de distribución $f_{i}$ también con $\alpha$ componentes. Mediante esta funcion de distribucionel campo \textbf{que campo} se puede obtener las variables macroscópicas del problema.

Para el desarrollo de la solución de los problemas que se describieron en \ref{cap2}, se utiliza el modelo pseudopotencial de dos ecuaciones con operador MRT, siendo desarrollado por Fogliatto \cite{fogliatto2018modelado} y \colorbox{green}{citar eq energia}. \textbf{Me parece que no iria este parrafo}


\textbf{No decis nada de las ecuaciones de estado? Es algo clave de los pseudopotenciales. No digo una descripcion detallada, pero por lo menos decir que se basan en ecuaciones de estado para el calculo de fuerzas de interaccion, y que esta EOS determina el tensor de presion recuperado, las densidades de coexistencia y otras propiedades como calor latente. 
Separacion automatica de fases?}


\section{Modelo pseudopotencial de dos ecuaciones con operador MRT}

Los modelos 

El LBM surgió históricamente como una evolución de los modelos de lattice gas cellular
automata (LGCA) [18–21], las primeras aproximaciones fueron propuestas por McNamara
et al. [22, 23] e Higuera et al. [24] con la intención explı́cita de solucionar problemas
de ruido estadı́stico del LGCA. La idea básica era remplazar los números booleanos de
ocupación por sus correspondientes promedios en ensamble. Con posterioridad, He et
al. [16] derivaron formalmente la LBE desde la ecuación continua de Boltzmann (1.1),
demostrando que el LBM es independiente del LGCA. En la literatura contemporánea,
la conexión entre el LBM y el LGCA solo se menciona por motivos históricos, siendo
entendido el LBM como una discretización particular de la BE.
Siguiendo la idea del BGK, la utilización de operadores de colisiones del tipo relajación
al equilibrio es extensamente utilizada en el LBM, incluso desde las primeras propuesta
[22, 24, 25]. Existen diferentes modelos de colisiones en el LBM, se pueden destacar: LBM-
BGK (o LBM-SRT) [16–18, 26] basado en el modelo BGK o de tiempo de relajación simple
(SRT); LBM-MRT [25, 27–32] donde se utiliza un operador de colisiones con múltiples
tiempos de relajación; y LBM-TRT [33, 34] con dos tiempos de relajación. Siendo los
modelos SRT y TRT casos especiales del LBM-MRT. Otro modelo frecuentemente utilizado
es el Entropic lattice Boltzmann (ELBM) [35, 36]. Los modelos ELBM y LBM-TRT no se
utilizan en éste trabajo.

En este capı́tulo se presentan detalles de una discretización particular, en el tiempo y
en el espacio de fases, de la ecuación de Boltzmann (BE). Se incluye una derivación formal
de la ecuación de lattice Boltzmann (LBE) desde la BE utilizando el modelo de colisiones
BGK. Se presenta una ecuación generalizada de lattice Boltzmann que da origen al LBM
con múltiples tiempos de relajación (MRT),





\subsection{Ecuación hidrodinámica}


La resolución de las ecuaciones hidrodinámicas pueden analizarse mediante la evolución de una función de distribución dada por \cite{li2013lattice}: 	

\begin{equation}
    \mathbf{f}(\mathbf{x} + \mathbf{e} \> \delta_{t} , t + \delta_{t}) = \mathbf{M}^{-1} \left[ \mathbf{m} - \mathbf{\Lambda}(\mathbf{m} - \mathbf{m}^{(eq)}) + \delta_{t} \left( \mathbf{I} - 0,5 \mathbf{\Lambda} \right) \mathbf{\bar{S}}  \right]_{(\mathbf{x},t)} 
    \label{eq:fieldmom}
\end{equation}

donde $\textit{f}_{\alpha}$ es la distribución de densidad en el espacio de poblaciones, t el tiempo, $\mathbf{x}$ la posición espacial, \textit{\textbf{e}} las velocidades discretas a lo largo de la direcciones $\alpha$ y $\delta_{t}$ el paso de tiempo. En este caso, la notación usada en la Ec. (\ref{eq:fieldmom}) implica que la compontente $\alpha$-ésima del miembro izquierdo está dada por $f_{\alpha}(x + e_{\alpha} \delta_{t}  , t + \delta_{t} )$ . El miembro derecho de la Ec. (\ref{eq:fieldmom}) corresponde a la etapa de post-colisión definida en el espacio de momentos, donde \textbf{M} es una matriz de transformación ortogonal, $\mathbf{m} = \mathbf{M} \cdot \mathbf{f}$ , $\mathbf{m}_{eq} = \mathbf{M} \cdot \mathbf{f}_{eq}$ , \textbf{I} el tensor identidad y $\mathbf{\bar{S}} = \mathbf{M} \mathbf{S}$ el término de fuente. Para una grilla D2Q9, $ \Lambda$ es una matriz diagonal:
    
\begin{align}
    \mathbf{\Lambda}  = diag ( {\tau_{\rho }}^{-1},{\tau_{e}}^{-1},{\tau_{\zeta }}^{-1},{\tau_{j}}^{-1},{\tau_{q}}^{-1},{\tau_{j}}^{-1},{\tau_{q}}^{-1},{\tau_{\nu }}^{-1},{\tau_{\nu}}^{-1}) 
    \label{eq:lambda}
\end{align}

mientras que la distribución de equilibrio está dada por:
        

\begin{align}
    m_{eq} =  \rho  \left( 1, - 2 + 3 {|u|}^{2} , 1 - 3{|u|}^{2} , u_{x} , - u_{x} , u_{y} , - u_{y} , {u_{x}}^{2} - {u_{y}}^{2} , u_{x} u_{y} \right) 
    \label{eq:m}
\end{align}


donde la densidad y velocidad macroscópica se obtienen mediante:

\begin{equation}
        \rho = \sum_{\alpha} f_{\alpha}
\end{equation}

\begin{equation}
    \rho \> \mathbf{u} = \sum_{\alpha} {\mathbf{e}}_{\alpha} \> f_{\alpha} + 0,5 \> {\delta}{t} \> \mathbf{F}
\end{equation}

En este caso, $ {\mathbf{F}} = (F_{x} , F_{y} ) = {\mathbf{F}}_{b} + {\mathbf{F}}_{int} $ es la fuerza total, ${\mathbf{F}}_{b}$ la fuerza volumétrica y ${\mathbf{F}}_{int}$ representa la fuerza de interacción que actúa sobre el sistema a través de un potencial $\psi(x)$:
    
\begin{equation}
    {\mathbf{F}}_{int} = - G \> \psi(\mathbf{x}) \sum_{\alpha=1}^{N} w({|{\mathbf{e}}_{\alpha}|}^{2}) \> \psi (\mathbf{x} + {\mathbf{e}}_{\alpha} \> \delta_{t}) \> {\mathbf{e}}_{\alpha} 
    \label{eq:fint}
\end{equation}

En la Ec. \ref{eq:fint}, G corresponde a la intensidad de interacción, $w({|{\mathbf{e}}_{\alpha}|}^{2})$ son los pesos correspondientes a una grilla D2Q9 y $\psi$ está dado por:

\begin{equation} 
    \psi(\rho) = \sqrt{\frac{2 (p_{EOS} - \rho {c_{s}}^{2})}{G {c}^{2}}}
\end{equation}

En el presente trabajo se adopta una EOS de VdW \textbf{Que es VdW? Que es EOS?}:

\begin{equation}
    \rho_{EOS} = \frac{\rho r t}{1- \rho B} - A {\rho}^{2}
\end{equation}

donde \textit{a} y \textit{b} son parámetros que determinan los valores críticos de temperatura, presión y densidad, y fueron fijados en $\textit{a} = 1$ y $\textit{b} = 4$. Finalmente, la fuerza de interacción se incorpora en la etapa de colisión mediante un término de fuente apropiado:

\begin{equation}
    \bar{S} = 
    \left[ \begin{array}{c} 
        0\\
        6 \mathbf{u}\cdot \mathbf{F} + \frac{12 \sigma {|{\mathbf{F}_{int}|}}^{2} }{{\psi}^{2} \delta_{t} (\tau_{e} - 0,5)}\\
        6 \mathbf{u}\cdot \mathbf{F} - \frac{12 \sigma {|{\mathbf{F}_{int}|}}^{2} }{{\psi}^{2} \delta_{t} (\tau_{\zeta } - 0,5)}\\
        F_{x}\\
        -F_{x}\\
        F_{y}\\
        -F_{y}\\
        2(u_{x} F_{x} - u_{y} F_{y} )\\
        (u_{x} F_{x} + u_{y} F_{y} )\\              
    \end{array}
    \right]    
\end{equation}

donde $\sigma$ es un parámetro libre que es utilizado para ajustar el problema de inconsistencia termodinámica, es decir, la diferencia entre las densidades de cada fase obtenidas en la simulación y aquellas determinadas por la EOS correspondiente.

\textbf{Se podria agregar aca cual es la ecuacion macroscopica recuperada. Esta al final del paper del International Journal of… Solo para decir que despues de aplicar todo lo anterior, la densidad y velocidad van a seguir la descripcion dada por esas ecuaciones.}

\subsection{Ecuación energía}

Se debe acoplar una segunda ecuación LB para poder incorporar transferencia de calor al modelo de \cite{li2013lattice}. En particular, puede adoptarse una segunda distribución de poblaciones \textit{g} bajo un operador de colisión MRT:

\begin{equation}
    \mathbf{g^*}(\mathbf{x} + \mathbf{e} \delta_{t} ,t + \delta_{t}) = \mathbf{M}^{-1} \left[ \mathbf{n} - \mathbf{Q}(\mathbf{n} - \mathbf{n}^{(eq)}) + \delta_{t} \left( I - 0,5 Q \right) \hat{\Gamma}  \right]_{(\mathbf{x},t)}
    \label{eq:fieldenergy}
\end{equation}

donde $n = M g$ y $\hat{\Gamma}$es una fuente en el espacio de momentos. La temperatura macroscópica \textit{T} puede recuperarse mediante:

\begin{equation}
    T = \sum_{\alpha} g_{\alpha} + \frac{1}{2} \delta_{t} {\hat{\Gamma}}_{0}
\end{equation}

La matriz de coeficientes de relajación \textbf{Q} está compuesta por una parte diagonal

\begin{equation}
    \textit{diag} (Q) = {( q_{0} , q_{1} , q_{2} , q_{3} , q_{4} , q_{5} , q_{6} , q_{7} , q_{8} )}^{T}
\end{equation}

pero, a diferencia de $\lambda$ , presenta elementos extra diagonales no nulos dados por:

\begin{equation}
    Q_{3,4} = q_{4} \left( \frac{q_{3}}{2} - 1 \right)
\end{equation}

\begin{equation}
    Q_{5,6} = q_{6} \left( \frac{q_{5}}{2} - 1 \right)
\end{equation}

Si se define una distribución de equilibrio \textit{n} como:

\begin{equation}
    {\mathbf{n}}_{eq} = T { \left( 1, \alpha_{1}, \alpha_{2}, u_{x}, -u_{x}, u_{y}, -u_{y}, 0, 0 \right) }^{T}
\end{equation}

y el término fuente mediante:

\begin{equation}
    \hat{\Gamma} = {( s, 0, 0, 0, 0, 0, 0, 0, 0 )}^{T}
\end{equation}

con 

\begin{equation}
    s = \frac{\chi}{\rho} \bigtriangledown T \cdot \bigtriangledown \rho + T \left[ 1 - \frac{1}{\rho c_{\nu}} {\left( \frac{\delta p_{EOS}}{\delta T} \right)}_{\rho} \right] \bigtriangledown \cdot \mathbf{u}
\end{equation}

entonces la Ec. \ref{eq:fieldenergy} puede recuperar adecuadamente la ecuación macroscópica derivada por \cite{markus2011simulation}:

\begin{equation}
    \delta_{t} T + \bigtriangledown \cdot ( \mathbf{u} T ) = \chi {\bigtriangledown }^{2} T + s
\end{equation}

Por simplicidad, en este trabajo se considera una difusividad térmica $\chi$ constante, la cual queda determinada mediante los factores de relajación de \textbf{Q} y los parámetros libres de ${\textbf{n}}_{eq}$ :

\begin{equation}
    \chi = \delta_{t} \left( \frac{1}{q_{3}} - \frac{1}{2} \right) \left( \frac{ 4 + 3 \alpha_{1} + 2 \alpha_{2}}{6} \right)
\end{equation}


\textbf{Falta agregar las condiciones de contorno. Para la hidrodinamica usamos el esquema de Zuo-He, o bounce-back de componentes de no equilibrio, en las paredes
Zou, Q., He, X., 1997. On pressure and velocity boundary conditions for the lattice Boltzmann BGK model. Physics of Fluids 9, 1591. https://doi.org/10.1063/1.869307
y para la ecuacion de energia usamos el esquema de Inamuro para recuperar la temperatura en las paredes
Inamuro, T., Yoshino, M., Inoue, H., Mizuno, R., Ogino, F., 2002. A Lattice Boltzmann Method for a Binary Miscible Fluid Mixture and Its Application to a Heat-Transfer Problem. Journal of Computational Physics 179, 201–215. https://doi.org/10.1006/jcph.2002.7051}

%%% Local Variables: 
%%% mode: latex
%%% TeX-master: "template"
%%% End: 