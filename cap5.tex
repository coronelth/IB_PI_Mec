\chapter{Conclusiones}
\graphicspath{{figs/cap4/}}
\label{cap5}

En el presente trabajo se realizó un código numérico desarrollado mediante los lenguajes de programación C y CUDA C, para resolver problemas de transferencia de calor en flujos multifásicos con cambio de fase.
El modelo utilizado es el modelo de lattice Boltzmann de dos (2) ecuaciones pseudopotenciales con operador MRT, siédno el mismo del tipo D2Q9.
La validación del código se realizó en dos (2) GPU diferentes, siendo NVIDIA Geforce GTX 760 y  NVIDIA Geforce GTX 970; para simple precisión y doble precisión.
La validación se realizó por medio de tres (3) problemas físicos, siendo ellos:

\begin{itemize}
    
    \item la Construcción de Maxwell, el cuál obtiene las densidades de coexistencia de fases de un fluido 

    \item la estratificación de un fluido Van der Waals con temperatura no uniforme; siéndo el problema unidimensional con campo gravitatorio y temperaturas fijas en los extremos.

    \item la generación de burbujas en una placa horizontal calefaccionada.

\end{itemize}

\section{Construcción de Maxwell}

Para el problema de la Construcción de Maxwell, se reprodució el resultado que obtuvo Fogliatto en [], para el cuál el valor del parámetro $\sigma = 0,125$ del operador MRT es el que ajusta mejor la curva de coexistencia de fases para un fluido con la Ecuación de estado de Van der Waals de parámetros $ a = 0,5 $ y $ b = 4,0 $. 

Para la GPU NVIDIA Geforce GTX 760 en simple precisión se obtuvo una ganancia del código realizado en \textbf{CUDA C} es de xx veces con respecto al código de \textbf{C} para un tamaño de xx bloques y la cantidad de xxx node de malla. Por el comportamiento que se observó, si sigue aumentando el número de nodos que posee la malla, la tendencia de 


\section{Trabajo futuro}

La siguiente línea de desarrollo para éste trabajo son la mejora contínua de las funciones que posee el código realizado, por lo que se puede seguir haciendo \textit{profiling} del mismo ya sea en la utilización de las memorias que posee cada uno de los \textit{threads} 