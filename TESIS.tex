%%%%%%%%%%%%%%%%%%%%%%%%%%%%%%%%%%%%%%%%%%%%%%%%%%%%%%%%%%%%%%%%%%%%%%%%%%%%%%%%
% \documentclass[12pt,papel,twoside]{ibtesis}
%\documentclass[12pt,screen,oneside,pagebackref]{ibtesis}
\documentclass[12pt,papel,oneside]{ibtesis}
%\documentclass[12pt,papel,preprint,oneside]{ibtesis}


%%%%%%%%%%%%%%%%%%%%% Paquetes extra %%%%%%%%%%%%%%%%%%%%%%%%%%%%%%%%%%%%%%%%%%%
% Por conveniencia: aqu\'{\i} puede cargar todos los paquetes y definir los comandos 
% que necesite
\usepackage{ibextra}
\usepackage[T1]{fontenc}

\usepackage{graphicx}
\usepackage{subcaption}
\usepackage{multicol}
\usepackage{wrapfig}
\usepackage{adjustbox}
\usepackage{latexsym}
\usepackage{tabulary,tabularx}
\usepackage{units}
\usepackage{fancyvrb}
\usepackage{multirow}
\usepackage{rotating}
\usepackage{array}
\usepackage{gensymb}
\usepackage{float}
\usepackage{enumitem}
\usepackage{mathrsfs}

%\usepackage{hyperref}
%\hypersetup{
%    colorlinks,
%    citecolor=black,
%    filecolor=black,
%    linkcolor=black,
%    urlcolor=black
%}

%%%%%%%%%%%%%%%%%%%%%%%%%%%%%%%%%%%%%%%%%%%%%%%%%%%%%%%%%%%%%%%%%%%%%%%%%%%%%%%%
%%%%%%%%%%%%%%%%%%%%% Informacion sobre la tesis %%%%%%%%%%%%%%%%%%%%%%%%%%%%%%%
\title{Implementacióon de modelos lattice-Boltzmann para flujo multifásico con transferencia de calor en unidades de procesamiento gráfico}
\author{Thomás Coronel}
\director{Mgter. Ezequiel Fogliatto}
\codirector{Ing. Pablo Argañaraz}
\carrera{Proyecto Integrador de la Carrera de Ingeniería Mecánica}
\grado{}
\laboratorio{Mecánica Computacional\\
%Departamento Física de Neutrones\\
Centro Atómico Bariloche}
\jurado{ Dr. René Cejas Bolecek (Instituto Balseiro - Centro Atómico Bariloche) \\ 
Dr. Flavio Colavecchia (Instituto Balseiro - Centro Atómico Bariloche))\\ }
\palabrasclave{lattice-Boltzman,transferencia de calor,LBM}%formato de Tesis, Lineamientos de escritura, Instituto Balseiro}
\keywords{lattice-Boltzman,heat transfer, LBM}%Thesis format, Templates, Instituto Balseiro}
% Si queremos poner la fecha manualmente:
\date{Junio de 2020}

%%%%%%%%%%%%%%%%%%%%%%%%%%%%%%%%%%%%%%%%%%%%%%%%%%%%%%%%%%%%%%%%%%%%%%%%%%%%%%%%
%\titlepagefalse % Si no quiere compilar la portada descomente esta linea
%\includeonly{resumen,cap1} % Compilar s\'{o}lo estos archivos 
\graphicspath{{figs/}} % Lugar donde encontrar las figuras generales (se puede poner uno en cada cap{\'{\i}}tulo)
%%%%%%%%%%%%%%%%%%%%%%%%%%%%%%%%%%%%%%%%%%%%%%%%%%%%%%%%%%%%%%%%%%%%%%%%%%%%%%%%


\begin{document}

% Dentro del environment 'preliminary' va:
% la dedicatoria, resumen, abstract, indices

\begin{preliminary}

% Escriba su dedicatoria
\dedicatoria{
	Inserte su dedicatoria
}

%%% \'{I}ndices %%%%

\begin{resumen}%

En este trabajo se implementó un código numérico para resolver problemas de mecánica de fluidos con transferencia de calor en flujos multifásicos con cambio de fase utilizando Unidades de Procesamiento Gráfico (GPU). Se utilizó la arquitectura de las GPU por su diseño, el cual permite realizar la ejecución de instrucciones con elevado nivel de paralelismo, y así reducir los costos computacionales de los algoritmos involucrados.

El modelo de lattice Boltzmann (LBM) pseudopotencial es el utilizado para abordar los problemas de interés, el cual resuelve de manera indirecta Ecuaciones Diferenciales Ordinarias (EDO) no lineales por medio de ecuaciones lineales más sencilla, y además tiene la ventaja de resultar altamente paralelizable. En particular, para el presente trabajo se utilizó un LBM con dos ecuaciones pseudopotencial con operador MRT.

El código realizado se implementó en los lenguajes de programación \textsc{C}, \textsc{Cuda C} y \textsc{Python}, en dónde el desarrollo del proyecto se llevó a cabo mediante la herramienta \textsc{CMake} con la posibilidad de seleccionar su tipo de variables (\textit{float} o \textit{double}). El proyecto se planificó para que los \textit{kernels} de \textsc{Cuda C} compilados puedan ser implementados en un \textit{script} de \textsc{Python} con su módulo \textsc{PyCuda}.

La validación del código se realizó en CPU y GPU para diferentes problemas numéricos. Se realizó una comparación entre los tiempos de cálculo del código en los diferentes lenguajes utilizados, en donde se obtuvo que la implementación en \textsc{Cuda C} llega a ser 23 veces más rápida que el equivalente en \textsc{C} para un único proceso, en las GPU y CPU evaluadas durante este Proyecto Integrador. Por otro lado, al realizar la comparación del código implementadoo en \textsc{PyCuda} con respecto al equivalente en \textsc{Cuda C}, se obtuvo que el primero es alrededor de un 10 \% más lento que el segundo. En cuanto a la comparación de la conveniencia de utilización de tipos de variable \textit{float} o \textit{double}, se obtuvo que la diferencia porcentual entre las  precisiones es cercana al 0,003 \%, y que el tiempo de cálculo en variables tipo \textit{double} es aproximadamente un 25 \% mayor que en \textit{float}.



\end{resumen}

\begin{abstract}%

In this work, a numerical code for multiphase flow with phase change and heat transfer was implemented in Graphics Processing Units (GPU). The GPU architecture is used by its design, which performs executions in parallel and thereby reduce the computational costs of the algorithms implemented. 

A pseudopotential lattice Boltzmann model (LBM) was used, which indirectly recovers the solution of nonlinear differential equations from the solution of a much simpler equation with a highly paralellizable algorithm. In particular, a pseudopotential LBM with MRT operator was implemented.

Code was implemented under \textsc{CMake} on the programming languajes \textsc{C}, \textsc{CUDA C} and \textsc{Python}, with the posibility to select variable types at compilation time. \textsc{CUDA C} kernels were also compiled to be used within \textsc{Python} scripts by means of the \textsc{PyCuda} module.

Code validation was performed on CPU and GPU for different numerical benchmarks. A comparison was made between the calculation times of the code on the different programming languages, where it was obtained that the implementation on \textsc{Cuda C} improvement from one to two orders of magnitude to the equivalent implementation on \textsc{C}. On the other hand, when comparing the code implemented on PyCuda with respect to the equivalent on Cuda C, the former was found to be about 10\% slower. Numerical solutions computed with different variable types (i.e. float or double) showed relative differences lower than 0.003\%,  while the computational time is increased by 25\% between each case.

\end{abstract}


%%% Local Variables: 
%%% mode: latex
%%% TeX-master: "template"
%%% End: 


\tableofcontents                %\'{I}ndice

\begin{abreviaturas}
	
	Inserte aquí las abreviaturas de símbolos
	
\end{abreviaturas}

\end{preliminary}


% Podemos usar cualquiera de los dos comandos: \input o \include para incluir el texto
\chapter{Introducción}
\graphicspath{{figs/cap1/}}
\label{cap1}


En el estudio de la Mecánica de los Fluidos, es de importancia los problemas de transferencia de calor de flujos multi-fásicos con cambios de fase. 
Para dichos problemas cuyas aplicaciones industriales entre otras son la transferencia de calor  que se produce en las barras de elementos combustibles del núcleo de un reactor nuclear.
Se realizan mediciones de las variables físicas involucradas en el estudio de dicho problema, en base a las mediciones se realizan modelos para explicar la fenomenología.
Lo que no se posee en la actualidad es un modelo que se pueda simular numéricamente y que a su vez posea un bajo costo computacional para realizarlo
El estudio de las variables físicas  

La mecánica de los fluidos se puede describir en tres niveles: macroscópica, mesoscópica y microscópica.
En el nivel macroscópico, predominan las leyes físicas de conservación de la masa, momento y energía aplicada a un volumen de control establecidas por un conjunto de ecuaciones diferenciales (ecuaciones de masa, momento y energía) que gobiernan el comportamiento del fluido. La Mecánica de Fluidos Computacional (\textit{Computational Fluid Dynamics} o CFD) es utilizada para resolver esas ecuaciones que gobiernan el comportamiento físico utilizando distintos métodos numéricos. En contraste el método de lattice Boltzmann(LBM) es una aproximación del nivel mesoscópica. LBM estudia la micro-dinámica de partículas ficticias utilizando modelos cinéticos simplificados. La cual provee un camino alternativo de simular la mecánica de fluidos. La naturaleza de la cinética brinda distintas características de LBM tales que es claro el panorama de los procesos de advección y colisión de partículas de fluidos simuladas; la estructura simple del algoritmo, la fácil implementación de condiciones de contorno y el natural paralelismo. Todos éstos interesantes atributos hacen que LBM sea una potente herramienta numérica para la simulación de sistemas de fluidos envueltos en problemas físicos complejos.

Los fluidos como el aire y el agua son frecuentemente conocidos en nuestra vida diaria. Físicamente todos los fluidos son compuestos de un gran conjunto de átomos o moléculas que chocan unas con otras moviéndose  aleatoriamente.
Interacciones de moléculas en un fluido son usualmente más débiles que las mismas en un sólido y un fluido puede ser deformado continuamente bajo una pequeña aplicación de esfuerzo
Usualmente la dinámica microscópica de las moléculas del fluido son muy complicadas y demuestran una fuerte inhomogeneidad y fluctuaciones.
Por el otro lado la dinámica macroscópica del fluido el cual es el resultado medio del movimiento de las moléculas en un medio homogéneo y continuo.
También puede ser explicado mediante modelos matemáticos de la dinámica de los fluidos una fuerte dependencia del largo y el tamaño de las escalas y cuál es el fluido observado.
Generalmente el movimiento de un fluido puede ser descripto por tres tipos de modelos matemáticos acuerdo a lo que se observan en las distintas escalas, por ejemplo microscópico en modelos de escala molecular, teorías cinéticas en la escala mesoscópica y modelos continuos para escalas macroscópicas.

Los modelos matemáticos de los flujos de fluidos, como también las ecuaciones de Newton para un basto número de moléculas, o las ecuaciones de Boltzman para la función de e distribución simple o las ecuaciones de Navier-Stokes para las variaciones de flujos macroscópicos, son extremadamente difíciles de resolver analíticamente de no ser imposibles.
La precisión de los modelos numéricos sin embargo han provisto de manera satisfactoria soluciones aproximadas de dichas ecuaciones.
Particularmente con la rapidez del desarrollo del software y hardware computacional y la tecnología, las simulaciones numéricas han comenzado a ser una importante metodología para la dinámica de fluidos.


El más exitoso y popular método de simulación de fluidos es la tećnica CFD , el cual su principal diseño está basado en  para resolver ecuaciones hidrodinámicas basadas en supociones de continuidad.
En CFD el dominio del flujo está compuesto en en conjunto de de sub-dominios con una malla computacional, , las ecuaciones matemáticas son discretizadas usando algunos esquemas de discretización numérica como elementos finitos, volúmenes finitos o diferencia finitas; los cuales resultan en un sistema algebraico de sistemas de ecuaciones para las variables discretas del fluido asociadas a la malla computacional. 
Computacionalmente son llevados a cabo para encontrar una solución aproximada resolviendo el problema algebraico de sistemas de ecuaciones usando un algoritmo secuencial o paralelo.
 
La Simulación de Dinámica Molecular (\textit{Molecular Dynamic Simulation} o MDS) es una técnica en la cual el movimiento individual de átomos o moléculas del fluido son registrados para resolver las ecuaciones de Newton mediante una computadora.
La principal ventaja de MDS es el aspecto macroscópico del fluido puede ser directamente conectado con el comportamiento molecular, en donde la estructura molecular y las interacciones microscópicas pueden ser descriptas de una manera directa.



%%% Local Variables: 
%%% mode: latex
%%% TeX-master: "template"
%%% End: 



\appendix
\chapter{Actividades relacionadas con la Práctica Profesional Supervisada y de Proyecto y Diseño}\label{ap1}
\graphicspath{{figs/}}

\section{Práctica profesional supervisada}

\section{Proyecto y diseño}

Las actividades de proyecto y diseño (PyD) realizadas para llevar a cabo el presente Proyecto Integrador de la carrera Ingeniería Nuclear fueron:
\begin{enumerate}
	\item Caracterización del diseño del SABR, desarrollada en el capítulo \ref{}.
	\item Validación del código OpenMC, detallada en el capítulo \ref{}.
	\item Cálculo del término fuente de neutrones debido a las reacciones de fusión, especificado en el capítulo \ref{}.
	\item Muestreo de la fuente externa de neutrones, indicado en el capítuo \ref{}.
	\item Cálculo neutrónico del reactor híbrido fusión-fisión, especificado en el capítulo \ref{}.
\end{enumerate}



\begin{biblio}
\bibliography{biblio}
\end{biblio}


\begin{postliminary}

\listoffigures                  %Figuras

\listoftables                   %Tablas

\begin{seccion}{Agradecimientos}
\begin{small}	

Inserte aquí su agradecimiento.

\end{small}


\end{seccion}


\end{postliminary}



\end{document}

