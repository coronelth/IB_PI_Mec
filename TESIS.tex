%%%%%%%%%%%%%%%%%%%%%%%%%%%%%%%%%%%%%%%%%%%%%%%%%%%%%%%%%%%%%%%%%%%%%%%%%%%%%%%%
% \documentclass[12pt,papel,twoside]{ibtesis}
%\documentclass[12pt,screen,oneside,pagebackref]{ibtesis}
\documentclass[12pt,papel,oneside]{ibtesis}
%\documentclass[12pt,papel,preprint,oneside]{ibtesis}


%%%%%%%%%%%%%%%%%%%%% Paquetes extra %%%%%%%%%%%%%%%%%%%%%%%%%%%%%%%%%%%%%%%%%%%
% Por conveniencia: aqu\'{\i} puede cargar todos los paquetes y definir los comandos 
% que necesite
\usepackage{ibextra}
\usepackage[T1]{fontenc}

\usepackage[all]{xy}
\usepackage{graphicx}
%\usepackage{subfigure} % subfigurasy
\usepackage{subcaption}
\usepackage{multicol}
\usepackage{wrapfig}
\usepackage{adjustbox}
\usepackage{latexsym}
\usepackage{tabulary,tabularx}
\usepackage{units}
\usepackage{fancyvrb}
\usepackage{multirow} 
\usepackage{rotating}\newpage
\usepackage{array}
\usepackage{gensymb}
\usepackage{float}
\usepackage{enumitem}
\usepackage{mathrsfs}
\usepackage{listings}
\usepackage[none]{hyphenat}

%\usepackage{hyperref}
%\hypersetup{
%    colorlinks,
%    citecolor=black,
%    filecolor=black,
%    linkcolor=black,
%    urlcolor=black
%}

%%%%%%%%%%%%%%%%%%%%%%%%%%%%%%%%%%%%%%%%%%%%%%%%%%%%%%%%%%%%%%%%%%%%%%%%%%%%%%%%
%%%%%%%%%%%%%%%%%%%%% Informacion sobre la tesis %%%%%%%%%%%%%%%%%%%%%%%%%%%%%%%
\title{Implementación de modelos lattice Boltzmann para flujo multifásico con transferencia de calor en unidades de procesamiento gráfico}
\author{Thomás Coronel}
\director{Mgter. Ezequiel Fogliatto}
\codirector{Ing. Pablo Argañaras}
\carrera{Proyecto Integrador de la Carrera de Ingeniería Mecánica}
\grado{}
\laboratorio{Departamento de Mecánica Computacional\\
%Departamento Física de Neutrones\\
Centro Atómico Bariloche}
\jurado{ Dr. René Cejas Bolecek  \\ 
Dr. Flavio Colavecchia \\ }
\palabrasclave{Lattice Boltzmann, flujo multifásico, transferencia de calor, GPU}%formato de Tesis, Lineamientos de escritura, Instituto Balseiro}
\keywords{Lattice Boltzmann, multiphase flow, heat transfer, GPU}%Thesis format, Templates, Instituto Balseiro}
% Si queremos poner la fecha manualmente:
\date{Junio de 2020}

%%%%%%%%%%%%%%%%%%%%%%%%%%%%%%%%%%%%%%%%%%%%%%%%%%%%%%%%%%%%%%%%%%%%%%%%%%%%%%%%
%\titlepagefalse % Si no quiere compilar la portada descomente esta linea
%\includeonly{resumen,cap1} % Compilar s\'{o}lo estos archivos 
\graphicspath{{figs/}} % Lugar donde encontrar las figuras generales (se puede poner uno en cada cap{\'{\i}}tulo)
%%%%%%%%%%%%%%%%%%%%%%%%%%%%%%%%%%%%%%%%%%%%%%%%%%%%%%%%%%%%%%%%%%%%%%%%%%%%%%%%


\begin{document}  

% Dentro del environment 'preliminary' va:
% la dedicatoria, resumen, abstract, indices

\begin{preliminary}

% Escriba su dedicatoria
\dedicatoria{
A mi heroína\\ ejemplo y \\modelo a seguir,\\
para vos mamá.
}

%%% \'{I}ndices %%%%

\begin{resumen}%

En este trabajo se implementó un código numérico para resolver problemas de mecánica de fluidos con transferencia de calor en flujos multifásicos con cambio de fase utilizando Unidades de Procesamiento Gráfico (GPU). Se utilizó la arquitectura de las GPU por su diseño, el cual permite realizar la ejecución de instrucciones con elevado nivel de paralelismo, y así reducir los costos computacionales de los algoritmos involucrados.

El modelo de lattice Boltzmann (LBM) pseudopotencial es el utilizado para abordar los problemas de interés, el cual resuelve de manera indirecta Ecuaciones Diferenciales Ordinarias (EDO) no lineales por medio de ecuaciones lineales más sencilla, y además tiene la ventaja de resultar altamente paralelizable. En particular, para el presente trabajo se utilizó un LBM con dos ecuaciones pseudopotencial con operador MRT.

El código realizado se implementó en los lenguajes de programación \textsc{C}, \textsc{Cuda C} y \textsc{Python}, en dónde el desarrollo del proyecto se llevó a cabo mediante la herramienta \textsc{CMake} con la posibilidad de seleccionar su tipo de variables (\textit{float} o \textit{double}). El proyecto se planificó para que los \textit{kernels} de \textsc{Cuda C} compilados puedan ser implementados en un \textit{script} de \textsc{Python} con su módulo \textsc{PyCuda}.

La validación del código se realizó en CPU y GPU para diferentes problemas numéricos. Se realizó una comparación entre los tiempos de cálculo del código en los diferentes lenguajes utilizados, en donde se obtuvo que la implementación en \textsc{Cuda C} llega a ser 23 veces más rápida que el equivalente en \textsc{C} para un único proceso, en las GPU y CPU evaluadas durante este Proyecto Integrador. Por otro lado, al realizar la comparación del código implementadoo en \textsc{PyCuda} con respecto al equivalente en \textsc{Cuda C}, se obtuvo que el primero es alrededor de un 10 \% más lento que el segundo. En cuanto a la comparación de la conveniencia de utilización de tipos de variable \textit{float} o \textit{double}, se obtuvo que la diferencia porcentual entre las  precisiones es cercana al 0,003 \%, y que el tiempo de cálculo en variables tipo \textit{double} es aproximadamente un 25 \% mayor que en \textit{float}.



\end{resumen}

\begin{abstract}%

In this work, a numerical code for multiphase flow with phase change and heat transfer was implemented in Graphics Processing Units (GPU). The GPU architecture is used by its design, which performs executions in parallel and thereby reduce the computational costs of the algorithms implemented. 

A pseudopotential lattice Boltzmann model (LBM) was used, which indirectly recovers the solution of nonlinear differential equations from the solution of a much simpler equation with a highly paralellizable algorithm. In particular, a pseudopotential LBM with MRT operator was implemented.

Code was implemented under \textsc{CMake} on the programming languajes \textsc{C}, \textsc{CUDA C} and \textsc{Python}, with the posibility to select variable types at compilation time. \textsc{CUDA C} kernels were also compiled to be used within \textsc{Python} scripts by means of the \textsc{PyCuda} module.

Code validation was performed on CPU and GPU for different numerical benchmarks. A comparison was made between the calculation times of the code on the different programming languages, where it was obtained that the implementation on \textsc{Cuda C} improvement from one to two orders of magnitude to the equivalent implementation on \textsc{C}. On the other hand, when comparing the code implemented on PyCuda with respect to the equivalent on Cuda C, the former was found to be about 10\% slower. Numerical solutions computed with different variable types (i.e. float or double) showed relative differences lower than 0.003\%,  while the computational time is increased by 25\% between each case.

\end{abstract}


%%% Local Variables: 
%%% mode: latex
%%% TeX-master: "template"
%%% End: 


\tableofcontents                %\'{I}ndice

\begin{abreviaturas}
	
	CFD : Computational Fluid Dynamics\\
	
	CPU : Central Processing Unit\\
	
	EOS : Equation of State\\
	
	GPU : Graphics Processing Unit\\
	
	LBE : lattice Boltzmann equation\\
	
	LBM : lattice Boltzmann Method\\
	
	MRT : multiple relaxation times\\
	
	SU  : Speed Up\\
	
	SD : Speed Down\\
	
	VdW : Van der Waals\\
	
	
\end{abreviaturas}

\end{preliminary}


% Podemos usar cualquiera de los dos comandos: \input o \include para incluir el texto
\chapter{Introducción}
\graphicspath{{figs/cap1/}}
\label{cap1}



En el estudio de la Mecánica de los Fluidos, son de importancia los problemas de transferencia de calor de flujos multi-fásicos con cambios de fase. Uno de los más relevantes por su importancia en la aplicación industrial y que contínuamente está en investigación debido a su complejidad; es la interacción entre el fluido de un reactor nuclear con sus barras de elemento combustible.

El punto crítico de calor (\textit{Critical Heat Flux} o CHF ) es aquél dónde la curva de Nukiyama de un material aumenta en varios órdenes de magnitud se temperatura si se incremente infinitesimalmente su potencia. Conocer el CHF de un reactor nuclear determina el punto de operación del mismo; debido a que el ente regulatorio nuclear que avala el funcionamiento de un reactor, determina la potencia de operación del mismo en base a la alcanzada en el CHF.

Nukiyama estudió, en la curva que lleva su nombre, cómo es la variación de la temperatura según la potencia que se le esté aplicandoa un material inmerso en un fluido y describió con detalle las diferentes etapas del proceso de ebullición. Actualmente la única forma de validar el CHF de las barras de elemento combustible de un reactor es mediante ensayos experimentales, los cuáles insumen mucho tiempo de espera (años) para realizar su prueba, porque son escasos los lugares que cuentan con la tecnología para brindar el servicio, además de ser costosos. 

Es de importancia contar con un código numérico que se encuentre validado, reproduzca la fenomenología de forma efectiva, sea fácil de utilizar cambiando la geometría y/o parámetros del fluido a utilizar. Los costos de fabricación disminuirían considerablemente al no ser necesaria una gran iteración de prototipos (en el problema mencionado).

Actualmente el dilema para reproducir numéricamente los tipos de problemas mencionados es que debido a la escala del fluido (\textit{mesoscópica}) con las técnicas convencionales de resolución de Mecánica de Fluidos Computacional(\textit{Computational Fluid Dynamicso} CFD) no se obtienen resultados en un tiempo razonable y que también tenga en cuenta todos las interacciones físicas que ocurren. En parte es debido a que los métodos tradicionales resuelven ecuaciones diferenciales, donde su tiempo de cálculo varía cuadráticamente con el número de elementos de malla utilizada para discretizar el problema físico.

Uno de los métodos numéricos que se están desarrollando para que el costo computacional sea cada vez menor es el método de lattice Boltzmann. Éstos métodos resuelven las Ecuaciones Diferenciales Ordinarias (EDO) lineales por medio de una ecuación lineal más sencilla, lo cuál reduce drásticamente el tiempo de cálculo. Las variables que se obtienen cumplen con la EDO de interés. Debido a que están basados en realizar operaciones elementales en los elementos de la malla discretizada y a su vez no dependen del total de nodos de la malla, lo hace altamente paralelizable. 

\section{Descripción de las escalas de los fluidos}

Los fluidos como el aire y el agua son conocidos en nuestra vida diaria. Físicamente los fluidos están compuestos de un gran conjunto de átomos o moléculas que chocan unas con otras moviéndose  aleatoriamente. Usualmente las interacciones de las moléculas de un fluido son más débiles que las de un sólido, por ello mediante la aplicación de un pequeño esfuerzo al fluido, éste puede ser deformado de manera continua.


Usualmente la dinámica microscópica de las moléculas del fluido son muy complicadas y demuestran una fuerte inhomogeneidad y fluctuaciones.
Por otro lado la dinámica macroscópica del fluido, el cual es el resultado medio del movimiento de las moléculas en un medio homogéneo y continuo.
Mediante modelos matemáticos puede explicarse la dinámica de los fluidos , según el fluido observado y su fuerte dependencia del tamaño de su escala.

La mecánica de los fluidos se puede describir en tres niveles: macroscópica, mesoscópica y microscópica.
Generalmente el movimiento de un fluido puede ser descripto por tres tipos de modelos matemáticos acuerdo a lo que se observan en las distintas escalas, por ejemplo microscópico en modelos de escala molecular, teorías cinéticas en la escala mesoscópica y modelos continuos para escalas macroscópicas.


Los modelos matemáticos de los flujos de fluidos son :

\begin{itemize}
	\item ecuación de Newton para una cantidad elevada de moléculas (escala microscópica)
	\item ecuación de Boltzman para la función de distribución simple (escala mesoscópica)
	\item ecuaciónes de aproximación de la mecánica del contínuo, como ser la ecuación de Navier-Stokes para macro-variaciones de flujos (escala macroscópica).
\end{itemize}

que resultan extremadamente difíciles de resolver analíticamente de no ser imposibles.

 La precisión de los modelos numéricos sin embargo han provisto de manera satisfactoria soluciones aproximadas de dichas ecuaciones.
Particularmente con la rapidez del desarrollo del software y hardware computacional y la tecnología, las simulaciones numéricas han comenzado a ser una importante metodología para la dinámica de fluidos.

Una de las vertientes de  investigación y aplicación más popular de simulación de fluidos es la técnica Mecánica de Fluidos Computacional (\textit{Computational Fluid Dynamics} o CFD) , el cual está diseñado para resolver ecuaciones hidrodinámicas basadas en supociones de continuidad. Siendo sus técnicas más difundidas la de elementos finitos, volúmenes finitos entre otras.
En CFD el dominio del flujo está compuesto según la técnica \textit{meshless} utilizada, como  diferencia finitas; los cuales resultan en un sistema algebraico de sistemas de ecuaciones para las variables discretas del fluido asociadas a la malla computacional. 
Computacionalmente son llevados a cabo para encontrar una solución aproximada resolviendo el problema algebraico de sistemas de ecuaciones usando un algoritmo secuencial o paralelo.
 
La Simulación de Dinámica Molecular (\textit{Molecular Dynamic Simulation} o MDS) es una técnica en la cual el movimiento individual de átomos o moléculas del fluido son registrados para resolver las ecuaciones de Newton.
La principal ventaja de MDS es el aspecto macroscópico del fluido puede ser directamente conectado con el comportamiento molecular, en donde la estructura molecular y las interacciones microscópicas pueden ser descriptas de una manera directa.

El método de lattice Boltzmann ( \textit{lattice Boltzman Method} o LBM) es una aproximación del nivel mesoscópica. LBM estudia la micro-dinámica de partículas ficticias utilizando modelos cinéticos simplificados, lo cual provee un camino alternativo de simular la mecánica de fluidos. La naturaleza de la cinética brinda distintas características de LBM tales que es claro el panorama de los procesos de advección y colisión de partículas de fluidos simuladas. La estructura del algoritmo es simple y de fácil implementación en las condiciones de contorno, además presenta un paralelismo natural. Todos éstos interesantes atributos hacen que LBM sea una potente herramienta numérica para la simulación de sistemas de fluidos envueltos en problemas físicos complejos\cite{guo2013lattice}. 


\section{Descripción general de LBM}

Para resolver un problema de mecánica de fluidos de forma numérica, se ha de discretizar el espacio físico mediante alguna técnica \textit{meshless}. Tradicionalmente los problemas se resueltos mediante la suposición del contínuo, y son llevados a cabo mediante las ecuaciónes de Navier-Stokes, que implican resolver un sistema algebraico de ecuaciones por ser EDO lineales. Según la cantidad de elementos que posea la malla, se regirá el tiempo de cálculo del problema; por ir aumentando de forma cuadrática el tiempo de cálculo, según la cantidad de elementos.

Uno de los métodos utilizados para resolver de manera indirecta las EDO lineales es el método de lattice Boltzmann. El cual consiste en resolver de forma sencilla una ecuación, cuyas variables obtenidas se pueden relacionar con la solución de la EDO lineal.

Los modelos de lattice Boltzmann provienen de la ecuación de Boltzmann, ésta se encuentra discretizada en espacios para resolver los problemas según: espacio físico, espacio de velocidades y espacio temporal. 

La discretización espacial surge de realizar un mallado a la región donde se plantea resolver un dado problema físico, y, cada nodo de la malla tiene asignado un espacio de velocidades. Todos los modelos se encuentran clasificados según el tipo \textit{DdQq} (\textit{d}-dimensiones  \textit{q}-velocidades), siendo \textit{q} la discretización realizada en el espacio de velocidades. La figura \ref{fig:D1Q3_D2Q9} muestra como es el conjunto de velocidades para el modelo D1Q3 y D2Q9.

El espacio de velocidades indica cómo es la propagación de las propiedades que poseen los nodos de la grilla. La velocidad de grilla del nodo i-ésimo se denota $\mathbf{e}_{i}$ y posee \textit{q} componentes. Para el modelo D2Q9 la figura \ref{fig:D1Q3_D2Q9} muestra un esquema de las velocidades de grilla del nodo i - ésimo y la Ec. (\ref{eq:velgrilla}) el valor adoptado.

\begin{figure}[h!]
	\centering
	\includegraphics[width=.8\textwidth]{figs/cap1/D1Q3_D2Q9}
	\caption{Conjunto de velocidades de los modelos D1Q3 y D2Q9. \cite{kruger2017lattice}}
	\label{fig:D1Q3_D2Q9}	
\end{figure}


\begin{equation}
{\mathbf{e}}_{i} =  
\left( \begin{array}{c} 
e_{i0} \\ e_{i1}\\ e_{i2}\\ e_{i3}\\ e_{i4}\\ e_{i5}\\
e_{i6}\\ e_{i7}\\ e_{i8}\\
\end{array}
\right) =
\left( \begin{array}{c} 
(0,0,0) \\ (1,0,0) \\ (0,1,0) \\(-1,0,0) \\ (0,-1,0) \\ (1,1,0) \\
(-1,1,0) \\ (-1,-1,0) \\ (1,-1,0)\\ 
\end{array}
\right) 
\label{eq:velgrilla}
\end{equation}




En el presente trabajo se resolverán los problemas descriptos en Cap. (\ref{cap4}), los cuáles son de 2 dimensiones y se adoptó el tipo de modelo D2Q9. 

Para los problemas de flujos multifásicos existen cuatro modelos generales de LB , siendo ellos:

\qquad \qquad color gradient, free energy, phase field y pseudopotential.

El modelo que se utilizará en el presente trabajo es el pseudopotencial, el cuál está basado en proponer un potencial de interacción entre las partículas del fluido. El potencial se utiliza para calcular la fuerza de interacción entre las partículas del fluido y está dado según una Ecuación de estado. 

El modelo posee un campo de distribución de probabilidades \textit{f} que tiene asociado \textit{q} componentes, siendo dependiente de la fuerza de interacción. Por medio de \textit{f} se pueden recuperar las variables macroscópicas asociadas del problema a resolver.

A continuación se datallará un ejemplo sencillo de un modelo \textit{D2Q9}. La Figura \ref{fig:grilla_D2Q9} muestra un esquema de las direcciones de las velocidades de grilla del nodo i-ésimo. En la misma figura se esquematiza como  es el proceso de colisión y advección (\textit{streaming}) que se realiza para actualizar los estados de los nodos cuando transcurre un paso de tiempo de la discretización temporal del problema. 


\begin{figure}[h!]
	\centering
	\includegraphics[width=8cm]{grilla_stre_colli_intro.png}
	\caption{Colisión y streaming de un modelo D2Q9 de LBM.}
	\label{fig:grilla_D2Q9}
\end{figure}


\begin{align}
	\mathbf{f}_{\alpha} (\mathbf{x} + \mathbf{e}_{\alpha} \mathbf{\delta}_{t}, t + \mathbf{\delta}_{t})  = \mathbf{f}_{\alpha} (\mathbf{x}, t) - \frac{1}{\tau} (\mathbf{f}_{\alpha} - {\mathbf{f}_{\alpha}}^{eq})
	\label{eq:field_intro} 
\end{align}

El término derecho de la Ec. (\ref{eq:field_intro}) contiene el proceso de \textit{colisión} y el izquierdo al \textit{streaming}; $\alpha$ corresponde al  $\alpha$-ésimo componente de los \textit{q} componentes del modelo, en éste ejemplo $\alpha = 1, 2, ...,9$. $f^{eq}$ es la función de distribución en estado de equilibrio. $\tau$ es un parámetro de relajación del modelo que depende fuertemente de las propiedades macroscópicas; como ser la densidad $\rho$, viscosidad $\nu$ entre otras.

Por medio de la \textit{colisión} y luego del \textit{streaming} se obtienen los parámetros macroscópicos el problema. Para éste caso sencillo se obtiene $\rho$ y $\mathbf{u}$ de las Ec. (\ref{eq: rho.1}) y (\ref{eq: u.1}) respectivamente.

\begin{align}
	\rho = \sum_{\alpha} \mathbf{f}_{\alpha}
	\label{eq: rho.1}
\end{align}

\begin{align}
	\rho \mathbf{u}= \sum_{\alpha} \mathbf{e}_{\alpha} \mathbf{f}_{\alpha}
	\label{eq: u.1}
\end{align}

El método numérico en éste caso tiene los siguiente forma de resolución:
\newline
{\scriptsize

\xymatrix{\>\ar@{->}[r]&\textbf{Colisión}\ar@^{->}[r]&\textbf{Streaming}\ar@{->}[r]&\textbf{C. de Contorno}\ar@{->}[r]&\textbf{Cálculo $\rho$, \textit{F} y \textit{U}\ar@{->}[r]}&\textbf{Actualización$\rho$, \textit{F} y \textit{U}}\ar@{->}[r]&\> \ar@/^{7mm}/[llllll]_{SIGUIENTE \> PASO \> DE \> TIEMPO}}
}
.
\newline 
\newline 

Por medio de las variables macroscópicas que se obtienen, se recupera la solución de la ecuación diferencial de interés

Debido a que cada nodo de la grilla debe realizar la misma operación de colisión de manera independiente del resto, el modelo es altamente paralelizable. Y además las operaciones matemáticas que deben ejecutarse en el operador de colisión son sencillas, no implicando un gran costo computacional.


\section{Descripción GPU}

Una Unidad de Procesamiento Gráfico (\textit{Graphics Processing Unit} o GPU) es un  circuito electrónico diseñado para realizar operaciones con punto flotante para renderizar píxeles en una pantalla. Están optimizadas para actuar en paralelo de forma simultánea instrucciones simples.

La GPU trabaja en conjunto con una Unidad Central de Procesamiento (\textit{Central Processing Unit} o CPU), debido a ello no presenta circuitos de control en su arquitectura. Dicho espacio que se encuentra ocupado en las CPU por el circuito de control las GPU lo disponen para incrementar el espacio en su chip de Unidades Aritmético Lógicas (\textit{Arithmetic Logic Unit} o ALU).

El esquema de la Figura \ref{fig:cpu_gpu_transis} se muestra cuanlitativamente la cantidade de transistores dedicados a diferentes tareas en la CPU comparado con la GPU.

\begin{figure}[h!]
	\centering
	\includegraphics[width=10cm]{cpu_gpu.png}
	\caption{Comparación cualitatica del uso de transistores entre CPU y GPU \cite{rinaldi2011modelos}.}
	\label{fig:cpu_gpu_transis}
\end{figure}

Las GPUs estan acondicionadas para llevar a cabo los cálculos que presentan los métodos numéricos de LB. Los LBM consisten en que a cada uno de los nodos de la grilla discretizada se le realicen operaciones elementales como los de la ecuación \ref{eq:field_intro}. Las GPU se encuentran optimizadas para ejecutar cálculos sobre los píxeles de un monitor, los cuales pueden considerarse como una grilla de nodos. A través de la paralelización la performance de las GPUs es alta, siendo capaces de procesar múltiples vértices y píxeles simultáneamente. Las simulaciones numéricas se producen en un menor tiempo en las GPU que sobre las CPU debido al alto paralelismo alcanzado por la GPU. \cite{rinaldi2011modelos}.


El lenguaje de programación que se utiliza en las placas gráficas es CUDA y fue desarrollado mediante la empresea NVIDIA. Se basa en el lenguaje de programación \textbf{C} con ciertas modificaciones para que los procesos sean en paralelo. Dichos procesos se especifican para que puedan ser lanzados en un número de \textit{blocks} y \textit{threads} de ejecución.






\section{Descripción y alcance del proyecto integrador}

Mediante el método LBM desarrollado por Fogliatto en \cite{fogliatto2018modelado}, \cite{fogliatto2019simulation} y \colorbox{green}{cita de ec energia} se realizó la implementación de un código numérico desarrollado en los lenguajes de programación \textsc{C} y \textsc{CUDA C}; para resolver problemas de transferencia de calor en flujos multifásicos con cambio de fase usando GPU de la forma más robusta posible.

La compilación del código se hizo mediante la herramienta multiplataforma \textsc{CMake}, por ser un proyecto complejo. Las bibliotecas que se generaron mediante \textsc{CMake} pueden ser utilizadas mediante el lenguaje \textsc{Python}. La utilización de \textsc{Python} es debida a la compatibilidad de aplicar  el código en las distintos sistemas operativos , como \textit{Linux} y \textit{Windows}. Además de que existe mucho soporte en éste lenguaje que cada vez se incrementa su número de usuarios por su facilidad de uso.

El proyecto se encuentra en \textbf{Git Hub} pudiendo ser descargado en \url{ https://github.com/efogliatto/LBCUDA_Test}.


La validación del código se hizo mediante los siguientes tres problemas:
\begin{itemize}
	\item \textit{construcción de Maxwell}
	\item \textit{estratificación de un fluido VdW con temperatura no uniforme}
	\item \textit{generación de burbujas sobre una superficie horizontal calefaccionada}
\end{itemize} 

las cuáles se encuentran explicadas en el Capítulo (\ref{cap4}).

Para los dos primeros problemas, se realizó una comparación en cuanto difieren los resultados de simple precisión y doble precisión del resultado analítico deseado. 

Se realizó un Speed Up, comparando el código en \textsc{C} y en \textsc{CUDA C}, ésto para ambas precisiones. Luego se realizó un Speed Up de las precisiones, tomando los códigos de \textsc{C} y \textsc{CUDA C} por separado.

Se implementó en \textsc{Python} parte del código realizado en \textsc{CUDA C}, exportando su biblioteca compilada mediante la biblioteca \textsc{PyCuda} de \textsc{Python}. También se realizó un Speed Up, ésta vez comparanco los resultados obtenidos en \textsc{CUDA C} con los de \textsc{Python}.


%%% Local Variables: 
%%% mode: latex
%%% TeX-master: "template"
%%% End: 

\chapter{Modelo de lattice Boltzmann}
\graphicspath{{figs/cap2/}}
\label{cap2}

\section{LBM Multifásicos}

En éste capítulo se presentara cuál es el modelo de LB para la resolución de problemas de transferencia de calor con flujos multifásicos y cambio de fase.

Generalmente es difícil resolver las ecuaciones de la mecánica de los fluidos. Las soluciones analíticas de los problemas que pueden ser halladas son escasas, como el caso de flujos \textit{Couette} o \textit{Poisueuille}. Problemas que contengan una geometría más compleja u otras condiciones de contorno; poseen una gran dificultad para encontrar la solución de las ecuaciones de la mecánica de fluidos, si es que el problema la posee. Debido a ello las soluciónes se obtienen numéricamente \cite{kruger2017lattice}(sec 3.1). Es de importancia el desarrollo de métodos numéricos que resuelvan los problemas de forma paralela para así reducir el tiempo de cálculo.

Los problemas que se plantean resolver en el presente trabajo son de transferencia de calor en flujos multifásicos con cambio de fase, la escala de fluido que se decide adoptar es la mesoscópica. Considerando la escala y siendo un problema multifásico, se opta por resolver numéricamente mediante LBM. 


Los mayoría de los modelos para resolver los flujos multifásicos son tres clasificados en cuatro categorias generales : \textit{color gradient}, \textit{Shan Chen model} o \textit{pseudopotential}, \textit{Free-energy} y \textbf{Falta phase-field}. 

\begin{itemize}
	
	\item \textit{Color gradient} fue el primer modelo de LBME para flujos multifásicos siendo desarrollado por Gunstensen \cite{gunstensen1991lattice}. Las fases y las interacciones entre las partículas son denotadas mediante diferentes colores. Por medio del modelado local del gradiente de color que se encuentra asociado a la diferencia de las densidades de las dos fases, se conoce como es la segregación y separación de las fases.
	
	\item \textit{Shan Chen model} o \textit{pseudopotential} surge de representar la fenomenología de \textit{color gradient} por medio de una redistribución de las particulas del fluido. La fuerza de interacción proviene de la diferencia entre las fuerzas promedio del modelo molecular  entre ambos lados de la interfaz. Shan and Chen \cite{shan1993lattice} presentaron un modelo de LBE (referenciado como modelo SC) que podría representar la interacción entre partículas fluidas de forma más precisa y directa introduciendo un pseudo potencial. 
	
	\item \textit{Free-energy} es un tipo de modelo alternativo de LBE desarrollado por Swift \cite{swift1995lattice} para modelos multifásicos/multicomponentes basado en la teoría de energía libre (\textit{free-energy}). La idea básica del nuevo método es realizar una función de distribución de equilibrio basada en funciones de energía libre, en las cuales se incorpora el tensor de presión termodinámico.\cite{guo2013lattice}(sec 7).
	
	\iffalse
	Debido a que la fenomenología representada en los modelos de LBE de color y pseudo potenciales son la misma. \textbf{se corta la oracion}
	\fi
	
\end{itemize}




\section{Modelo pseudopotencial}

El modelo pseudopotencial es el adoptado en el presente trabajo para abordar los problemas descriptos en el Cap. \ref{cap4} por lo cual se detalla en ésta sección. Este modelo multifásico tiene la particularidad de que la frontera entre las fases no es resuelta con exactitud. Dicha interfaz es representada de forma difusa con un cierto tamaño en la grilla, siendo una importante ventaja para el cálculo puesto que la interfaz no debe ser reconstruida.\cite{parrill2019reviews}.


La obtención de la ecuación de lattice Boltzmann (LBE) a través de la ecuación de Boltzmann se encuentra descripta en \cite{kruger2017lattice} (Sec 3.4 y 3.5 ). Primero se explica la discretización en el espacio de velocidades, en el segundo la del espacio físico, temporal e integración de las características.

El problema físico a resolver cuenta con una región, la cuál se le realizará un mallado para discretizar el espacio. El nodo i - ésimo de la malla posee las coordenadas ${\bar{X}}_{i} = (x,y,z)$, a su vez densidad $\rho_{i}$ y temperatura $T_{i}$. La velocidad en el nodo tiene las componentes ${\bar{U}}_{i} = ({U}_{ix},{U}_{iy},{U}_{iz})$. El espacio de velocidades indica como es la propagación de las propiedades en la grilla. Dicha velocidad de grilla $\mathbf{e}_{i}$ posee $\alpha$ componentes donde $\alpha = q $. 

Para el modelo D2Q9 la figura \ref{fig:D1Q3_D2Q9} muestra un esquema de las velocidades de grilla del nodo i - ésimo y la Ec. (\ref{eq:velgrilla}) el valor adoptado.

En el presente trabajo el LBM es $D2Q9$. 


\begin{equation}
    {\mathbf{e}}_{i} =  
    \left( \begin{array}{c} 
                e_{i0} \\ e_{i1}\\ e_{i2}\\ e_{i3}\\ e_{i4}\\ e_{i5}\\
                e_{i6}\\ e_{i7}\\ e_{i8}\\
            \end{array}
    \right) =
    \left( \begin{array}{c} 
        (0,0,0) \\ (1,0,0) \\ (0,1,0) \\(-1,0,0) \\ (0,-1,0) \\ (1,1,0) \\
        (-1,1,0) \\ (-1,-1,0) \\ (1,-1,0)\\ 
    \end{array}
    \right) 
    \label{eq:velgrilla}
\end{equation}

Cada nodo de la grilla contará con una distribución $f_{i}$, asociada al espacio de poblaciones. En el caso más simple se cuenta con una distribución y esa asociada a la densidad. La distribución $f_{i}$ posee $\alpha$ componentes. Mediante ésta funcion de distribución, pueden recuperarse variables macroscópicas del problema, como ser : $\rho$ , $U$,$T$ .

En éste modelo la distribución $f$ se calcula a través de las fuerzas de interacción que poseen las partículas de cada nodo. Por lo que se debe proporcionar un potencial que describa a las fuerzas de interacción, dicho potencial estará dado por una Ecuación de estado (\textit{Equation of state} o EOS).

Es de importancia la elección de la EOS a utilizar, puesto que según ella se recupera el tensor de presión, las variables $\rho$ , $U$ , $T$ ; a una dada presión y temperatura las densidades de coexistencia de líquido y gas, como el calor latenten entre otras propiedades intrínsecas del fluido.

\section{Ecuación de estado y fluido de Van der Waals}

Al ser multifásicos los problemas a desarrollar, es de importancia conocer las leyes que gobiernan las fases de los fluidos. La ley de gases ideales \ref{eq:gas_ideal} es una EOS , cabe destacar que la mayor supusición es que las partículas del gas son puntuales. Donde \textit{p} es la presión (atm), V es el volúmen (L), \textit{n} número de moles, R constante universal de los gases y \textit{T} temperatura. La Ley caracteriza el comportamiento para gases de baja densidad.

\begin{align}
p V = n R T
\label{eq:gas_ideal}
\end{align}

La EOS de Van der Waals (VdW) \ref{eq:VdW_P} fue propuesta para caracterizar el comportamiento de los gases reales, siendo $V_m = \frac{V}{n}$ el volúmen molar. Las constantes \textit{a} y \textit{b} son características de cada fluido.

\begin{align}
p = \frac{R T}{V_m - b} - a {(\frac{1}{V_m})}^2
\label{eq:VdW_P}
\end{align}

El parámetro \textit{a}  ($\frac{atm L^2}{mol^2}$) caracterizan la interacción que poseen las moléculas del gas entre sí, y \textit{b} ($\frac{L}{mol}$) da una idea del volumen molar mínimo que posee una partícula del fluído (éste parámetro define a la partícula del gas con un dado volúmen en vez de ser puntual como en la Ley de gases ideales).

La EOS de VdW es utilizada en el presente trabajo para modelar el potencial del modelo LBM \textit{pseudopotential} de interacción entre las partículas.


\section{Modelo pseudopotencial de dos ecuaciones con operador MRT}

En ésta sección se describirá cuál es el LBM que se utilizará para resolver los problemas descriptos en el Cap. \ref{cap4}. La denominación del tipo de operador, \textit{multiple relaxation times} o MRT, proviene por los parámetros que se deben ajustar y proponer para que el modelo concuerde con los valores esperados.

Se deben resolver dos ecuaciones, la primera es la hidrodinámica que representa a la conservación de masa y momento; la segunda a la de energía, descriptas en \colorbox{green}{cita de ec energia}.


\subsection{Ecuación hidrodinámica}

Por medio de la función de distribución Ec. \ref{eq:fieldmom} \cite{li2013lattice} se realiza la resolución de la ecuación hidrodinámica,

\begin{equation}
    \mathbf{f}(\mathbf{x} + \mathbf{e} \> \delta_{t} , t + \delta_{t}) = \mathbf{M}^{-1} \left[ \mathbf{m} - \mathbf{\Lambda}(\mathbf{m} - \mathbf{m}^{(eq)}) + \delta_{t} \left( \mathbf{I} - 0,5 \mathbf{\Lambda} \right) \mathbf{\bar{S}}  \right]_{(\mathbf{x},t)} 
    \label{eq:fieldmom}
\end{equation}

donde $\mathbf{x}$ es la posición espacial, \textit{t} el tiempo, $\delta_{t}$ el paso de tiempo , \textit{\textbf{e}} la velocidad de grilla en sus direcciones $\alpha$ y $\textit{f}$ es la distribución de densidad en el espacio de poblaciones. La notación utilizada implica que la componente $\alpha$-ésima del miembro izquiero de la Ec. (\ref{eq:fieldmom}) esté dado por $f_{\alpha}(x + e_{\alpha} \delta_{t}  , t + \delta_{t} )$. El miembro derecho de la Ec. (\ref{eq:fieldmom}) corresponde a la etapa de post-colisión definida en el espacio de momentos, donde \textbf{I} es el tensor identidad, \textbf{M} una matriz de transformación ortogonal, $\mathbf{m} = \mathbf{M} \cdot \mathbf{f}$ , $\mathbf{m}_{eq} = \mathbf{M} \cdot \mathbf{f}_{eq}$ , $\mathbf{\bar{S}} = \mathbf{M} \mathbf{S}$ el término de fuente y $ \Lambda$ es una matriz diagonal.

Para el modelo de grilla D2Q9 utilizado en el presente trabajo $\mathbf{m}_{eq}$ y $ \Lambda$ están dados por Ec. \ref{eq:m} y \ref{eq:lambda} respectivamente.

\begin{align}
m_{eq} =  \rho  \left( 1, - 2 + 3 {|u|}^{2} , 1 - 3{|u|}^{2} , u_{x} , - u_{x} , u_{y} , - u_{y} , {u_{x}}^{2} - {u_{y}}^{2} , u_{x} u_{y} \right) 
\label{eq:m}
\end{align}

    
\begin{align}
    \mathbf{\Lambda}  = diag ( {\tau_{\rho }}^{-1},{\tau_{e}}^{-1},{\tau_{\zeta }}^{-1},{\tau_{j}}^{-1},{\tau_{q}}^{-1},{\tau_{j}}^{-1},{\tau_{q}}^{-1},{\tau_{\nu }}^{-1},{\tau_{\nu}}^{-1}) 
    \label{eq:lambda}
\end{align}


La densidad macroscópica es obtenida mediante la Ec. (\ref{eq:rho}).

\begin{equation}
        \rho = \sum_{\alpha} f_{\alpha}
        \label{eq:rho}
\end{equation}

Por medio de la Ec. (\ref{eq:U}) se obtiene la velocidad.

\begin{equation}
    \rho \> \mathbf{u} = \sum_{\alpha} {\mathbf{e}}_{\alpha} \> f_{\alpha} + 0,5 \> {\delta}{t} \> \mathbf{F}
    \label{eq:U}
\end{equation}

Para el modelo D2Q9 la fuerza total $\mathbf{F}$ posee sólo dos componentes $F_{x} , F_{y}$. Donde $ {\mathbf{F}} = {\mathbf{F}}_{b} + {\mathbf{F}}_{int} $, siendo ${\mathbf{F}}_{b}$ la fuerza volumétrica y ${\mathbf{F}}_{int}$ la fuerza de interacción que hay en el sistema dadas por Ec.(\ref{eq:Fb}) y (\ref{eq:fint}) respectivamente.

\begin{equation}
	{\mathbf{F}}_{b} = \rho \> \mathbf{g}
	\label{eq:Fb}
\end{equation}


\begin{equation}
{\mathbf{F}}_{int} = - G \> \psi(\mathbf{x}) \sum_{\alpha=1}^{N} w({|{\mathbf{e}}_{\alpha}|}^{2}) \> \psi (\mathbf{x} + {\mathbf{e}}_{\alpha} \> \delta_{t}) \> {\mathbf{e}}_{\alpha} 
\label{eq:fint}
\end{equation}

Donde G corresponde a la intensidad de interacción, $w({|{\mathbf{e}}_{\alpha}|}^{2})$ son los pesos correspondientes a una grilla D2Q9 y $\psi$ es el potencial :

\begin{equation} 
    \psi(\rho) = \sqrt{\frac{2 (p_{EOS} - \rho {c_{s}}^{2})}{G {c}^{2}}}
    \label{eq:psi}
\end{equation}

Donde $c_{s}$ es la velocidad del sonido en el medio, la EOS adoptada en el presente trabajo es la de VdW :

\begin{equation}
    p_{EOS} = \frac{\rho R T}{1- \rho B} - A {\rho}^{2}
\end{equation}

donde \textit{A} y \textit{B} son parámetros que determinan los valores críticos de temperatura, presión y densidad. Finalmente, la fuerza de interacción se incorpora en la etapa de colisión mediante un término de fuente apropiado:

\begin{equation}
    \bar{S} = 
    \left[ \begin{array}{c} 
        0\\
        6 \mathbf{u}\cdot \mathbf{F} + \frac{12 \sigma {|{\mathbf{F}_{int}|}}^{2} }{{\psi}^{2} \delta_{t} (\tau_{e} - 0,5)}\\
        6 \mathbf{u}\cdot \mathbf{F} - \frac{12 \sigma {|{\mathbf{F}_{int}|}}^{2} }{{\psi}^{2} \delta_{t} (\tau_{\zeta } - 0,5)}\\
        F_{x}\\
        -F_{x}\\
        F_{y}\\
        -F_{y}\\
        2(u_{x} F_{x} - u_{y} F_{y} )\\
        (u_{x} F_{x} + u_{y} F_{y} )\\              
    \end{array}
    \right]    
    \label{eq:termino_fuente_s}
\end{equation}
\\
donde $\sigma$ es un parámetro libre del modelo MRT, al igual que $\Lambda$ , el cual se utiliza para ajustar las diferencias de fases obtenidas de la EOS y de la simulación realizada.

Se puede recuperar la ecuación de Navier - Stokes utilizando Ec. [\ref{eq:rho} - \ref{eq:termino_fuente_s} ] mediante el análisas de Chapman-Enskog limitado para un número de Mach bajo \cite{li2013lattice}. Siendo las ecuaciones recuperadas \cite{fogliatto2019simulation} \cite{li2013lattice}:

\begin{equation}
	\frac{\partial \rho }{\partial t}  + \nabla \cdot \left( \rho \mathbf{u} \right) = 0
\end{equation}
\begin{equation}
	\frac{\partial \left( \rho \mathbf{u}\right)}{\partial t} + \nabla \cdot \left( \rho \mathbf{u} \mathbf{u}\right) = - \nabla \left( \rho {c_{s}}^{2} \mathbf{I} \right) + \nabla \mathbf{\Pi} + \mathbf{F} - 2 G^{2} c^{4} \sigma \nabla \left( {|\nabla \psi	|}^{2} I	\right) + O (\partial^{5})
\end{equation}
\begin{equation}
\mathbf{F} = - G c^{2} \left[	\psi \nabla \psi + \frac{1}{6} c^{2} \psi \nabla \left( \nabla^{2} \psi\right) + ...\right] + \mathbf{F}_{b} = \mathbf{F}_{int} + \mathbf{F}_{b}
\end{equation}
\\
Donde $\mathbf{\Pi}$ es el tensor de viscosidad dado por :

\begin{equation}
	\mathbf{\Pi} = \rho \nu \left[	\nabla \mathbf{u} + {\left(\nabla \mathbf{u}\right)}^{T}\right] + \rho \left(\ \xi - \nu \right) \left( \nabla \mathbf{u}\right) \mathbf{I}
\end{equation}
\\
siendo $\quad\nu = {c_{s}}^{2} (\tau_{\nu}- 0,5) \delta_{t}\quad$ la viscosidad cinemática y $\quad\xi = {c_{s}}^{2} (\tau_{\nu}- 0,5) \delta_{t}\quad$ la viscosidad volumétrica.



\subsection{Ecuación energía}

Para tener en cuenta la transferencia de calor en el modelo, es necesario adicionar otra ecuación de LB acoplandola con la primera\cite{li2013lattice}. Para nuestro caso, se utiliza la distribución de poblagicones \textit{g} en la Ec. (\ref{eq:fieldenergy}), la cuál también posee un operador de colisión MRT. Donde $\mathbf{n} = \mathbf{M} \mathbf{g}$ es una distribución de momentos y $\hat{\Gamma}$ es una fuente en el espacio de momentos.


\begin{equation}
    \mathbf{g}(\mathbf{x} + \mathbf{e} \delta_{t} ,t + \delta_{t}) = \mathbf{M}^{-1} \left[ \mathbf{n} - \mathbf{Q}(\mathbf{n} - \mathbf{n}^{(eq)}) + \delta_{t} \left( I - 0,5 Q \right) \hat{\Gamma}  \right]_{(\mathbf{x},t)}
    \label{eq:fieldenergy}
\end{equation}

En éste caso los parámetros libres del modelo MRT vienen dados en parte por la matriz de coeficientes de relajación \textbf{Q}, siendo compuesta por la diagonal que se indica en Ec.(\ref{eq:Q_matriz}) y además por los elementos no nulos $Q_{3,4}$ y $ Q_{5,6}$ que se indican en Ec. (\ref{eq:Q_34}) y (\ref{eq_Q_56})  respectivamente.

\begin{equation}
    \textit{diag} (Q) = {( q_{0} , q_{1} , q_{2} , q_{3} , q_{4} , q_{5} , q_{6} , q_{7} , q_{8} )}^{T}
    \label{eq:Q_matriz}
\end{equation}
\begin{equation}
    Q_{3,4} = q_{4} \left( \frac{q_{3}}{2} - 1 \right)
    \label{eq:Q_34}
\end{equation}
\begin{equation}
    Q_{5,6} = q_{6} \left( \frac{q_{5}}{2} - 1 \right)
    \label{eq_Q_56}
\end{equation}

La distribución de equilibrio $\mathbf{n}_{eq}$ se encuentra definida en Ec.(\ref{eq:n_eq}), siendo $\alpha_{1}$ y $\alpha_{2}$ parámetros libres del modelo MRT.

\begin{equation}
    {\mathbf{n}}_{eq} = T { \left( 1, \alpha_{1}, \alpha_{2}, u_{x}, -u_{x}, u_{y}, -u_{y}, 0, 0 \right) }^{T}
    \label{eq:n_eq}
\end{equation}

La temperatura macroscópica \textit{T} puede recuperarse mediante:

\begin{equation}
T = \sum_{\alpha} g_{\alpha} + \frac{1}{2} \delta_{t} {\hat{\Gamma}}_{0}
\end{equation}



Siendo el término fuente :

\begin{equation}
    \hat{\Gamma} = {( s, 0, 0, 0, 0, 0, 0, 0, 0 )}^{T}
\end{equation}

con 

\begin{equation}
    s = \frac{\chi}{\rho} \nabla T \cdot \nabla \rho + T \left[ 1 - \frac{1}{\rho c_{\nu}} {\left( \frac{\partial p_{EOS}}{\partial T} \right)}_{\rho} \right] \nabla \cdot \mathbf{u}
    \label{eq:s_chica}
\end{equation}

La recuperación de la ecuación del calor por medio de Ec. [\ref{eq:fieldenergy} - \ref{eq:s_chica} ] resulta \cite{markus2011simulation}:

\begin{equation}
    \frac{\partial T}{\partial t} + \nabla \cdot ( \mathbf{u} T ) = \chi \> {\nabla }^{2} T + s
\end{equation}

Donde $\chi$ es la difusividad térmica, considerándose constante por simplicidad, obteniéndose mediante Ec. (\ref{eq:chi}). La difusividad térmica queda determinada por los coeficientes de relajación $\mathbf{Q}$ y los parámetros libres $\alpha_{1}$ y $\alpha_{2}$ de ${\textbf{n}}_{eq}$.

\begin{equation}
    \chi = \delta_{t} \left( \frac{1}{q_{3}} - \frac{1}{2} \right) \left( \frac{ 4 + 3 \alpha_{1} + 2 \alpha_{2}}{6} \right)
    \label{eq:chi}
\end{equation}

\subsection{Condiciones de contorno}

Cuando se implementa la colisión en los nodos que se encuentran en la frontera de la región a resolver, no se poseen todos los valores de la función de distribución de poblaciones $\mathbf{f}$ y $\mathbf{g}$. Por lo que es necesario determinar cómo se obtienen los mismos. 

Los problemas que se desarrollarán en el Cap. \ref{cap4} tienen una región rectangular o cuadrada, con la la particularidad que sus  condiciones de contorno en los bordes es periódica, ya sea en el eje \textit{X}, \textit{Y} o en ambos. Por lo que no existe la dificultad de fijar los valores de $\mathbf{f}$ y $\mathbf{g}$ en nodos que se encuentran en las vértices de la región. Siendo los casos a desarrollar  el de los nodos que se encuentran en las aristas del cuadrado o rectángulo. Las cuatro (4) aristas de la región se resuelven de forma idéntica por lo que sólo se  desarrollará una.

La Figura (\ref{fig:CC_hidro}) ilustra las direcciones de las componentes de $\mathbf{f}$ y $\mathbf{g}$ para un nodo que se encuentra en la frontera. La disposición observada es la que se evaluará para las condiciones de contorno hidrodinámica y de energía.  

\begin{figure}[h!]
	\centering
	\includegraphics[width=0.85\textwidth]{figs/cap2/CC_hidrodinamica.png}
	\caption{Bloques de threads organizados en una grilla de bloques \cite{rinaldi2011modelos}.}
	\label{fig:CC_hidro}
\end{figure}

Por convención las direcciones que se conocen los valores de las funciones de distribución pertenecen al conjunto A y las que no se conocen al conjunto B. Por lo que $ 0, 1, 2, 4, 5, 8 \in A$ y $ 3, 6, 7 \in B$ 


\subsubsection{Condición hidrodinámica}

El desarrollo de Zou \cite{zou1997pressure} es el utilizado para la ecuación hidrodinámica. Primeramente el método \textit{bounceback} es aplicado a la dirección \textit{3}, resultando $f_{3} = f_{1}$ como se observa en la Figura \ref{fig:CC_hidro}. Luego se analizan las Ec. (\ref{eq:rho}) y (\ref{eq:U}) que son puestas de la siguiente forma:

\begin{equation}
	\begin{array}{c}
	f_{3} + f_{6} + f_{7} = \rho - \left( f_{0} + f_{1} + f_{2} + f_{4} + f_{5} + f_{8}	 \right)\\
	f_{6} - f_{7} = \rho u_{x} - \left( f_{1} - f_{3} - f_{7} + f_{8} 	 \right)\\
	f_{3} + f_{6} + f_{7} = \rho u_{y} + \left( f_{1} + f_{5} + f_{8} \right)
	\end{array}
\end{equation}

Para finalizar se aplican las condiciones de no deslizamiento, para éste caso $\> u_{y} = 0\quad$ y $\quad\mu \frac{\partial u}{\partial x} = \tau_{wall}\quad$, siendo $\tau_{wall}$ el esfuerzo de corte realizado en la arista analizada. Se obtenienen las siguientes condiciones de contorno a la distribución de poblaciones $\mathbf{f}$:

\begin{equation}
\begin{array}{c}
f_{i3} = f_{i1}\\
f_{i7} = f_{i5} - 0,5 (f_{i4} - f_{i2}) - 0,25 (F_{ix} + F_{iy})\\
f_{i6} = f_{i8} + 0,5 (f_{i4} - f_{i2}) - 0,25 (F_{ix} + F_{iy})\\
\end{array}
\end{equation}

donde el índice \textit{i} indica que es el nodo \textit{i}-ésimo que de la arista y $F_{ix}$, $F_{iy}$ es la fuerza del nodo. 


\subsubsection{Condición de energía}

La condición de contorno a realizar en la ecuación de energía, es mantener una temperatura fija en una de las aristas o en secciones de la misma. Por lo cual se utiliza el desarrollo de Inamuro \cite{inamuro2002lattice}.

Primeramente se calcula el valor de la distribución de equilibrio del nodo $i$-ésimo $\mathbf{g}_{eq\>i}$ utilizando Ec. (\ref{eq:n_eq}) y $\mathbf{M}$ de la forma $\mathbf{g_{eq}} = \mathbf{M}^{-1} \mathbf{n_{eq}}$. Se prosigue a calcular el parámetro $\beta$:

\begin{equation}
\beta = \frac{T_{cc} - \sum_{A} g_{A}}{\sum_{B} g_{B}}
\label{eq:beta}
\end{equation}

donde $T_{cc}$ es la temperatura fijada como condición de contorno, $g_{A}$ adopta los valores conocidos de las direcciones pertenecientes al cpnjunto A ($0, 1, 2, 4, 5, 8 $), mientras que $g_{B}$ adopta los valores de la dirección de $\mathbf{g_{eq}}$ calculada recientemente para las direcciones de B ($3, 6, 7$).

Por último en las direcciones del conjunto B, el valor de $g$ resulta:

\begin{align}
	g_{B} = \beta \> g_{eq \> B} 
\end{align}

A modo de ejemplo se detalla el valor que adoptará la dirección \textbf{3}:

\begin{equation}
	g_{3} = \frac{T_{cc} - \overbrace{\left( g_{0} + g_{1} +g_{2} + g_{4} + g_{5} + g_{8} \right)}^{\sum_{A} g_{A}} }{\underbrace{\left( g_{eq\>3} + g_{eq\>6} + g_{eq\>7} \right)}_{\sum_{B} g_{B} }} \quad g_{eq\>3}
\end{equation}

%%% Local Variables: 
%%% mode: latex
%%% TeX-master: "template"
%%% End: 

%\chapter{Implementación del código numérico en GPU}
\chapter{Código numérico de LBM en GPU}
\graphicspath{{figs/cap3/}}
\label{cap3}



%\section{Implementación del código}

En el presente capítulo se realizará la descripción de la implementación del código numérico del LBM descripto en la Sec. (\ref{sec:LBM_2_ec_MRT}); como también las implicancias de elaborar la implementación en una GPU de forma eficiente.

El lenguaje de programación \textbf{C} desarrollado por Dennis MacAlistair Ritchie será utlizado primeramente para confeccionar el código. Dicho lenguaje brinda instrucciones a la CPU de una PC para ser ejecutadas. Las CPU son diseñadas óptimamente para que sus instrucciones sean procesadas de forma secuencial en los núcleos que poseen; aunque también se permite realizar los procesos en paralelo, según la cantidad de núcleos.

Luego se implementará un código en \textbf{CUDA C} el cuál fue desarrollado por la empresa NVIDIA. Este lenguaje permite ejecutar instrucciones en una GPU, la cuál está diseñada para que los procesos a realizar sean de forma paralela.

Ambos códigos , \textbf{C} y \textbf{CUDA C}, son compilados en bibliotecas estáticas mediante \textbf{CMake}. Con las bibliotecas compiladas se prosigue a utilizarlas mediante el lenguaje de programación interpretado \textbf{Python}. La bibliotecas necesarias para que el código de \textbf{C} y \textbf{CUDA C} puedan ser llevadas a cabo en \textbf{Python} son \textit{Ctypes} y \textit{PyCuda} respectivamente.

Se eligió la programación en \textbf{C} y \textbf{CUDA C} para comparar la eficiencia en el tiempo de cálculo, debido a que el lenguaje \textbf{CUDA C} es una extensión del lenguaje \textbf{C}. La diferencia principal es que \textbf{CUDA C}  tiene una forma particular de escribir las funciones que se ejecutarán en la GPU, las cuáles son llamadas \textit{kernel}.

La implementación en \textbf{Python} es debido a su facilidad de programación, el cuál permite incorporar las bibliotecas compiladas y obtener una mayor versatilidad de problemas a resolver. \textbf{Python} puede ser utilizado en distintos sistemas oparativos como \textit{Linux}, \textit{Windows} y \textit{Mac OS}. 

En \textbf{Python} sólo se realizará la implementación con la biblioteca \textit{Pycuda}, debido a que se espera una mayor eficiencia en tiempos de cálculo que en \textit{Ctypes}.




\section{Programación en GPU}




\textbf{Cuidado con las citas. Es muy de la tesis de Rinaldi}


Una computadora (\textit{Personal Computer} o PC) posee como procesador principal la CPU, cuyo diseño se encuentra optimizado para realizar tareas secuenciales. Comercialmente vienen de una amplia variedad de núcleos, en el rango de 8 a 64 núcleos en el caso de los Procesadores AMD Ryzen™ Threadripper, 4 a 8 núcleos en los Procesador AMD FX™, en el caso de Intel se encuentra el Procesador Intel® Core™ serie X con 18 núcleos y  Intel® Core™ I3-9100T de 4 núcleos entre otros. \cite{edp:2020:amd} \cite{icp:2020:intel}

Es posible realizar tareas y procesos en paralelo en la CPU proveyendo instrucciones a cada núcleo del procesador de forma independiente, siendo más eficiente en el tiempo de ejecución de los procesos realizados.

A su vez una PC opcionalmente puede contener un coprocesador siendo una GPU. Dicha placa se encuentra diseñada para realizar operaciones en paralelo, realizándolas en varios hilos de ejecución (\textit{threads}).




El procesador principal que tiene una PC es la CPU, por lo cual se denomina \textit{host}, la GPU es un coprocesador y se denomina \textit{device}. Las ejecuciones de los procesos en la CPU están diseñadas para que se efectúen de manera secuecncial, en cuánto las de la GPU en paralelo; las últimas realizándose en varios hilos de ejecución (\textit{threads}). Las operaciones en paralelo que son analizadas por la CPU y las deriva a la GPU son llevadas a cabo mediante funciones llamadas \textit{kernel}. \textit{Host} y \textit{device} poseen su propia memoria RAM, llamadas \textit{host memory} y \textit{device memory} respectivamente. \cite{rinaldi2011modelos}

Los \textit{threads} que posee una GPU se pueden agrupar de dos formas. Una de ellas es mediante bloques (\textit{thread block}) y el otro es de grilla de bloques (\textit{grid}). Todos los \textit{threads} del mismo \textit{thread block} poseen acceso a una memoria compartida que es de acceso rápido y permite sincronizar las ejecuciones que se les asigna. Debido a la posibilidad de sincronización de los mismos se evita el riesgo de que varios \textit{threads} acceden de manera simultánea al mismo lugar de memoria. Un \textit{grid} es un conjunto de \textit{thread block}, en dónde las instrucciones del \textit{kernel} son paralelizadas, ésto vence la limitación de hardware del finito número de \textit{threads} por bloque. Debido a que la ejecución del proceso en los bloques de \textit{threads} de una grilla pueden ser ejecutados en tiempos distintos, es necesario realizar un sincronización entre los mismos para que su comunicación sea segura y no haya conflictos\cite{tolke2010implementation}. La figura \ref{fig:block_grid_threads} muestra el concepto de \textit{grid} y \textit{thread block}.

\newpage
\begin{figure}[h!]
	\centering
	\includegraphics[width=0.45\textwidth]{figs/cap3/threads_block_grid.jpg}
	\caption{Bloques de threads organizados en una grilla de bloques \cite{rinaldi2011modelos}.}
	\label{fig:block_grid_threads}
\end{figure}

\section{Programación en CUDA C}

La programación de las GPU se lleva a cabo mediante el lenguaje \textbf{CUDA}, el cuál es una extención del lenguaje \textbf{C} debido su familiaridad y uso extendido, por lo que el lenguaje se denomina \textbf{CUDA C}. La realización de procesos en paralelos es ejecutada mediante funciones llamadas \textit{kernel} y son del tipo \textit{void}.

Existen tres tipos de funciones que se pueden llevar a cabo y son:

\begin{itemize}
	
	\item \textbf{host} función clásica de C que se ejecuta en la CPU, siendo invocable únicamente por funciones que se ejecuten en la CPU. 

	\item \textbf{global} es una función \textit{kernel} invocada desde la CPU para ejecutarse en la GPU. 
%	Debe especificar la cantidad de bloques y de \textit{threads} por bloque a lanzar la función.
	
	\item \textbf{device} es una función que se ejecuta en la GPU y únicamente puede ser llamada desde un \textit{kernel}.
	
\end{itemize}

En el presente trabajo sólo se utilizaran funciones de tipo \textbf{host} y \textbf{global}, pudiéndose realizar en un futuro el  \textit{profiling} mediante el uso de las funciones \textbf{device}.

Otra particularidad que se presenta en la programación es el manejo de la memoria; por parte del \textit{host} como del \textit{device}, siendo abordado posteriormente.
\newpage

\subsection{Programación de un \textit{kernel}}

Para visualizar las diferencias de programación de una función \textbf{CUDA C } (\textit{kernel}) , con una típica función de \textbf{C}, se muestra a continiación como ejemplo la programación para ambos lenguajes de la Ec. (\ref{eq:rho}): 

\begin{align*}
	\rho = \sum_{\alpha} f_{\alpha}
\end{align*}

donde se muestra primeramente su programación en \textbf{C}

{\footnotesize
	\begin{frame}{}
		\lstset{language=C,
			framesep=2mm,
			basicstyle=\ttfamily,
			keywordstyle=\color{blue}\ttfamily,
			stringstyle=\color{red}\ttfamily,
			commentstyle=\color{green}\ttfamily,
			morecomment=[l][\color{magenta}]{\#}
		}
		\begin{lstlisting}[frame=single]
#include <momentoDensity.h>
#include <stdio.h>

void momentoDensity(scalar* rho, scalar* field, basicMesh* mesh) {
	
	// Suma de todas las componentes
	
	for( uint i = 0 ; i < mesh->nPoints ; i++ ) {
		
		rho[i] = 0;	    
		for( uint j = 0 ; j < mesh->Q ; j++ ) {
			rho[i] += field[ i*mesh->Q + j ];
		}		
	}
}
		\end{lstlisting}
		
	\end{frame}
}

Donde \textbf{basicMesh* mesh} es un puntero a una estructura, la cuál posee información del mallado que se realizó al dominio del problema a resolver ; como por ejemplo la cantidad de nodos que posee la malla (\textbf{nPoints}) y la cantidad de direcciones que posee el modelo en su espacio de velocidades \textbf{Q}. El vector  \textbf{scalar* rho} es de dimensión \textit{nPoints} y contiene los valores de $\rho$, por último \textbf{scalar* field} es un vector que posee las \textbf{Q} componentes de la función de distribución de poblaciones \textit{f} para cada uno de los  \textit{nPoints} nodos de la malla.

Cabe destacarse que por los resultados obtenidos mientras se realizó la programación del código en \textbf{C}, se vio que el uso de la función \textbf{for} para éstos casos es más eficiente que el de \textbf{while}.

La programación de la Ec.(\ref{eq:rho}) implementada en \textbf{CUDA C} es la siguiente :
\newpage

{\footnotesize
	\begin{frame}{}
		\lstset{language=C,
			framesep=2mm,
			basicstyle=\ttfamily,
			keywordstyle=\color{blue}\ttfamily,
			stringstyle=\color{red}\ttfamily,
			commentstyle=\color{green}\ttfamily,
			morecomment=[l][\color{magenta}]{\#}
		}
		\begin{lstlisting}[frame=single]
#include <cudaMomentoDensity.h>
#include <cuda_runtime.h>
#include <stdio.h>
#include <stdlib.h>

extern "C" __global__ void cudaMomentoDensity(cuscalar* field,
				              cuscalar* rho,
					      int np,
					      int Q ) {
							
	int idx = threadIdx.x + blockIdx.x*blockDim.x;	
	if( idx < np ) {	
		int j= 0;		
		cuscalar sum = 0;		
		while ( j < Q ) {		
			sum += field[ idx*Q + j ];			
			j++;			
		}				
		rho[idx] = sum;	
	}
}		
		\end{lstlisting}
		
	\end{frame}
}

En éste caso se pasa de forma distinta los valores de la cantidad de nodos (\textbf{np}) y de la cantidad de velocidades del modelo \textit{DdQq} (\textbf{Q}). La distinción en los argumentos que se pasan es debido a como es que \textbf{CUDA} permite el manejo de las estructuras y de las decisiones que se tomaron cuando se desarrollaba el código para uno u otro lenguaje.

Es de importancia conocer explícitamente que es lo que realizan las siguientes líneas:

{\footnotesize
	\begin{frame}{}
		\lstset{language=C,
			framesep=2mm,
			basicstyle=\ttfamily,
			keywordstyle=\color{blue}\ttfamily,
			stringstyle=\color{red}\ttfamily,
			commentstyle=\color{green}\ttfamily,
			morecomment=[l][\color{magenta}]{\#}
		}
		\begin{lstlisting}
	int idx = threadIdx.x + blockIdx.x*blockDim.x;	
	if( idx < np ) {	
		\end{lstlisting}
		
	\end{frame}
}

las cuáles indican que se realizará de forma paralela las ejecuciones que se encuentran entre \textcolor{blue}{if} ( idx < np)\{...\}, donde es necesario identificar los \textit{threads} a dónde serán llevados a cabo las tareas. El identificador es \textit{idx} donde blockIdx.x  indica la cantidad de \textit{threads} por bloques y blockDim.x el numero de \textit{block's}.

A modo de ejemplo si tenemos que $\quad np = 2048 \quad$ y que la cantidad de \textit{thread} por \textit{block} es 512 \cite{zone2020cuda}, con indicar que se utilizarán cuatro (4) \textit{block's} todo el cálculo se realiza simultáneamente, si la cantidad es menor se necesitará más tiempo para la tarea.

Se colola \textcolor{blue}{extern} \textcolor{red}{''C''} para que el compilador sepa que es una función de \textbf{CUDA} y además para lo implemente en una biblioteca tipo \textit{ptx} para \textbf{Python}.

Resta ver cómo es el llamado de las funciones realizadas en el \textit{main}, por lo que en \textbf{C} se tiene:


{\footnotesize
	\begin{frame}{}
		\lstset{language=C,
			framesep=2mm,
			basicstyle=\ttfamily,
			keywordstyle=\color{blue}\ttfamily,
			stringstyle=\color{red}\ttfamily,
			commentstyle=\color{green}\ttfamily,
			morecomment=[l][\color{magenta}]{\#}
		}
		\begin{lstlisting}
		momentoDensity( rho, field_f, &mesh);
		\end{lstlisting}
		
	\end{frame}
}

la cuál no requiere de ninguna explicación. Mientras que en \textbf{CUDA C} se tiene:


{\footnotesize
	\begin{frame}{}
		\lstset{language=C,
			framesep=2mm,
			basicstyle=\ttfamily,
			keywordstyle=\color{blue}\ttfamily,
			stringstyle=\color{red}\ttfamily,
			commentstyle=\color{green}\ttfamily,
			morecomment=[l][\color{magenta}]{\#}
		}
		\begin{lstlisting}
cudaMomentoDensity<<<ceil(mesh.nPoints/xgrid)+1,xgrid>>>(
 		deviceField, deviceRho, cmesh.nPoints, cmesh.Q);  
cudaDeviceSynchronize();

		\end{lstlisting}
		
	\end{frame}
}

en donde < < <, > > > indica la cantidad de \textit{block} en que se realizará la tarea, como así también la cantidad de \textit{threads} en cada \textit{block}.

\begin{align*}
		<<<\quad \overbrace{ceil(mesh.nPoints/xgrid)+1}^{cantidad \>de\> \textit{threads}\> por\> \textit{block}}\quad,\quad \underbrace{xgrid}_{cantidad\>de\>block} \quad>>>
\end{align*}



\subsection{Utilización de la memoria de \textit{host} y \textit{device}}


Las funciones que son realizadas en \textbf{C} pueden retornar alguna variable mientras que en \textbf{CUDA C} los \textit{kernel} son del tipo \textit{void}, lo cuál implica que es necesario almacenar memoria en el \textit{device} para una variable, y pasarla como argumento al \textit{kernel} para obtener lo pretendido. En la subsección siguiente se explicará con detalle cómo realizar la allocación.


La allocación de memoria en el \textit{host} es la misma que se utiliza en C: \textbf{malloc($\>$)}, mientras que en el \textit{device} se realiza mediante: \textbf{cudaMalloc($\>$)}, la cual recibe los siguientes argumentos:
{\footnotesize
\begin{frame}{}
	\lstset{language=C,
		framesep=2mm,
		basicstyle=\ttfamily,
		keywordstyle=\color{blue}\ttfamily,
		stringstyle=\color{red}\ttfamily,
		commentstyle=\color{green}\ttfamily,
		morecomment=[l][\color{magenta}]{\#}
	}
	\begin{lstlisting}
cudaMalloc(void **devPtr, size_t size);
	\end{lstlisting}

\end{frame}
}
(devPtr) puntero para allocar la memoria del \textit{device} , (size\_t) memoria en bytes a reservar. Se debe tener en cuenta que la memoria se reserva de manera lineal.

La transferencia de datos entre los dos tipos de memoria se efectua mediante la función \textbf{cudaMemcpy($\>$)}, la cual recibe los siguientes argumentos:
{\footnotesize
\begin{frame}{}
	\lstset{language=C,
		framesep=2mm,
%		baselinestretch=1.2,
		basicstyle=\ttfamily,
		keywordstyle=\color{blue}\ttfamily,
		stringstyle=\color{red}\ttfamily,
		commentstyle=\color{green}\ttfamily,
		morecomment=[l][\color{magenta}]{\#}
	}
	\begin{lstlisting}
cudaMemcpy(void *dst, void *src, size_t count, cudaMemcpyKind kind);
	\end{lstlisting}
	
\end{frame}
}
(dst) puntero con la dirección de destino de los datos, (src) puntero con la dirección de origen, (count) es la cantidad de bytes a transferir y (kind) es el tipo de transferencia a realizar\cite{zone2020cuda}. En la tabla \ref{tab:cudamemcy} se encuentran los cuatro tipos posibles de transferencia.

\begin{table}[h!]
	\centering
	\begin{tabular}{|c|c|}
		\hline
		\multicolumn{1}{|l|}{TIPO DE TRANSFEREMCIA} & \multicolumn{1}{l|}{SENTIDO DE TRANSFERENCIA} \\ \hline
		\textbf{cudaMemcpyHostToHost}               & host host                                     \\ \hline
		\textbf{cudaMemcpyHostToDevice}             & host device                                   \\ \hline
		\textbf{cudaMemcpyDeviceToHost}             & device host                                   \\ \hline
		\textbf{cudaMemcpyDeviceToDevice}           & device device                                 \\ \hline
	\end{tabular}
	\caption{Tipos de transferencias de datos en CUDA \cite{represa2016introduccion}.}
	\label{tab:cudamemcy}
\end{table}

El lanzamiento de un \textit{kernel} en varios bloques tiene una particularidad en la ejecución de los procesos, los \textit{threads} de cada bloque llevan realizan los procesos en diferentes tiempos por ser independientes. \textbf{Que quiere decir esta oracion. Parece no estarbien escrita} Para evitar problemas en las operaciones que se realizan en la memoria pasada al \textit{kernel} ( surge en parte debido a que un \textit{kernel} es una función \textbf{void} ), se utiliza la función \textbf{cudaDeviceSynchronize($\>$)}. Esta función hace que los bloques realicen su ejecución y antes de proceder a la siguiente instrucción espera a que todos los bloques de \textit{threads} hallan terminado. Para sincronizar los \textit{threads} de un mismo bloque en un función \textit{device} se utiliza \textbf{syncthreads($\>$)}.


\subsection{Ejemplo de programacion de un kernel}
















\section{Arquitectura de memoria}

\begin{figure}[h!]
	\centering
	\includegraphics[width=\textwidth]{figs/cap3/Schematization-of-CUDA-architecture-Schematic-representation-of-CUDA-threads-and-memory.png}
	\caption{Esquematización de la arquitectura de CUDA. Izquierda: lanzamiento de un \textit{kernel} desde el \textit{host}. Derecha: jerarquía de memoria.  \cite{nobile2014cutauleaping}.}
	\label{fig:schedule_architecture_cuda}
\end{figure}

Según la arquitectura que posea un procesador, son su respectiva jerarquía en memoria, accesos y latencias se piensa cómo llevar a cabo la implementación de un código. Por ello es de importancia conocer éstas características. 

Se mencionó anteriormente que el \textit{host} y \textit{device} poseen su propia memoria. La transferencia de datos de una memoria a otra tiene una muy alta latencia en cualquiera de los dos sentidos. 

%Por lo que al realizar la implementación de un código hay que minimizar la transferencia para que el tiempo de ejecución de los procesos sea mínimo.

Cuando el \textit{host} efectúa su rutina de ejecución y se encuentra con un lanzamiento de \textit{kernel}, éste será llevado a cabo en múltiples \textit{threads} del \textit{device}. Una representación de ello se muestra en la figura \ref{fig:schedule_architecture_cuda}. 

Los procesos en el \textit{device} tienen almacenados los datos según una jerarquía de memoria, con su respectiva limitación de acceso que se observa en la figura \ref{fig:schedule_architecture_cuda}. Los \textit{threads} pueden acceder a datos de muchas memorias diferentes en distintos procesos, dichas memorias son las siguientes:

\begin{itemize}
	\item  \textit{register memory} es visible para un único \textit{thread}
	\item \textit{local memory} tiene las mismas caracteísticas que \textit{register memory} pero con una performance menor.
	\item \textit{shared memory} es visible por todos los \textit{threads} de un mismo bloque, posee una baja latencia en su acceso.
	\item \textit{global memory} es visible por todos los \textit{threads} de la grilla y también por el \textit{host}, posee una alta latencia de acceso.
	
	Las siguientes memorias poseen una alta latencia de acceso son asignadas y son aignadas para usos específicos y generalmente se almacenan en \textit{caché}:
	
	\item \textit{constant memory} es visible por todos los \textit{threads} de la grilla siendo únicamente de lectura. El uso de ésta memoria puede reducir el ancho de banda de memoria requerido en comparación con la {global memory}
	\item \textit{texture memory} es otra memoria en la que sólo se puede leer y tienen acceso todos los \textit{threads} de la grilla. Al realizar lecturas de \textit{threads} ó \textit{threadblocks} adyacentes su performance en comparación con \textit{global memory} es mayor. 
	
\end{itemize}




\textbf{yo agregaria aca que este codigo compilado puede usarse en Python, y decir algo general de como hay que compilarso e importarlo con PyCUDA}

%%% Local Variables: 
%%% mode: latex
%%% TeX-master: "template"
%%% End: 

\chapter{Descripción de los problemas en fluidos con transferencia de calor }
\graphicspath{{figs/cap4/}}
\label{cap4}

En el presente capíulo se realiza la descripción de los problemas con transferencia de calor para fluidos multifásicos con cambio de fase llevados a cabo para validar los códigos numéricos desarrollados, siendo éstos la \textit{Construcción de Maxwell}, la \textit{Estratificación de un fluido Van Der Waals con temperatura no uniforme} y la \textit{Generación de bubbles en una superficie horizontal calefaccionada}.

\section{Construcción de Maxwell}

Como se desarrollo en el Cap. \ref{cap2} la Ec. (\ref{eq:VdW_P}) es una EOS que modela el comportamiento de un gas real.

\begin{align*}
	p = \frac{R T}{V_m - B} - A {\left(\frac{1}{V_m}\right)}^2
\end{align*}

La Ec. (\ref{eq:VdW_P}) puede ser representada gráficamente en un diagrama $P - V_m$. \\ La Figura \ref{fig:P_V_CO2} muestra el diagrama mencionado para el dióxido de carbono ($CO_2$) a distintas temperaturas: 373 (K), 304 (K) y 270 (K). Para $T = 270 \> K$ se vislumbra que en $p = 44,08 \> atm$ la gráfica se intersecta en tres valores de $V_m$, siendo dos de ellos estables; por lo que se observa que hay dos volúmenes molares de coexistencia, indicando las fases líquido y gaseosa.

La construcción de Maxwell, también llamada regla de igualdad de áreas, indicada en Ec. (\ref{eq:maxwell_Construction}), es un procedimiento analítico para encontrar las densidades de coexistencia del líquido y gas. Donde $P$ es la presión de la EOS y $p_0$ es una presión constante. Al realizar la integral propuesta surge que las áreas \textbf{A1} y \textbf{A2} de la Figura \ref{fig:P_V_CO2} deben ser iguales.

\begin{equation}
\int_{V_{m,l}}^{V_{m,g}} P d V_m = p_0 (V_{m,l} -  V_{m,g})
\label{eq:maxwell_Construction}
\end{equation}

\begin{figure}[htbp]
	\centering
	\includegraphics[width=.8\textwidth]{figs/cap4/Diagrama_P_V_del_CO2_Multiphase_LBM}
	\caption{Diagrama $P - V_m$ de la EOS de VdW del $CO_2$ con las constantes $a = 3,592$ y $b = 0,04267$, representando par a $T = 270 \> K$ los volúmenes molares del líquido y gas. \cite{huang2015multiphase}}
	\label{fig:P_V_CO2}	
\end{figure}


La Ec. (\ref{eq:VdW_P}) se puede re-estructurar como Ec. (\ref{eq:rho_eos}) puesto que $\rho = \frac{1}{v}$, siendo $v$ el volúmen másico y relacionando $V_m$ con $v$ según cada fluido. Donde para un dado valor de temperatura tendremos la coexistencia de fases con su densidad $\rho_l$ para la fase líquida y $\rho_g$ para la gaseosa.

\begin{equation*}
p_{EOS} = \frac{\rho R T}{1- \rho b} - a {\rho}^{2} \nonumber 
%\label{eq:VdW_rho}
\end{equation*}

Para un dado valor de temperatura, llamado temperatura crítica (\textit{$T_c$}) comienzan a coexistir las dos fases. En el ejemplo mostrado de la Figura \ref{fig:P_V_CO2} $T_c = 304 \> K$. Analíticamente $T_c$ surge de aplicar el criterio de la primera y segunda derivada a la Ec.(\ref{eq:rho}) como se indica en Ec.(\ref{eq:criterio_1_2_deriv}) y se deben conocer los parámetros \textit{a} y \textit{b}.

\begin{equation}
	\frac{\partial\> p}{\partial\> V_{m}} = 0 \qquad \qquad \frac{\partial^{2} \> p}{\partial\> {V_{m}}^{2}} = 0
	\label{eq:criterio_1_2_deriv}
\end{equation}

Realizando adecuadamente la adimensionalización  de la Ec.(\ref{eq:rho_eos}) se puede graficar una curva de coexistencia $T_r - \rho_r$  como se observa en la Figura \ref{fig:T_r_rho_r_analitico}, siendo $T_r = \frac{T}{T_c}$ y $\rho_r = \frac{\rho}{\rho_c}$.

\begin{figure}[htbp]
	\centering
	\includegraphics[width=.8\textwidth]{figs/cap4/Diagrama_T_r_vs_rho_r_analitico}
	\caption{Curva de coexistencia de fases para un fluido de VdW con los parámetros $a = 0,5 $ y $b = 4,0 $.}
	\label{fig:T_r_rho_r_analitico}	
\end{figure}

\newpage
Es de importancia destacar que las curvas de coexistencia dependen de la EOS que se esté utilizando para describir el comportamiento del fluido, en éste caso de estudio, la EOS es de VdW y posee  cómo parámetros \textit{a} y \textit{b} para describir los distintos fluidos .

\subsection{Validación}

La validación de éste problema se hizo utilizando los parámetros $a =0,5$ y $b = 4,0$; para un tamaño de malla de 201 x 201 nodos y $T_r$ variando con un paso de $0,025$ en el rango de $[0,6 - 0,975]$.  Los valores que se utilizaron de $\Lambda$ y $\mathbf{M}$ para el modelo de LBM realizado  con el resto de los parámetros se encuentran en el Apéndice (\ref{parametros_MxC}).
%La descripción de cómo se realizó la curva de coexistencia analítica se encuentra en el Apéndice (xxxx).

La Figura(\ref{fig:v_760_MxC_c_simple}) muestra la validación del código realizado en \textsc{C} para simple precisión en una GPU NVIDIA Geforce GTX 760; con distintos parámetros $\sigma$ del modelo MRT, donde se observa que el valor de $\sigma = 0,125$ es el que mejor ajusta a la curva de coexistencia. La Figura (\ref{fig:v_760_MxC_cuda_simple}) muestra el resultados obtenidos del código realizado en \textsc{Cuda C} en simple precisión en la misma GPU.

\begin{figure}[htbp]
	\centering
	\includegraphics[width=\textwidth]{figs/cap4/v_760_MxC_c_simple}
	\caption{Curva de coexistencia de fases para un fluido de VdW con los parámetros $a = 0,5 $ y $b = 4,0 $, obtenida en simple precisión en la GPU NVIDIA Geforce GTX 760 en el código desarrollado en \textsc{C}. $\sigma = 0.075[\bigtriangleup]$	 $\sigma = 0.125[\bigcirc]$ y $\sigma = 0.200[\diamondsuit]$ }
 	\label{fig:v_760_MxC_c_simple}	
\end{figure}

\begin{figure}[htbp]
	\centering
	\includegraphics[width=\textwidth]{figs/cap4/v_760_MxC_cuda_simple}
	\caption{Curva de coexistencia de fases para un fluido de VdW con los parámetros $a = 0,5 $ y $b = 4,0 $, obtenida en simple precisión en la GPU NVIDIA Geforce GTX 760 en el código desarrollado en \textsc{Cuda C}. $\sigma = 0.075[\bigtriangleup]$	 $\sigma = 0.125[\bigcirc]$ y $\sigma = 0.200[\diamondsuit]$ }
	\label{fig:v_760_MxC_cuda_simple}	
\end{figure}

Ambos resultados mostrados en las Figuras (\ref{fig:v_760_MxC_c_simple}) y (\ref{fig:v_760_MxC_cuda_simple}) fueron realizados para 50000 pasos de tiempo en cada uno de los valores de $T_r$. El valor obtenido de las densidades de coexistencia de fases entre los códigos de \textsc{C} y \textsc{Cuda C} resultan exactamente iguales. 

El resultado que se obtuvo de realizar la validación en doble precisión es que los dos códigos desarrollados obtienen el mismo valor en las densidades de fase.

\newpage
\subsection{Comparación de precisiones}

En el análisis de la comparación de las precisiones, sólo se presentan los resultados obtenidos mediante el código de \textsc{C}; debido a que los resultados de \textsc{Cuda C} son idénticos.

La Figura (\ref{fig:v_760_MxC_c_comparacion}) muestra los resultados de las densidades de coexistencia de fases para simple y doble precisión, no siendo apreciable la diferencia.

\begin{figure}[htbp]
	\centering
	\includegraphics[width=\textwidth]{figs/cap4/v_760_MxC_c_comparacion}
	\caption{Curva de coexistencia de fases para un fluido de VdW con los parámetros $a = 0,5 $ y $b = 4,0 $, obtenida en simple precisión[$\bigcirc$] y doble precisión [$\diamondsuit$] en la GPU NVIDIA Geforce GTX 760 en el código desarrollado en \textsc{C}.} 
	\label{fig:v_760_MxC_c_comparacion}	
\end{figure}


La Tabla (\ref{tab:comp_MxC_precisiones_10}) contiene los valores de las densidades de coexistencia: analítico, y los obtenidos mediante el código en simple precisión y doble precisión. De la Figura (\ref{fig:v_760_MxC_c_comparacion}) se observa que se ajusta mejor al resultado analítico la fase líquida que la fase gaseosa, por ello la Tabla (\ref{tab:comp_MxC_precisiones_10}) presenta esta diferenciación.

Para dar una idea de proximidad a la solución analítica se adoptó como parámetro la distancia entre los vectores de densidad de fase obtenido y el analítico. La distancia se calculó por medio de la Norma Euclídea, siendo calculada la distancia entre dos vectores \textbf{\textit{A}} y \textbf{\textit{B}} con \textit{i} elementos como indica la Ec.(\ref{eq:norma_euclidea}):

\begin{align}
dist(\mathbf{A},\mathbf{B}) = \sqrt{\sum_i {\left( a_i - b_i \right)}^2  }
\label{eq:norma_euclidea}
\end{align}

Los resultados que muestra la Tabla (\ref{tab:comp_MxC_precisiones_10}) en cuanto a la distancia de los vectores se observa que en doble precisión los resultados se aproximan mejor en la densidad de coexistencia de la fases, en la fase gaseosa; mientras que para la densidad de coexistencia de fase, de la fase líquida, se aproxima mejor mediante simple precisión. 

Porcentualmente para la fase gaseosa, la distancia calculada en simple precisión es 0,0034 \% mayor que en doble precisión; para la fase líquida la distancia en doble precisión es 0,0491 \% mayor que en simple precisión.


% Please add the following required packages to your document preamble:
% \usepackage{multirow}
\begin{table}[h!]
\centering
%\resizebox{17cm}{!}{
	\begin{tabular}{|c|c|c|c|c|c|c|}
	\hline
	& \multicolumn{3}{c|}{${\rho_{r}}_{\>gaseoso}$}      & \multicolumn{3}{c|}{${\rho_{r}}_{\>líquido}$} \\ \hline
	$\mathbf{T_r}$    & \textbf{Analítico}      & \textbf{Simple}       & \textbf{Doble}     & \textbf{Analítico}      & \textbf{Simple}     & \textbf{Doble}   \\ \hline
	0.600 & 0.0599097 & 0.0653772 & 0.0653724 & 1.32424  & 1.31956 & 1.31975 \\ \hline
	0.625 & 0.0733723 & 0.0848112 & 0.0847656 & 1.46149  & 1.46239 & 1.4624  \\ \hline
	0.650 & 0.0897449 & 0.0951336 & 0.095124  & 1.56762  & 1.56858 & 1.56859 \\ \hline
	0.675 & 0.107606  & 0.113153  & 0.113147  & 1.56762  & 1.65819 & 1.65821 \\ \hline
	0.700 & 0.128332  & 0.13353   & 0.133511  & 1.6572   & 1.73703 & 1.73704 \\ \hline
	0.725 & 0.150966  & 0.15182   & 0.151811  & 1.73595  & 1.80812 & 1.80813 \\ \hline
	0.750 & 0.177353  & 0.177323  & 0.177319  & 1.80706  & 1.87326 & 1.87328 \\ \hline
	0.775 & 0.206739  & 0.206086  & 0.206075  & 1.87233  & 1.93364 & 1.93364 \\ \hline
	0.800 & 0.23938   & 0.244229  & 0.244223  & 1.93243  & 1.98754 & 1.98759 \\ \hline
	0.825 & 0.277393  & 0.281299  & 0.281284  & 1.98899  & 2.04083 & 2.04085 \\ \hline
	0.850 & 0.319677  & 0.323471  & 0.323456  & 2.0423   & 2.09124 & 2.09124 \\ \hline
	0.875 & 0.368925  & 0.371915  & 0.371891  & 2.09234  & 2.14114 & 2.14117 \\ \hline
	0.900 & 0.425549  & 0.428416  & 0.428335  & 2.14012  & 2.18665 & 2.18665 \\ \hline
	0.925 & 0.493618  & 0.495973  & 0.495947  & 2.18563  & 2.23017 & 2.23019 \\ \hline
	0.950 & 0.493618  & 0.580402  & 0.580368  & 2.22933  & 2.27407 & 2.27411 \\ \hline
	0.975 & 0.578746  & 0.690126  & 0.690247  & 2.27117  & 2.312   & 2.312   \\ \hline
	\textbf{Distancia} & -         & 2.05233   & 2.05226   & -        & 0.14234 & 0.14241 \\ \hline
\end{tabular}%}
    \caption{Comparación de las precisiones con respecto a cuánto se acercan al valor analitico, tanto para doble, como simple precision, la norma utilizada para medir la distancia de los vectores es la norma  euclídea. Para el problema de la Construcción de Maxwell con la GPU NVIDIA Geforce GTX 760.}
    \label{tab:comp_MxC_precisiones_10}
    \end{table}



\newpage

\subsection{Speed Up}

En la presente sección se muestran las mejora en el tiempo de cálculo realizados para el código de \textsc{C} y \textsc{Cuda C}. La comparación se realizó en simple y doble precisión; en dos GPU, siendo las mismas NVIDIA Geforce GTX 760 y NVIDIA Geforce GTX 970. Se tomó una $T_r$ fija y se varió el tamaño de la grilla, de manera que ésta siempre fuese cuadrada, respetando un número de nodos de potencia de 2 en los lados del cuadrado. La cantidad de \textit{thread blocks} que se utilizó para realizar la comnparación en el código de \textsc{CUDA} fueron de potencia de 2.

\subsubsection{NVIDIA Geforce GTX 760}

Los tamaños de grilla que se utilizaron para realizar las pruebas de tiempo de ésta placa, tienen el rango de grilla de 16x16 nodos hasta 2048x2048 nodos. La cantidad de \textit{thread blocks} que se utilizó fueron de 1 a 512.

Las Figuras (\ref{fig:s_760_MxC_simple_1.0}) y (\ref{fig:s_760_MxC_double_1.0}) muestran el \textit{Speed Up} obtenido comparando los códigos de \textsc{C} y \textsc{Cuda C}, donde la Figura (\ref{fig:s_760_MxC_simple_1.0}) está obtenida con simple precisión y la Figura (\ref{fig:s_760_MxC_double_1.0}) en doble precisión. El mejor resultado en ambos casos se obtuvo para un número de \textit{thread block} igual a 64, donde la mejora fue de 18.67 y 11.40 en simple y doble precisión respectivamente, para el mayor número de elementos de malla.


\begin{figure}[htbp]
	\centering
	\includegraphics[width=\textwidth]{figs/cap4/s_760_MxC_simple_10}
	\caption{Speed Up realizado para el problema de la Construcción de Maxwell con la GPU NVIDIA Geforce GTX 760 en simple precisión, comparando los códigos de \textsc{C} y \textsc{Cuda C}.} 
	\label{fig:s_760_MxC_simple_1.0}	
\end{figure}

\begin{figure}[htbp]
	\centering
	\includegraphics[width=\textwidth]{figs/cap4/s_760_MxC_double_10}
	\caption{Speed Up realizado para el problema de la Construcción de Maxwell con la GPU NVIDIA Geforce GTX 760 en doble precisión, comparando los códigos de \textsc{C} y \textsc{Cuda C}.} 
	\label{fig:s_760_MxC_double_1.0}	
\end{figure}

\newpage

Las Figuras (\ref{fig:c_760_MxC_c_10}) y (\ref{fig:c_760_MxC_cuda_10}) muestran el \textit{Speed Up} obtenido comparando simple precisión y doble precisión, donde la Figura (\ref{fig:c_760_MxC_c_10}) está obtenida el código de \textsc{C} y la Figura (\ref{fig:c_760_MxC_cuda_10}) en el código de \textsc{Cuda C}. 

En el código de \textsc{C} para el mayor número de elementos de la malla, el resultado de tiempos de cálculo en doble precisión es apenas 1,026 veces mayor que en  simple precisión. En contraste, al fijarse el resultado para un número de \textit{thread block} igual a 64 (el que mayor ganancia obtuvo), el tiempo de cálculo en doble precisión es 1,68 veces mayor que en simple precisión; para el mayor número de elementos de malla calculado.

\begin{figure}[htbp]
	\centering
	\includegraphics[width=\textwidth]{figs/cap4/c_760_MxC_c_10}
	\caption{Speed Up realizado para el problema de la Construcción de Maxwell con la GPU NVIDIA Geforce GTX 760 en en el código de \textsc{C}, comparando simple precisión y doble precisión.} 
	\label{fig:c_760_MxC_c_10}	
\end{figure}

\begin{figure}[htbp]
	\centering
	\includegraphics[width=\textwidth]{figs/cap4/c_760_MxC_cuda_10}
	\caption{Speed Up realizado para el problema de la Construcción de Maxwell con la GPU NVIDIA Geforce GTX 760 en en el código de \textsc{Cuda C}, comparando simple precisión y doble precisión.} 
	\label{fig:c_760_MxC_cuda_10}	
\end{figure}

Los valores que se obtuvieron en las Figuras (\ref{fig:s_760_MxC_simple_1.0}), (\ref{fig:s_760_MxC_double_1.0}), (\ref{fig:c_760_MxC_c_10}) y (\ref{fig:c_760_MxC_cuda_10}) se encuentran en el Apéndice \ref{apend_MxC_760}

\subsubsection{NVIDIA Geforce GTX 970}

Los tamaños de grilla que se utilizaron para realizar las pruebas de tiempo de ésta placa, tienen el rango de grilla de 16x16 nodos hasta 4096x4096 nodos en simple precisión y de 16x16 nodos hasta 2048x2048 nodos en doble precisión . La cantidad de \textit{thread blocks} que se utilizó fueron de 1 a 512.

Las Figuras (\ref{fig:s_970_MxC_simple_10}) y (\ref{fig:s_970_MxC_double_10}) muestran el \textit{Speed Up} obtenido comparando los códigos de \textsc{C} y \textsc{Cuda C}, donde la Figura (\ref{fig:s_970_MxC_simple_10}) está obtenida con simple precisión y la Figura (\ref{fig:s_970_MxC_double_10}) en doble precisión. El mejor resultado en ambos casos se obtuvo para un número de \textit{thread block} igual a 32, donde la mejora fue de 23.39 y 10.96 en simple y doble precisión respectivamente, para el mayor número de elementos de malla.

Las Figuras (\ref{fig:c_970_MxC_c_10}) y (\ref{fig:c_970_MxC_cuda_10}) muestran el \textit{Speed Up} obtenido comparando simple precisión y doble precisión, donde la Figura (\ref{fig:c_970_MxC_c_10}) está obtenida el código de \textsc{C} y la Figura (\ref{fig:c_970_MxC_cuda_10}) en el código de \textsc{Cuda C}.

En el código de \textsc{C} para el mayor número de elementos de la malla, el resultado de tiempos de cálculo en doble precisión es apenas 1,035 veces mayor que en  simple precisión. En contraste, al fijarse el resultado para un número de \textit{thread block} igual a 32 (el que mayor ganancia obtuvo), el tiempo de cálculo en doble precisión es 1,29 veces mayor que en simple precisión; para el mayor número de elementos de malla calculado.

\begin{figure}[htbp]
	\centering
	\includegraphics[width=\textwidth]{figs/cap4/s_970_MxC_simple_10}
	\caption{Speed Up realizado para el problema de la Construcción de Maxwell con la GPU NVIDIA Geforce GTX 970 en simple precisión, comparando los códigos de \textsc{C} y \textsc{Cuda C}.} 
	\label{fig:s_970_MxC_simple_10}	
\end{figure}

\begin{figure}[htbp]
	\centering
	\includegraphics[width=\textwidth]{figs/cap4/s_970_MxC_double_10}
	\caption{Speed Up realizado para el problema de la Construcción de Maxwell con la GPU NVIDIA Geforce GTX 970 en doble precisión, comparando los códigos de \textsc{C} y \textsc{Cuda C}.} 
	\label{fig:s_970_MxC_double_10}	
\end{figure}


\begin{figure}[htbp]
	\centering
	\includegraphics[width=\textwidth]{figs/cap4/c_970_MxC_c_10}
	\caption{Speed Up realizado para el problema de la Construcción de Maxwell con la GPU NVIDIA Geforce GTX 970 en en el código de \textsc{C}, comparando simple precisión y doble precisión.} 
	\label{fig:c_970_MxC_c_10}	
\end{figure}

\begin{figure}[htbp]
	\centering
	\includegraphics[width=\textwidth]{figs/cap4/c_970_MxC_cuda_10}
	\caption{Speed Up realizado para el problema de la Construcción de Maxwell con la GPU NVIDIA Geforce GTX 970 en en el código de \textsc{Cuda C}, comparando simple precisión y doble precisión.} 
	\label{fig:c_970_MxC_cuda_10}	
\end{figure}

Los datos de las Figuras (\ref{fig:s_970_MxC_simple_10}), (\ref{fig:s_970_MxC_double_10}), (\ref{fig:c_970_MxC_c_10}) y (\ref{fig:c_970_MxC_cuda_10}) se encuentran en las Tablas (\ref{tab:s_970_MxC_simple_10}), (\ref{tab:s_970_MxC_double_10}), (\ref{fig:c_970_MxC_c_10}) y (\ref{tab:c_970_MxC_cuda_10}) del Apéndice \ref{apend_MxC_970}


\subsection{Análisis}

La utilización de simple precisión en el código de \textsc{Cuda C} es más conveniente que doble precisión. Una de las razones es que no hay demasiada diferencia entre los valores que se pueden obtener, según los resultados obtenidos no difirieren más del 0,003 \%. Otra de las razones es que debido a las mejoras obtenidas en los tiempos de cálculo en las placas NVIDIA Geforce GTX 760 y NVIDIA Geforce GTX 970 en el código de \textsc{Cuda C} de 18.67 y 23.39 respectivamente en simple precisión que el código de \textsc{C}. Las mejoras en doble precisión de las ganancias son de 11.40 y 10.96 respectivamente en las placas mencionadas. Por lo que el resultado obtenido en simple precisión difiere apenas un 0.003 \% que en doble precisión, además siendo 1.68 y 1.29 veces más rapido según la GPU utilizada.

\newpage

\section{Estratificación de un fluido VdW con temperatura no uniforme}

\begin{figure}[htbp]
	\centering
	\includegraphics[width=0.3\textwidth]{figs/cap4/esquema_problema_VdW}
	\caption{Esquema del problema de la estratificación de un fluido VdW. Donde se muestra el alto (H) de la cavidad y el ancho (L).} 
	\label{fig:esquema_VdW}	
\end{figure}

Se quiere resolver el problema de tener una cavidad unidimensional en presencia de un fluido cuya EOS es la de Van der Waals; con temperatura  no uniforme y fuerza de gravedad no nula. Éste problema fue desarrollado por Berberan-Santos \cite{berberan2002liquid} y que Fogliatto \cite{fogliatto2019simulation} extendió. 

Se toma como coordenada del problema \textit{y}, teniéndose una temperatura fija $T_{0}$ en $y = 0$ y $T_{1}$ en $y = H$ como se observa en la Figura (\ref{fig:esquema_VdW}) en presencia de la gravedad.

El gradiente de presión surge de realizar el balance de momento en un volumen de control, obteniéndose:

\begin{align}
	\frac{d P}{d y} = - g M C(y)
\end{align}

siendo \textit{g} la gravedad, \textit{M} el peso molecular, $C = \frac{1}{v}$ la fracción molar, \textit{v} el volumen molar y \textit{P} la presión.

Realizando la adimensionalisación de $ P_r = \frac{P}{P_c}$ , $ T_r = \frac{T}{T_c}$, $c = C v_c$ y $E_r = \frac{M g y}{R T_c}$; siendo \textit{c} el punto crítico en el cuál comienza la coexistencia de las dos fases, se pueden reemplazar en el gradiente de presión y una Ecuación de estado para obtener una ecuación adimensional con la concentración molar distribuida \cite{fogliatto2019simulation}.



Si se elije la EOS de VdW de la Ec.(\ref{eq:VdW_P}), se obtiene la siguiente ecuación adimensional diferencial:

\begin{align}
	\frac{d c}{d E_r} = - \left[ c + \frac{d T_r}{d E_r} \left( \frac{c}{1 - \frac{c}{3}}\right) \right] \left[	\frac{1}{\frac{T_r}{{\left(1- \frac{c}{3}\right)}^2} - \frac{9}{4} c}  \right] 
	\label{eq:adim_dif}
\end{align} 

Si se resuelve iterativamente la Ec. (\ref{eq:adim_dif}) se puede obtener dos curvas adimensionales correspondientes al perfil de temperatura y al de concentraciones.

En la Figura (\ref{fig:VdW_val_rho_y}) se muestra la curva adimensional $ \rho_r - Y_r $ , siendo $\rho_r = \frac{\rho}{\rho_c}$ y  $Y_r = \frac{y}{H}$  y la Figura (\ref{fig:VdW_val_T_y}) muestra la curva adimensional $ T_r - Y_r $. Ambas para distintos valores de $T_0$, mientras que en la otra pared $T_1 = cte = 0.99 \> T_r$.

\begin{figure}[htbp]
	\centering
	\includegraphics[width=\textwidth]{figs/cap4/VdW_val_rho_y}
	\caption{Perfil de densidad adimensional a lo largo de la longitud de la pared,para distintos valores de condición de temperatura fija en $y = 0$, para un fluido de VdW con los parámetros $a = 0,5 $ y $b = 4,0 $, obtenida en simple precisión en la GPU NVIDIA Geforce GTX 760 en el código desarrollado en \textsc{C}.} 
	\label{fig:VdW_val_rho_y}	
\end{figure}

\begin{figure}[htbp]
	\centering
	\includegraphics[width=\textwidth]{figs/cap4/VdW_val_T_y}
	\caption{Perfil de temperatura adimensional a lo largo de la longitud de la pared, para distintos valores de condición de temperatura fija en $y = 0$,para un fluido de VdW con los parámetros $a = 0,5 $ y $b = 4,0 $, obtenida en simple precisión en la GPU NVIDIA Geforce GTX 760 en el código desarrollado en \textsc{C}.} 
	\label{fig:VdW_val_T_y}	
\end{figure}

\newpage

\subsection{Validación}

La validación de éste problema se hizo utilizando los parámetros $a =0,5$ y $b = 4,0$; para un tamaño de malla de 3 x 300 nodos y $T_0 = T_r$, variando $T_r$ con un paso de $0,025$ en el rango de $[0,6 - 0,975]$. Mostrándose en los gráficos únicamente los valores de $T_r = 0,6 ; 0,7 ; 0,8 \> y \> 0,9$.    El valor de $\mathbf{M}$ utilizado es el mismo que es de la Construcción de Maxwell, el resto de los parámetros se encuentran en el Apéndice (\ref{parametros_VdW}).

La Figura(\ref{fig:v_760_VdW_c_simple_rho_y}) y (\ref{fig:v_760_VdW_c_simple_T_y})  muestran la validación del código realizado en \textsc{C} para simple precisión en una GPU NVIDIA Geforce GTX 760; con distintos valores de $T_0$. 

El quiebre abrupto y discontínuo de la densidad a lo largo de la pared en la Figura (\ref{fig:v_760_VdW_c_simple_rho_y}) muestra la coexistencia de las dos fases, para la solución analítica. Debido a que se tiene una solución contínua y un poco difusa porque el modelo toma una cierta cantidad de nodos como separación entre ambas fases, se presenta una mayor diferencia en el ajuste de la curva obtenida con la analítica en el cambio de fase. Lo mismo sucede en la Figura (\ref{fig:v_760_VdW_c_simple_T_y}).

El tiempo de vuelo utilizado en cada uno de los nodos de la malla fue de 750000. 
El resultado obtenido para ambas curvas validades es exactamente igual en el código de \textsc{C} y \textsc{Cuda C}.

\begin{figure}[htbp]
	\centering
	\includegraphics[width=\textwidth]{figs/cap4/v_760_VdW_c_simple_rho_y}
	\caption{Perfil de densidad adimensional a lo largo de la longitud de la pared, para valores de $T_0 = T_r$ , siendo $T_r = 0,6 ; 0,7 ; 0,8 y 0,9$, para un fluido de VdW con los parámetros $a = 0,5 $ y $b = 4,0 $, obtenida en simple precisión en la GPU NVIDIA Geforce GTX 760 en el código desarrollado en \textsc{C}.}
	\label{fig:v_760_VdW_c_simple_rho_y}	
\end{figure}

\begin{figure}[htbp]
	\centering
	\includegraphics[width=\textwidth]{figs/cap4/v_760_VdW_c_simple_T_y}
	\caption{Perfil de temperatura adimensional a lo largo de la longitud de la pared, para valores de $T_0 = T_r$ , siendo $T_r = 0,6 ; 0,7 ; 0,8 \>y\> 0,9$, para un fluido de VdW con los parámetros $a = 0,5 $ y $b = 4,0 $, obtenida en simple precisión en la GPU NVIDIA Geforce GTX 760 en el código desarrollado en \textsc{C}.}
	\label{fig:v_760_VdW_c_simple_T_y}	
\end{figure}

\newpage

\subsection{Speed Up}

En la presente sección se muestran las mejora en el tiempo de cálculo realizados para el código de \textsc{C} y \textsc{Cuda C}. La comparación se realizó en simple y doble precisión; en las GPU NVIDIA Geforce GTX 760 y NVIDIA Geforce GTX 970. Se tomó una $T_r$ fija y se varió el tamaño de la grilla, de manera que ésta siempre tuviese un ancho constante de 300 nodos y variando la dimensión en el eje \textsc{Y} respetando un número de nodos de potencia de 2 (exceptuando el primer valor que es 3) . La cantidad de \textit{thread blocks} que se utilizó para realizar la comnparación en el código de \textsc{CUDA} fueron de potencia de 2.

\subsubsection{NVIDIA Geforce GTX 760}

Los tamaños de grilla que se utilizaron para realizar las pruebas de tiempo de ésta placa, tienen el rango de grilla de 3x300 nodos hasta 1638400x300 nodos. La cantidad de \textit{thread blocks} que se utilizó fueron de 1 a 512.

Las Figuras (\ref{fig:s_760_VdW_simple_10}) y (\ref{fig:s_760_VdW_double_10}) muestran el \textit{Speed Up} obtenido comparando los códigos de \textsc{C} y \textsc{Cuda C}, donde la Figura (\ref{fig:s_760_VdW_simple_10}) está obtenida con simple precisión y la Figura (\ref{fig:s_760_VdW_double_10}) en doble precisión. El mejor resultado en ambos casos se obtuvo para un número de \textit{thread block} igual a 64, donde la mejora fue de 13.26 y 7.88 en simple y doble precisión respectivamente, para el mayor número de elementos de malla.


\begin{figure}[htbp]
	\centering
	\includegraphics[width=\textwidth]{figs/cap4/s_760_VdW_simple_10}
	\caption{Speed Up realizado para el problema de la Estratificación de un fluido Van der Waals con la GPU NVIDIA Geforce GTX 760 en simple precisión, comparando los códigos de \textsc{C} y \textsc{Cuda C}.} 
	\label{fig:s_760_VdW_simple_10}	
\end{figure}

\begin{figure}[htbp]
	\centering
	\includegraphics[width=\textwidth]{figs/cap4/s_760_VdW_double_10}
	\caption{Speed Up realizado para el problema de la Estratificación de un fluido Van der Waals con la GPU NVIDIA Geforce GTX 760 en doble precisión, comparando los códigos de \textsc{C} y \textsc{Cuda C}.} 
	\label{fig:s_760_VdW_double_10}	
\end{figure}

\newpage

Las Figuras (\ref{fig:c_760_VdW_c_10}) y (\ref{fig:c_760_VdW_cuda_10}) muestran el \textit{Speed Up} obtenido comparando simple precisión y doble precisión, donde la Figura (\ref{fig:c_760_VdW_c_10}) está obtenida el código de \textsc{C} y la Figura (\ref{fig:c_760_VdW_cuda_10}) en el código de \textsc{Cuda C}. 

En el código de \textsc{C} para el mayor número de elementos de la malla, el resultado de tiempos de cálculo en doble precisión es apenas 1,052 veces mayor que en  simple precisión. En contraste, al fijarse el resultado para un número de \textit{thread block} igual a 64 (el que mayor ganancia obtuvo), el tiempo de cálculo en doble precisión es 1,77 veces mayor que en simple precisión; para el mayor número de elementos de malla calculado.

\begin{figure}[htbp]
	\centering
	\includegraphics[width=\textwidth]{figs/cap4/c_760_VdW_c_10}
	\caption{Speed Up realizado para el problema de la Estratificación de un fluido Van der Waals con la GPU NVIDIA Geforce GTX 760 en en el código de \textsc{C}, comparando simple precisión y doble precisión.} 
	\label{fig:c_760_VdW_c_10}	
\end{figure}

\begin{figure}[htbp]
	\centering
	\includegraphics[width=\textwidth]{figs/cap4/c_760_MxC_cuda_10}
	\caption{Speed Up realizado para el problema de la Estratificación de un fluido Van der Waals con la GPU NVIDIA Geforce GTX 760 en en el código de \textsc{Cuda C}, comparando simple precisión y doble precisión.} 
	\label{fig:c_760_VdW_cuda_10}	
\end{figure}

Los valores que se obtuvieron en las Figuras (\ref{fig:s_760_VdW_simple_10}), (\ref{fig:s_760_VdW_double_10}), (\ref{fig:c_760_VdW_c_10}) y (\ref{fig:c_760_VdW_cuda_10}) se encuentran en el Apéndice \ref{apend_VdW_760}

\newpage

\subsubsection{NVIDIA Geforce GTX 970}

Los tamaños de grilla que se utilizaron para realizar las pruebas de tiempo de ésta placa, tienen el rango de grilla de 3x300 nodos hasta 1638400x300 nodos. La cantidad de \textit{thread blocks} que se utilizó fueron de 1 a 512.

Las Figuras (\ref{fig:s_970_VdW_simple_10}) y (\ref{fig:s_970_VdW_double_10}) muestran el \textit{Speed Up} obtenido comparando los códigos de \textsc{C} y \textsc{Cuda C}, donde la Figura (\ref{fig:s_970_VdW_simple_10}) está obtenida con simple precisión y la Figura (\ref{fig:s_970_VdW_double_10}) en doble precisión. El mejor resultado en ambos casos se obtuvo para un número de \textit{thread block} igual a 32, donde la mejora fue de 15.95 y 13.29 en simple y doble precisión respectivamente, para el mayor número de elementos de malla.


\begin{figure}[htbp]
	\centering
	\includegraphics[width=\textwidth]{figs/cap4/s_970_VdW_simple_10}
	\caption{Speed Up realizado para el problema de la Estratificación de un fluido Van der Waals con la GPU NVIDIA Geforce GTX 970 en simple precisión, comparando los códigos de \textsc{C} y \textsc{Cuda C}.} 
	\label{fig:s_970_VdW_simple_10}	
\end{figure}

\begin{figure}[htbp]
	\centering
	\includegraphics[width=\textwidth]{figs/cap4/s_970_VdW_double_10}
	\caption{Speed Up realizado para el problema de la Estratificación de un fluido Van der Waals con la GPU NVIDIA Geforce GTX 970 en doble precisión, comparando los códigos de \textsc{C} y \textsc{Cuda C}.} 
	\label{fig:s_970_VdW_double_10}	
\end{figure}

\newpage

Las Figuras (\ref{fig:c_970_VdW_c_10}) y (\ref{fig:c_970_VdW_cuda_10}) muestran el \textit{Speed Up} obtenido comparando simple precisión y doble precisión, donde la Figura (\ref{fig:c_970_VdW_c_10}) está obtenida el código de \textsc{C} y la Figura (\ref{fig:c_970_VdW_cuda_10}) en el código de \textsc{Cuda C}. 

En el código de \textsc{C} para el mayor número de elementos de la malla, el resultado de tiempos de cálculo en doble precisión es apenas 1,045 veces mayor que en  simple precisión. En contraste, al fijarse el resultado para un número de \textit{thread block} igual a 32 (el que mayor ganancia obtuvo), el tiempo de cálculo en doble precisión es 1,25 veces mayor que en simple precisión; para el mayor número de elementos de malla calculado.

\begin{figure}[htbp]
	\centering
	\includegraphics[width=\textwidth]{figs/cap4/c_970_VdW_c_10}
	\caption{Speed Up realizado para el problema de la Estratificación de un fluido Van der Waals con la GPU NVIDIA Geforce GTX 970 en en el código de \textsc{C}, comparando simple precisión y doble precisión.} 
	\label{fig:c_970_VdW_c_10}	
\end{figure}

\begin{figure}[htbp]
	\centering
	\includegraphics[width=\textwidth]{figs/cap4/c_970_VdW_cuda_10}
	\caption{Speed Up realizado para el problema de la Estratificación de un fluido Van der Waals con la GPU NVIDIA Geforce GTX 970 en en el código de \textsc{Cuda C}, comparando simple precisión y doble precisión.} 
	\label{fig:c_970_VdW_cuda_10}	
\end{figure}

Los valores que se obtuvieron en las Figuras (\ref{fig:s_970_VdW_simple_10}), (\ref{fig:s_970_VdW_double_10}), (\ref{fig:c_970_VdW_c_10}) y (\ref{fig:c_970_VdW_cuda_10}) se encuentran en el Apéndice \ref{apend_VdW_970}

\subsection{Análisis}



\newpage


\begin{figure}[h!]
	\centering % <-- added
	\begin{subfigure}{0.25\textwidth}
		\includegraphics[width=\linewidth]{figs/cap4/bubble_5}
		\caption{t = 5000}
		\label{fig:1}
	\end{subfigure}\hfil % <-- added
	\begin{subfigure}{0.25\textwidth}
		\includegraphics[width=\linewidth]{figs/cap4/bubble_10}
		\caption{t = 10000}
		\label{fig:2}
	\end{subfigure}\hfil % <-- added
	\begin{subfigure}{0.25\textwidth}
		\includegraphics[width=\linewidth]{figs/cap4/bubble_20}
		\caption{t = 20000}
		\label{fig:3}
	\end{subfigure}
	
	\medskip
	\begin{subfigure}{0.25\textwidth}
		\includegraphics[width=\linewidth]{figs/cap4/bubble_25}
		\caption{t = 25000}
		\label{fig:4}
	\end{subfigure}\hfil % <-- added
	\begin{subfigure}{0.25\textwidth}
		\includegraphics[width=\linewidth]{figs/cap4/bubble_30}
		\caption{t = 30000}
		\label{fig:5}
	\end{subfigure}\hfil % <-- added
	\begin{subfigure}{0.25\textwidth}
		\includegraphics[width=\linewidth]{figs/cap4/bubble_35}
		\caption{t = 35000}
		\label{fig:6}
	\end{subfigure}
	
	\medskip
	\begin{subfigure}{0.25\textwidth}
		\includegraphics[width=\linewidth]{figs/cap4/bubble_40}
		\caption{t = 40000}
		\label{fig:7}
	\end{subfigure}\hfil % <-- added
	\begin{subfigure}{0.25\textwidth}
		\includegraphics[width=\linewidth]{figs/cap4/bubble_45}
		\caption{t = 45000}
		\label{fig:8}
	\end{subfigure}\hfil % <-- added
	\begin{subfigure}{0.25\textwidth}
		\includegraphics[width=\linewidth]{figs/cap4/bubble_50}
		\caption{t = 50000}
		\label{fig:9}
	\end{subfigure}
	\caption{Creación y desarrollo de una burbuja a partir de una superficie horizontal calefaccionada.}
	\label{fig:secuencia_burbujas}
\end{figure}

\section{Generación de burbujas sobre una superficie horizontal calefaccionada}

Mediante éste problema se quiere reproducir el fenómeno de ebullición nucleada. El trabajo en los dos problemas anteriores constituyen pasos para validar por partes el código. Ésto se debe a que el problema de la Construcción de Maxwell utiliza solamente una ecuación pseudopotencial, mientras que el problema de la estratificación de un fluido VdW utiliza las dos operaciones pseudopotenciales planteadas.

La diferencia del problema de la estratificación de VdW con el de la generación de burbujas, radica en que el primero es unidimensional, el segundo es bidimensional. A su vez, la aplicación del código cambia únicamente en realizar una función que imponga como condición de contorno un área de la superficie calefaccionada.

El valor de $\mathbf{M}$ utilizado es el mismo que los probelmas anteriores, el resto de los parámetros se encuentran en el Apéndice (\ref{parametros_bub}).

Los resultados que se obtuvieron se encuentra en las Figura (\ref{fig:secuencia_burbujas}), donde los tiempos de vuelo mostrados son de 5000, 10000, 20000, 25000, 30000, 35000, 40000, 45000 y 5000 pasos. El tamaño de grilla utilizado fue de L = 300 y H = 500, si se toma como esquema el mostrado en la Figura(\ref{fig:esquema_VdW}). La superficie calefaccionada se encuentra en el centro del lado L y posee un tamaño de 24 elementos de malla, siendo la temperatura de calefacción $T_{heat} = T_c$. El resto de la superficie se encuentra a $T = 0.8 T_c$ y la parte superior de la cavidad está a $T = 0.99 T_c$.

\newpage

\section{Resultados con \textsc{Python}}

La realización de los códigos numéricos mediante \textsc{C} y \textsc{Cuda C} tenian como primera finalidad conocer cuál es la ganancia en tiempo de cálculo realizando la implementación en GPU. Como segunda finalidad de realizar un código en \textsc{Cuda C} es compilar en código en una biblioteca para ser utilizado en \textsc{Python}.

Por falta de tiempo, no se pudo hacer un código en \textsc{Python} que resuelva los problemas mecionados en el presente capítulo. Lo que se realizó fue un \textit{Speed Up} 
de uno de las \textit{kernel} del código, el cual resuelve la Ec.(\ref{eq:rho}):

\begin{equation*}
	\rho = \sum_{\alpha} f_{\alpha}
\end{equation*}

Para éste análisis se inicializó a $f$ con valores aleatorios, haciendo una repetición de la ejecución de la función 100000 veces para realizar la toma de tiempo. Se varió el tamaño de la grilla, de manera que ésta siempre fuese cuadrada, respetando un número de nodos de potencia de 2 en los lados del cuadrado. La cantidad de \textit{thread blocks} que se utilizó para realizar la comparación en el código fue de potencia de 2. El tamaño de la grilla varió de 16x16 hasta 4096x4906 y la cantidad de \textit{thread blocks} utilizado es de 1 a 512.


\title{\textbf{NVIDIA Geforce GTX 760}}

La Figura (\ref{fig:s_cuda_760_test_simple_10}) muestra la ganancia obtenida en el código de \textsc{Cuda C} sobre \textsc{C}. Para 64 \textit{thread blocks} se obtuvo la mayor ganancia, siendo de 3.198. La Figura (\ref{fig:s_py_c_760_test_simple_10}) muestra la ganancia obtenida en el código de \textsc{PyCuda} sobre \textsc{C}. Para 128 \textit{thread blocks} se obtuvo la mayor ganancia, siendo de 2.867. La Figura (\ref{fig:s_py_760_test_simple_10}) muestra cuanto más se demora el código de \textsc{PyCuda} con respecto al de \textsc{Cuda C}. Para 128 y 512  \textit{thread blocks} se obtuvo el menor incremento de tiempos de cálculo, siendo de 1.144. 
%Coincide que la mayor ganancia con respecto al código de \textsc{C} por parte de \textsc{Python}, sea la menor en incrementar su tiempo de cálculo en comparación del código de \textsc{Cuda C}.

\begin{figure}[h!]
	\centering
	\includegraphics[width=\textwidth]{figs/cap4/s_cuda_760_test_simple_10}
	\caption{Speed Up realizado entre CUDA y C para la función de obtensión de la densidad con la GPU NVIDIA Geforce GTX 760 en simple precisión.} 
	\label{fig:s_cuda_760_test_simple_10}	
\end{figure}

\begin{figure}[h!]
	\centering
	\includegraphics[width=\textwidth]{figs/cap4/s_py_c_760_test_simple_10}
	\caption{Speed Up realizado entre PyCUDA y C para la función de obtensión de la densidad con la GPU NVIDIA Geforce GTX 760 en simple precisión.} 
	\label{fig:s_py_c_760_test_simple_10}	
\end{figure}


\begin{figure}[h!]
	\centering
	\includegraphics[width=\textwidth]{figs/cap4/s_py_760_test_simple_10}
	\caption{Speed Up realizado entre PyCUDA y CUDA para la función de obtensión de la densidad con la GPU NVIDIA Geforce GTX 760 en simple precisión.} 
	\label{fig:s_py_760_test_simple_10}	
\end{figure}



%
\title{\textbf{NVIDIA Geforce GTX 970}}

La Figura (\ref{fig:s_cuda_970_test_simple_10}) muestra la ganancia obtenida en el código de \textsc{Cuda C} sobre \textsc{C}. Para 64 \textit{thread blocks} se obtuvo la mayor ganancia, siendo de 4.815. La Figura (\ref{fig:s_py_c_970_test_simple_10}) muestra la ganancia obtenida en el código de \textsc{PyCuda} sobre \textsc{C}. Para 256 \textit{thread blocks} se obtuvo la mayor ganancia, siendo de 4.242. La Figura (\ref{fig:s_py_970_test_simple_10}) muestra cuanto más se demora el código de \textsc{PyCuda} con respecto al de \textsc{Cuda C}. Para 512  \textit{thread blocks} se obtuvo el menor incremento de tiempos de cálculo, siendo de 1.087; y para 256 \textit{thread blocks} es de 1.104 veces.

\begin{figure}[htbp]
	\centering
	\includegraphics[width=\textwidth]{figs/cap4/s_cuda_970_test_simple_10}
	\caption{Speed Up realizado entre CUDA y C para la función de obtensión de la densidad con la GPU NVIDIA Geforce GTX 970 en simple precisión.} 
	\label{fig:s_cuda_970_test_simple_10}	
\end{figure}

\begin{figure}[htbp]
	\centering
	\includegraphics[width=\textwidth]{figs/cap4/s_py_c_970_test_simple_10}
	\caption{Speed Up realizado entre PyCUDA y C para la función de obtensión de la densidad con la GPU NVIDIA Geforce GTX 970 en simple precisión.} 
	\label{fig:s_py_c_970_test_simple_10}	
\end{figure}

\begin{figure}[htbp]
	\centering
	\includegraphics[width=\textwidth]{figs/cap4/s_py_970_test_simple_10}
	\caption{Speed Up realizado entre PyCUDA y CUDA para la función de obtensión de la densidad con la GPU NVIDIA Geforce GTX 970 en simple precisión.} 
	\label{fig:s_py_970_test_simple_10}	
\end{figure}

Los valores de las Figuras (\ref{fig:s_cuda_760_test_simple_10}), (\ref{fig:s_py_c_760_test_simple_10}), (\ref{fig:s_py_760_test_simple_10}), (\ref{fig:s_cuda_970_test_simple_10}), (\ref{fig:s_py_c_970_test_simple_10}) y (\ref{fig:s_py_970_test_simple_10}) se encuentran en las Tablas (\ref{tab:s_cuda_760_test_simple_10}), (\ref{tab:s_py_c_760_test_simple_10}), (\ref{tab:s_py_760_test_simple_10}), (\ref{tab:s_cuda_970_test_simple_10}), (\ref{tab:s_py_c_970_test_simple_10}) y (\ref{tab:s_py_970_test_simple_10}) respectivamente, del Apéndice (\ref{ap_python}).
\section{Análisis}

Según los resultados obtenidos, se espera que utilizando la cantidad de \textit{thread blocks} que maximicen la ganancia en un código de \textsc{Python} sobre el código de \textsc{C}, el primero posea un incremento de un  15 \%  u 10 \% en comparación con el código de \textsc{Cuda C}, si es que se utiliza una GPU NVIDIA Geforce GTX 760/970 respectivamente.

Por lo que no se pierde mucha eficiencia programando en el lenguaje de \textsc{Python}.




%%% Local Variables: 
%%% mode: latex
%%% TeX-master: "template"
%%% End: 

\chapter{Conclusiones generales}
\graphicspath{{figs/cap4/}}
\label{cap5}

En el presente trabajo se realizó un código numérico para resolver problemas de transferencia de calor en flujos multifásicos con cambio de fase, en donde el método utilizado es el de lattice Boltzmann de dos ecuaciones pseudo-pontencial y operador MRT desarrollado por Fogliatto et al. \cite{fogliatto2019transferencia}. El modelo desarrollado resuelve problemas con discretización espacial de dominio regular y el tipo de modelo de grilla es el denominado D2Q9.

El código realizado utilizó el software \textsc{CMake} para preparar la compilación del mismo, el cual permite desarrollar proyectos que posean una gran cantidad de directorios de forma simple. El código se encuentra implementado en tres lenguajes de programación, siendo ellos \textsc{C}, \textsc{Cuda C} y \textsc{Python}. La compilación correspondiente a \textsc{C} se hizo en bibliotecas del tipo \textit{shared} mientras que las bibliotecas de \textsc{Cuda C} eran del tipo \textit{static}. Además los \textit{kernels} realizados en \textsc{Cuda C} son compilados en formato \textsc{ptx} para que \textsc{Python} los utilice mediante su módulo llamado \textsc{PyCuda}.

Las instrucciones de compilación de \textsc{CMake} se realizan por medio de un archivo de configuración principal llamado CMakeLists.txt. Se concretaron las siguientes opciones de configuración para compilar el código:

\begin{itemize}
	\item selección entre precisión simple o doble. 
	\item detección automática de la arquitectura de la GPU en la PC que se compila el código.
	\item compilación en \textsc{C} o en \textsc{C} y \textsc{Cuda C}.
\end{itemize}

El código numérico fue administrado mediante el software \textsc{Git} que permite el control de versiones de forma fácil y eficiente. El repositorio utilizado es el de la página web \textsc{GitHub} y puede ser descargado en \url{https://github.com/efogliatto/LBCUDA_Test}. En el transcurso del proyecto se comprobó que una buena práctica para desarrollar código es mediante el uso de ramas. En este proyecto se trabajó sobre tres ramas: \textit{master}, \textit{develop} y \textit{feature}. La versión 1.0 es la que se encontraba en la rama \textit{master} al finalizar el presente trabajo.

La validación del código fue realizado en dos PC diferentes, la primera contaba con una CPU Intel Core i7-3770 con una GPU NVIDIA GeForce GTX 760 y la segunda disponía de una CPU Intel Core i7-4770 con una GPU NVIDIA GeForce GTX 970. A su vez la validación se concretó en variables de simple precisión y doble precisión.

Los problemas físicos  que se utilizaron para realizar la validación del código son los siguientes:

\begin{itemize}
    
    \item construcción de Maxwell, el cual permite obtener las densidades de coexistencia de fases de un fluido a partir de la selección de una ecuación de estado para una dada condición de  presión, temperatura y densidad iniciales.

    \item estratificación de un fluido Van der Waals con temperatura no uniforme (1D). Este problema posee campo gravitatorio y temperaturas fijas en los extremos.

    \item generación de burbujas en una placa horizontal calefaccionada (2D).

\end{itemize}

\section{Construcción de Maxwell}

Para el problema de la Construcción de Maxwell, se reprodujo el resultado que obtuvo Fogliatto et al. \cite{fogliatto2019simulation}, para el cual el valor del parámetro $\sigma = 0,125$ del operador MRT es el que ajusta mejor la curva de coexistencia de fases para un fluido con la Ecuación de estado de VdW de parámetros $ a = 0,5 $ y $ b = 4,0 $. El tiempo de cálculo que lleva obtener  cada una de las densidades de coexistencia de fase, fue alrededor de 1724 segundos.

Para la GPU NVIDIA Geforce GTX 760 en simple precisión se obtuvo una ganancia del código realizado en \textsc{CUDA C} de 18.67 veces con respecto al código de \textbf{C} para un número de 64 \textit{thread block} y la cantidad de 4194304 ($2^{22}$) elementos de malla. Mientras que en la  GPU NVIDIA Geforce GTX 970 en las mismas condiciones se obtuvo una ganancia de 23.39 utilizando 32 \textit{thread block} .

En doble precisión, la ganancia de la GPU NVIDIA Geforce GTX 760 con 64 \textit{thread block} fue de 11.40 mientras que en GPU NVIDIA Geforce GTX 970 con 32 \textit{thread block} se obtuvo 10,96. Por el comportamiento que se observó en los resultados, la GPU NVIDIA Geforce GTX 760 llegó a una ganancia máxima, mientras que la GPU NVIDIA Geforce GTX 970 posee la tendencia de aumentar su ganancia a un número de elementos de malla mayor.

\newpage

Por otro lado, se compararon los resultados obtenidos en simple y doble precisión en la validación de las curvas de coexistencia, siendo el error que causa reducir la precisión de 0,003 \%.  Debido a que el error que introduce la precisión simple es despreciable, y puesto que el tiempo que se demora en doble precesión con respecto a simple precisión es de 1,68 y 1,29 según se utilice GPU NVIDIA Geforce GTX 760 y GPU NVIDIA Geforce GTX 970 respectivamente, se recomienda la utilización de simple precisión.

\section{Estratificación de un fluido VdW (1D)}

Para el problema unidimensional de la estratificación de un fluido VdW, se pudieron verificar los perfiles de densidad $\rho_r$ y de temperatura $T_r$ a lo largo de la cavidad. En donde el tiempo de cálculo que lleva realizar cada perfil es cercano a 776 segundos.

La mayor ganancia que se obtuvo para la GPU NVIDIA Geforce GTX 760 y GPU NVIDIA Geforce GTX 970 en simple precisión fue de 13.26 y 15.95, siendo para doble precisión en 7.88 y 13.29, tomando como comparación el código de \textsc{Cuda C} con el código de \textsc{C}. En todos los casos con 64 \textit{thread block}.

Por otro lado, para una cantidad de 32 \textit{thread block} el código de \textsc{Cuda C}  en simple precisión es 1.77 y 1.25 veces más rápido que en doble precisión para  las GPU NVIDIA Geforce GTX 760 y GPU NVIDIA Geforce GTX 970 respectivamente. 

Cabe destacar que en el problema de estratificación, que implica la resolución de dos ecuaciones, se observaron ganancias inferiores respecto a los problemas con una única ecuación.

\section{Generación de burbujas en una superficie horizontal calefaccionada (2D)}

A partir del problema de la estratificación de un fluido VdW con temperatura no uniforme, el cual es unidimensional, se pudo realizar una pequeña modificación en el código, para agregar una condición de contorno de calefacción. En esencia el código es exactamente el mismo y puede reproducir el comportamiento de generación de una burbuja en el proceso de ebullición. Por lo tanto, esto demuestra que este fenómeno complejo puede ser resuelto mediante LBM como el utilizado en este trabajo.

\newpage

\section{Eficiencia en \textsc{Python}}

En el presente trabajo se implementó mediante el módulo \textsc{PyCuda} de \textsc{Python} uno de los \textit{kernels} del código de \textsc{Cuda C}, y así probar la performance del código con la utilización de dicho módulo. Se comparó el tiempo de cálculo de \textsc{Python} para el \textit{kernel} obtenido respecto al tiempo de cálculo del mismo en \textsc{Cuda C} en simple precisión. El resultado obtenido es que el incremento porcentual del tiempo de cálculo del código en \textsc{Python} con respecto al código de \textsc{Cuda C} es de 15 \% y 10 \% para las GPU NVIDIA Geforce GTX 760 y GPU NVIDIA Geforce GTX 970 respectivamente.

\section{Trabajo futuro}

$\bullet$ Una de las líneas de desarrollo para este trabajo es la mejora en los \textit{kernel} del código de \textsc{Cuda C} para que la ganancia en los tiempos de cálculo con respecto al del código de \textsc{C} aumente. Una de las formas de mejorar el rendimiento puede ser mediante la utilización de \textit{shared memory} ó \textit{local memory} de los \textit{thread block}. 

$\bullet$ El código fue realizado para que la implementación en un modelo de grilla D3Q15 pueda hacerse de manera sencilla, sin involucrar grandes modificaciones, en donde sólo se tiene que agregar en el directorio \textit{latticeModel} este tipo de modelo. 

%$\bullet$ Realizar 

%Por otro lado se puede trabajar en el desarrollo de una forma distinta de almacenar la información de los nodos vecinos, debido a que cuando se accede al registro de la memoria para realizar el proceso de \textit{streaming}, no toda la información que toma al acceder sirve para el cálculo del \textit{streaming} de ése nodo. Por lo que 
%
%
% ya que cuando se realiza la lectura de los mismos no toda la memoria que es adquirida e utiliza toda la información y se debe investigar como almacenar la informació para queal momento de la lectura del registro correspondiente de memoria se maximice la utilización de ésa lectura.

%Una de las cosas que queda por investigar , es el almacenamiento de los valores de la función de distribución de poblaciones, debido a que en nuestro problema se poseen 3 matrices con la información. una de ellas es la matriz de vecinosy la siguiente es las de poblaciones. Puesto a que depende de cómo es la etiqueta que se realizan a los vecinos, éstos irán a ser buscados en la memoria de la maquina. Dependiendo de cómo estén almacenados los lugares de la memoria de los nodos vecinos, éstos pueden tardart más o menios. de ahí surge la posibilidad/idea de que se distribuya de una manera distinta la forma de almacenar la informacion de lkis nodos vecinos y así realizzar un al profiling que haga que el código tenga una mayor ganancia.

$\bullet$ A partir de la implementación en el módulo \textsc{PyCuda} de \textsc{Python} de uno de los \textit{kernels} del código de \textsc{Cuda C}, se puede realizar la implementación completa del código en \textsc{Python}.

%$\bullet$ Por cuestión de tiempo no se llegó a implementar un código en \textsc{Python} utilizando la biblioteca \textsc{PyCuda}. Se mostró la implementacion para uno de los \textit{kernel}, pudiendo continuar el proyecto a partir de ahí. 

$\bullet$ También se puede realizar una interfaz gráfica mediante \textsc{Python} para que el usuario pueda operar el código sin necesidad de saber utilizar la terminal. La compilación del código puede realizarse de forma tal que pueda usarse en el sistema operativo \textit{Windows}, ya que actualmente se utiliza en \textit{Linux}. 

\appendix
%\chapter{Tiempos de cálculo}
\label{ap:tiempos}

En el presente apéndice se indican los tiempos de cálculo obtenidos para las distintas 
%\chapter{Resultados de Speed Up para la PC con la CPU i7-4770 y GPU NVIDIA GeForce GTX 970}
\label{ap:970}

\section{Construcción de Maxwell (2D)}
\label{ap:970_MXC}

\begin{figure}[H]
	\centering
	\includegraphics[width=0.99\textwidth]{figs_2/cap4/s_970_MxC_simple_10}
	\caption{SU realizado para el problema de la Construcción de Maxwell en simple precisión con una CPU Intel Core i7-4770 y GPU NVIDIA GeForce GTX 970.} 
	\label{fig:s_970_MxC_simple_10}	
\end{figure}

\begin{figure}[H]
	\centering
	\includegraphics[width=0.99\textwidth]{figs_2/cap4/s_970_MxC_double_10}
	\caption{SU realizado para el problema de la Construcción de Maxwell en doble precisión con una CPU Intel Core i7-4770 y GPU NVIDIA GeForce GTX 970.} 
	\label{fig:s_970_MxC_double_10}	
\end{figure}

\begin{figure}[H]
	\centering
	\includegraphics[width=\textwidth]{figs_2/cap4/c_970_MxC_cuda_10}
	\caption{$SU_p$ realizado para el problema de la Construcción de Maxwell en en el código de \textsc{Cuda C} con la GPU NVIDIA GeForce GTX 970.} 
	\label{fig:c_970_MxC_cuda_10}	
\end{figure}

\newpage
\section{Estratificación de un fluido VdW (1D)}
\label{ap:970_VDW}

\begin{figure}[H]
	\centering
	\includegraphics[width=\textwidth]{figs_2/cap4/s_970_VdW_simple_10}
	\caption{SU realizado para el problema de la Estratificación de un fluido Van dar Waals en simple precisión con una CPU Intel Core i7-4770 y GPU NVIDIA GeForce GTX 970.} 
	\label{fig:s_970_VdW_simple_10}	
\end{figure}

\begin{figure}[H]
	\centering
	\includegraphics[width=\textwidth]{figs_2/cap4/s_970_VdW_double_10}
	\caption{SU realizado para el problema de la Estratificación de un fluido Van dar Waals en doble precisión con una CPU Intel Core i7-4770 y GPU NVIDIA GeForce GTX 970.} 
	\label{fig:s_970_VdW_double_10}	
\end{figure}

\begin{figure}[H]
	\centering
	\includegraphics[width=\textwidth]{figs_2/cap4/c_970_VdW_cuda_10}
	\caption{$SU_p$ realizado para el problema de la Estratificación de un fluido Van der Waals en en el código de \textsc{Cuda C} con la GPU NVIDIA GeForce GTX 970.} 
	\label{fig:c_970_VdW_cuda_10}	
\end{figure}
\chapter{Actividades relacionadas con la Práctica Profesional Supervisada y de Proyecto y Diseño}


\section{Práctica profesional supervisada}

La práctica profesional supervisada se llevó a cabo en el Departamento de Mecánica Computacional del Centro Atómico Bariloche, durante el último año de Ingeniería Mecánica. La misma fue supervisada por el Mgter. Ezequiel Fogliatto y el Ing. Pablo Argañaras.


\section{Proyecto y diseño}

Las actividades de proyecto y diseño (PyD) realizadas para llevar a cabo el presente Proyecto Integrador de la carrera Ingeniería Mecánica fueron:

\begin{enumerate}
	\item 	Identificación de la fenomenología a resolver, desarrollado en el Capítulo 2.
	\item   Implementación de una herramienta computacional combinando C, CUDA C y Python, explicada en el Capítulo 3.	
	\item   Validación de la herramienta mediante la resolución de problemas con solución analítica o conocida, desarrollada en el Capítulo 4.
	
\end{enumerate}

%\chapter{}
\label{ap}

En el presente apéndice se muestra cuál es la matriz $\mathbf{M}$ en el modelo de LBM. También se encuentran diferenciados según el problema resuelto, los valores de $\mathbf{\Lambda}$ y $\mathbf{Q}$.

\begin{equation}
\mathbf{M} =
\begin{bmatrix}
	1 & 1 & 1 & 1 & 1 & 1 & 1 & 1 & 1 \\
   -4 &-1 &-1 &-1 &-1 & 2 & 2 & 2 & 2 \\
    4 &-2 &-2 &-2 &-2 & 1 & 1 & 1 & 1 \\
    0 & 1 & 0 &-1 & 0 & 1 &-1 &-1 & 1 \\
    0 &-2 & 0 & 2 & 0 & 1 &-1 &-1 & 1 \\
    0 & 0 & 1 & 0 &-1 & 1 & 1 &-1 &-1 \\
    0 & 0 &-2 & 0 & 2 & 1 & 1 &-1 &-1 \\
    0 & 1 &-1 & 1 &-1 & 0 & 0 & 0 & 0 \\    
    0 & 0 & 0 & 0 & 0 & 1 &-1 & 1 &-1 \\        
\end{bmatrix}
\end{equation}

\section{Construcción de Maxwell}
\label{parametros_MxC}

\begin{equation}
G = -1.0 \quad c = 1.0 \quad \sigma = 0.125 \quad a = 0.5 \quad b = 4.0 
\end{equation}

\begin{equation}
\mathbf{g} = (0.0 \quad 0.0 \quad 0.0 ) \qquad \rho_c = \frac{1}{12} \qquad T_c = 0.037037037
\end{equation}

\section{Estratificación de un fluido VdW}
\label{parametros_VdW}

\begin{equation}
	diag(\mathbf{\Lambda}) = 
	\begin{bmatrix}
		1.0 & 0.8 & 1.1 & 1.0 & 1.1 & 1.0 & 1.1 & 0.8 & 0.8 \\
	\end{bmatrix}
\end{equation}

\begin{equation}
diag(\mathbf{Q}) = 
	\begin{bmatrix}
		1.0 & 1.0 & 1.0 & 0.8 & 1.0 & 0.8 & 1.0 & 1.0 & 1.0 \\
	\end{bmatrix}
\end{equation}

\begin{equation}
	\alpha_{1} = 1.0 \qquad 	\alpha_{2} = 1.0 \qquad C_{v} = 1.0
\end{equation}

\begin{equation}
	G = -1.0 \quad c = 1.0 \quad \sigma = 0.125 \quad a = 0.5 \quad b = 4.0 
\end{equation}

\begin{equation}
\mathbf{g} = (0.0 \quad-1.234567e^{-7}\quad 0.0 ) \qquad \rho_c = \frac{1}{12} \qquad T_c = 0.037037037
\end{equation}

\section{Generación de burbujas}
\label{parametros_bub}

\begin{equation}
	diag(\mathbf{\Lambda}) = 
		\begin{bmatrix}
			1.0 & 1.25 & 1.0 & 1.0 & 1.1 & 1.0 & 1.1 & 1.3 & 1.3 \\
		\end{bmatrix}
\end{equation}

\begin{equation}
diag(\mathbf{Q}) = 
\begin{bmatrix}
1.0 & 1.0 & 1.0 & 1.55 & 1.0 & 1.55 & 1.0 & 1.0 & 1.0 \\
\end{bmatrix}
\end{equation}

\begin{equation}
\alpha_{1} = -2.0 \qquad 	\alpha_{2} = 2.0 \qquad C_{v} = 5.0
\end{equation}


\begin{equation}
G = -1.0 \quad c = 1.0 \quad \sigma = 0.125 \quad a = 1.0 \quad b = 4.0 
\end{equation}

\begin{equation}
\mathbf{g} = (0.0 \quad -8.0^{-6} \quad 0.0 ) \qquad \rho_c = \frac{1}{12} \qquad T_c = 0.074074074
\end{equation}
%\chapter{}
\label{ap1}

En el presente apéndice se encuentran las Tablas que contienen los valores de Speed Up entre el código realizado en C y CUDA C; para el problema de la Construcción de Maxwell.
Listándose para las dos placas gráficas utilizadas en el trabajo, diferenciándose en simple y doble precisión.
Además se encuentra la tabla que hace la comparación de Speed Up de simple y doble precisión según el código y la GPU utilizada.


\label{apend_MxC}

\title{\textbf{GPU NVIDIA GEFORCE GTX 760}}

\label{apend_MxC_760}

\begin{table}[h!]
    \begin{tabular}{|c|c|c|c|c|c|c|c|c|}
    \hline
                   & \multicolumn{8}{c|}{\textbf{NUMERO DE ELEMENTOS DE LA MALLA}} \\ \hline
    \textbf{BLOCK} & $2^8$ & $2^10$& $2^12$& $2^14$& $2^16$& $2^18$& $2^20$& $2^22$\\ \hline
    1              & 0.57  & 0.69  & 0.74  & 0.78  & 0.78  & 0.79  & 0.78  & 0.78  \\ \hline
    2              & 0.82  & 1.23  & 1.42  & 1.51  & 1.52  & 1.54  & 1.52  & 1.52  \\ \hline
    4              & 1.29  & 2.14  & 2.63  & 2.88  & 2.93  & 2.97  & 2.93  & 2.94  \\ \hline
    8              & 1.36  & 3     & 4.46  & 5.24  & 5.44  & 5.55  & 5.49  & 5.5   \\ \hline
    16             & 1.33  & 4.53  & 7.13  & 8.87  & 9.52  & 9.85  & 9.79  & 9.8   \\ \hline
    32             & 1.17  & 4.36  & 8.65  & 12.45 & 14.42 & 15.62 & 15.7  & 15.73 \\ \hline
    64             & 1.17  & 4.34  & 10.36 & 13.64 & 16.72 & 18.43 & 18.59 & 18.67 \\ \hline
    128            & 1.16  & 4.26  & 10.07 & 11.83 & 14.49 & 16.25 & 16.61 & 16.72 \\ \hline
    256            & 1.08  & 4.23  & 10.04 & 11.39 & 13.94 & 15.31 & 15.5  & 15.54 \\ \hline
    512            & 1.08  & 3.31  & 8.19  & 11.21 & 12.65 & 13.23 & 13.22 & 13.27 \\ \hline
    \end{tabular}
    \caption{Speed Up realizado para el problema de la Construcción de Maxwell con la GPU NVIDIA Geforce GTX 760 en simple precisión}
    \label{tab:s_760_MxC_simple_10}
    \end{table}

\begin{table}[h!]
\centering
    \begin{tabular}{|c|c|c|c|c|c|c|c|c|}
    \hline
                   & \multicolumn{8}{c|}{\textbf{NUMERO DE ELEMENTOS DE LA MALLA}} \\ \hline
    \textbf{BLOCK} & $2^{8}$ & $2^{10}$& $2^{12}$& $2^{14}$& $2^{16}$& $2^{18}$& $2^{20}$& $2^{22}$\\ \hline
    1              & 0.48  & 0.57  & 0.61  & 0.63  & 0.65  & 0.65  & 0.65  & 0.65  \\ \hline
    2              & 0.72  & 1.02  & 1.17  & 1.22  & 1.26  & 1.27  & 1.26  & 1.26  \\ \hline
    4              & 1.13  & 1.79  & 2.14  & 2.3   & 2.39  & 2.42  & 2.39  & 2.39  \\ \hline
    8              & 1.19  & 2.49  & 3.56  & 4.08  & 4.33  & 4.41  & 4.37  & 4.36  \\ \hline
    16             & 1.12  & 3.72  & 5.63  & 6.68  & 7.2   & 7.45  & 7.41  & 7.34  \\ \hline
    32             & 0.89  & 3.25  & 6.5   & 9     & 10.22 & 10.64 & 10.53 & 10.51 \\ \hline
    64             & 0.89  & 3.25  & 8.15  & 9.37  & 10.32 & 10.9  & 10.95 & 11.4  \\ \hline
    128            & 0.89  & 3.22  & 8.11  & 7.24  & 7.88  & 8.23  & 8.11  & 8.21  \\ \hline
    256            & 0.82  & 3.19  & 8.13  & 7.08  & 7.8   & 8.2   & 8.19  & 8.29  \\ \hline
    512            & 0.82  & 2.7   & 6.67  & 6.92  & 7.46  & 7.95  & 8.11  & 8.29  \\ \hline
    \end{tabular}
    \caption{Speed Up realizado para el problema de la Construcción de Maxwell con la GPU NVIDIA Geforce GTX 760 en doble precisión}
    \label{tab:s_760_MxC_double_10}
    \end{table}

% Please add the following required packages to your document preamble:
% \usepackage{multirow}
\begin{table}[h!]
\centering
    \begin{tabular}{|c|c|c|c|c|c|c|c|c|}
    \hline
    \multirow{2}{*}{} & \multicolumn{8}{c|}{\textbf{NUMERO DE ELEMENTOS DE LA MALLA}} \\ \cline{2-9} 
                      & $2^{8}$ & $2^{10}$& $2^{12}$& $2^{14}$& $2^{16}$& $2^{18}$& $2^{20}$& $2^{22}$\\ \hline
    \textbf{C}        &1.004  &1.003  &1.008  &1.000  &1.025  &1.023  &1.024  &1.026  \\ \hline
    \end{tabular}
    \caption{Speed Up realizado para el problema de la Construcción de Maxwell con la GPU NVIDIA Geforce GTX 760 en C comparando simple y doble precisión.}
    \label{tab:c_760_MxC_c_10}
    \end{table}

    


\begin{table}[h!]
    \begin{tabular}{|c|c|c|c|c|c|c|c|c|}
    \hline
                   & \multicolumn{8}{c|}{\textbf{NUMERO DE ELEMENTOS DE LA MALLA}} \\ \hline
    \textbf{BLOCK} & $2^8$ & $2^10$& $2^12$& $2^14$& $2^16$& $2^18$& $2^20$& $2^22$\\ \hline
    1              & 1.19  & 1.22  & 1.22  & 1.23  & 1.23  & 1.24  & 1.22  & 1.23  \\ \hline
    2              & 1.14  & 1.21  & 1.23  & 1.23  & 1.24  & 1.24  & 1.23  & 1.24  \\ \hline
    4              & 1.14  & 1.2   & 1.24  & 1.25  & 1.25  & 1.25  & 1.25  & 1.26  \\ \hline
    8              & 1.15  & 1.21  & 1.26  & 1.28  & 1.29  & 1.29  & 1.29  & 1.29  \\ \hline
    16             & 1.2   & 1.22  & 1.28  & 1.33  & 1.35  & 1.35  & 1.35  & 1.37  \\ \hline
    32             & 1.32  & 1.34  & 1.34  & 1.38  & 1.45  & 1.5   & 1.53  & 1.54  \\ \hline
    64             & 1.32  & 1.34  & 1.28  & 1.46  & 1.66  & 1.73  & 1.74  & 1.68  \\ \hline
    128            & 1.32  & 1.33  & 1.25  & 1.63  & 1.88  & 2.02  & 2.1   & 2.09  \\ \hline
    256            & 1.31  & 1.33  & 1.25  & 1.61  & 1.83  & 1.91  & 1.94  & 1.92  \\ \hline
    512            & 1.31  & 1.23  & 1.24  & 1.62  & 1.74  & 1.7   & 1.67  & 1.64  \\ \hline
    \end{tabular}
    \caption{Speed Up realizado para el problema de la Construcción de Maxwell con la GPU NVIDIA Geforce GTX 760 en CUDA comparando simple y doble precisión.}
    \label{tab:c_760_MxC_cuda_10}
    \end{table}


\newpage

\title{\textbf{GPU NVIDIA GEFORCE GTX 970}}

\label{apend_MxC_970}

\begin{table}[h!]
\centering
\begin{tabular}{|c|c|c|c|c|c|c|c|c|c|}
\hline
               & \multicolumn{9}{c|}{\textbf{NUMERO DE ELEMENTOS DE LA MALLA}}        \\ \hline
\textbf{BLOCK} & $2^{8}$& $2^{10}$& $2^{12}$& $2^{14}$& $2^{16}$& $2^{18}$& $2^{20}$& $2^{22}$& $2^{24}$\\ \hline
1              & 0.18 & 0.5   & 0.85  & 0.51  & 0.71  & 0.95  & 0.7   & 0.72  & 0.84  \\ \hline
2              & 0.24 & 0.71  & 1.43  & 0.81  & 1.3   & 1.96  & 1.74  & 1.47  & 1.28  \\ \hline
4              & 0.27 & 0.88  & 1.99  & 1.15  & 2.26  & 3.75  & 3.36  & 2.6   & 2.49  \\ \hline
8              & 0.27 & 0.7   & 0.7   & 1.44  & 3.97  & 6.8   & 6.39  & 5.13  & 5.32  \\ \hline
16             & 0.21 & 0.47  & 1.1   & 1.52  & 5.44  & 10.81 & 7.47  & 9.75  & 12.34 \\ \hline
32             & 0.22 & 0.78  & 2.02  & 1.54  & 7.1   & 14.5  & 9.96  & 13.68 & 23.39 \\ \hline
64             & 0.26 & 0.93  & 2.24  & 1.56  & 6.87  & 12.97 & 9.13  & 13.49 & 21.11 \\ \hline
128            & 0.27 & 0.93  & 2.16  & 1.79  & 6.84  & 8.32  & 9.07  & 11.81 & 17.23 \\ \hline
256            & 0.25 & 0.36  & 0.82  & 2.2   & 6.78  & 7.98  & 8.85  & 9.44  & 22.94 \\ \hline
512            & 0.15 & 0.37  & 0.95  & 1.74  & 6.2   & 7.67  & 8.74  & 9.21  & 21.68 \\ \hline
\end{tabular}
\caption{Speed Up realizado para el problema de la Construcción de Maxwell con la GPU NVIDIA Geforce GTX 970 en simple precisión}
\label{tab:s_970_MxC_simple_10}
\end{table}

\begin{table}[]
    \begin{tabular}{|c|c|c|c|c|c|c|c|c|}
    \hline
                   & \multicolumn{8}{c|}{\textbf{NUMERO DE ELEMENTOS DE LA MALLA}} \\ \hline
    \textbf{BLOCK} & $2^8$ & $2^10$& $2^12$& $2^14$& $2^16$& $2^18$& $2^20$& $2^22$\\ \hline
    1              & 0.16  & 0.41  & 0.67  & 0.7   & 0.49  & 0.57  & 0.54  & 0.53  \\ \hline
    2              & 0.23  & 0.61  & 1.21  & 1.26  & 0.91  & 1.38  & 1.03  & 1.09  \\ \hline
    4              & 0.25  & 0.86  & 1.64  & 2.16  & 1.67  & 2.84  & 2.24  & 2.01  \\ \hline
    8              & 0.25  & 0.66  & 0.68  & 3.15  & 2.65  & 4.95  & 4.84  & 3.95  \\ \hline
    16             & 0.19  & 0.39  & 0.84  & 2.34  & 3.66  & 8.14  & 8     & 7.15  \\ \hline
    32             & 0.2   & 0.7   & 1.54  & 1.48  & 4.29  & 10.06 & 9.38  & 10.96 \\ \hline
    64             & 0.25  & 0.89  & 1.98  & 1.48  & 3.93  & 8.02  & 7.44  & 8.29  \\ \hline
    128            & 0.25  & 0.89  & 1.92  & 1.48  & 3.92  & 7.98  & 7.41  & 8.39  \\ \hline
    256            & 0.24  & 0.34  & 0.83  & 1.48  & 3.91  & 7.93  & 7.38  & 6.34  \\ \hline
    512            & 0.13  & 0.3   & 1.15  & 1.47  & 4.54  & 7.98  & 6.16  & 6.33  \\ \hline
    \end{tabular}
    \caption{Speed Up realizado para el problema de la Construcción de Maxwell con la GPU NVIDIA Geforce GTX 970 en doble precisión}
    \label{tab:s_970_MxC_double_10}
    \end{table}

% Please add the following required packages to your document preamble:
% \usepackage{multirow}
\begin{table}[]
    \begin{tabular}{|c|c|c|c|c|c|c|c|c|}
    \hline
    \multirow{2}{*}{} & \multicolumn{8}{c|}{\textbf{NUMERO DE ELEMENTOS DE LA MALLA}} \\ \cline{2-9} 
                      & $2^8$ & $2^10$& $2^12$& $2^14$& $2^16$& $2^18$& $2^20$& $2^22$\\ \hline
    \textbf{C}        & 1.007 & 1.018 & 1.03  & 1.03  & 1.022 & 1.021 & 1.034 & 1.035 \\ \hline
    \end{tabular}
    \caption{Speed Up realizado para el problema de la Construcción de Maxwell con la GPU NVIDIA Geforce GTX 970 en C comparando simple y doble precisión.}
    \label{tab:c_970_MxC_c_10}
    \end{table}

\begin{table}[h!]
\centering
    \begin{tabular}{|c|c|c|c|c|c|c|c|c|}
    \hline
                   & \multicolumn{8}{c|}{\textbf{NUMERO DE ELEMENTOS DE LA MALLA}} \\ \hline
    \textbf{BLOCK} & $2^{8}$ & $2^{10}$& $2^{12}$& $2^{14}$& $2^{16}$& $2^{18}$& $2^{20}$& $2^{22}$\\ \hline
    1              & 1.13  & 1.22  & 1.3   & 0.76  & 1.48  & 1.7   & 1.36  & 1.42  \\ \hline
    2              & 1.06  & 1.19  & 1.21  & 0.67  & 1.46  & 1.45  & 1.75  & 1.41  \\ \hline
    4              & 1.07  & 1.04  & 1.25  & 0.55  & 1.38  & 1.35  & 1.55  & 1.34  \\ \hline
    8              & 1.06  & 1.07  & 1.07  & 0.47  & 1.53  & 1.4   & 1.37  & 1.34  \\ \hline
    16             & 1.1   & 1.21  & 1.34  & 0.67  & 1.52  & 1.36  & 0.96  & 1.41  \\ \hline
    32             & 1.1   & 1.13  & 1.35  & 1.08  & 1.69  & 1.47  & 1.1   & 1.29  \\ \hline
    64             & 1.07  & 1.07  & 1.16  & 1.08  & 1.79  & 1.65  & 1.27  & 1.68  \\ \hline
    128            & 1.05  & 1.06  & 1.16  & 1.24  & 1.79  & 1.07  & 1.26  & 1.46  \\ \hline
    256            & 1.06  & 1.07  & 1.02  & 1.53  & 1.77  & 1.03  & 1.24  & 1.54  \\ \hline
    512            & 1.16  & 1.23  & 0.85  & 1.21  & 1.39  & 0.98  & 1.47  & 1.51  \\ \hline
    \end{tabular}
    \caption{Speed Up realizado para el problema de la Construcción de Maxwell con la GPU NVIDIA Geforce GTX 970 en CUDA comparando simple y doble precisión.}
    \label{tab:c_970_MxC_cuda_10}
    \end{table}



%\include{apend_3}
%\chapter{}
\label{ap_python}

En el presente apéndice se muestran los resultados obtenidos de un Speed Up al \textit{kernel} que calcula la densidad $\rho$ de la Ec.(\ref{eq:rho}) en la
GPU NVIDIA GEFORCE GTX 760 y GPU NVIDIA GEFORCE GTX 970, en los códigos de \textsc{C}, \textsc{Cuda C} y \textsc{Python}.

\begin{table}[h!]
\centering
    \begin{tabular}{|c|c|c|c|c|c|c|c|c|c|}
    \hline
                   & \multicolumn{9}{c|}{\textbf{NUMERO DE ELEMENTOS DE LA MALLA}} \\ \hline
    \textbf{BLOCK} & $2^{8}$ & $2^{10}$& $2^{12}$& $2^{14}$& $2^{16}$& $2^{18}$& $2^{20}$& $2^{22}$& $2^{24}$\\ \hline
		1                               & 0.193   & 0.389    & 0.365    & 0.302    & 0.325    & 0.348    & 0.345    & 0.347    & 0.347 \\ \hline
		2                               & 0.212   & 0.588    & 0.603    & 0.537    & 0.638    & 0.664    & 0.662    & 0.667    & 0.667 \\ \hline
		4                               & 0.158   & 0.549    & 0.836    & 1.009    & 1.068    & 1.243    & 1.236    & 1.243    & 1.242 \\ \hline
		8                               & 0.17    & 0.842    & 1.121    & 1.519    & 1.882    & 2.133    & 2.14     & 2.153    & 2.157 \\ \hline
		16                              & 0.197   & 0.775    & 1.303    & 1.953    & 2.653    & 3.054    & 3.101    & 3.132    & 3.135 \\ \hline
		32                              & 0.236   & 0.727    & 1.25     & 1.755    & 2.707    & 2.91     & 3.159    & 3.191    & 3.193 \\ \hline
		64                              & 0.244   & 0.591    & 1.262    & 1.833    & 2.425    & 3.124    & 3.164    & 3.195    & 3.198 \\ \hline
		128                             & 0.146   & 0.631    & 1.374    & 1.837    & 2.389    & 3.1      & 3.159    & 3.196    & 3.194 \\ \hline
		256                             & 0.165   & 0.751    & 1.187    & 1.701    & 2.373    & 3.107    & 3.157    & 3.181    & 3.194 \\ \hline
		512                             & 0.166   & 0.607    & 1.187    & 1.656    & 2.362    & 2.881    & 3.152    & 3.189    & 3.193 \\ \hline

    \end{tabular}
    \caption{Speed Up realizado entre CUDA y C para la función de colisión con la GPU NVIDIA Geforce GTX 760 en simple precisión}
    \label{tab:s_cuda_760_test_simple_10}
    \end{table}

\begin{table}[h!]
	\centering
	\begin{tabular}{|c|c|c|c|c|c|c|c|c|c|}
		\hline
		& \multicolumn{9}{c|}{\textbf{NUMERO DE ELEMENTOS DE LA MALLA}} \\ \hline
		\textbf{BLOCK} & $2^{8}$ & $2^{10}$& $2^{12}$& $2^{14}$& $2^{16}$& $2^{18}$& $2^{20}$& $2^{22}$& $2^{24}$\\ \hline
		1                               & 0.052   & 0.132    & 0.116    & 0.106    & 0.112    & 0.128    & 0.13     & 0.131    & 0.131 \\ \hline
		2                               & 0.055   & 0.167    & 0.183    & 0.196    & 0.239    & 0.26     & 0.254    & 0.257    & 0.258 \\ \hline
		4                               & 0.058   & 0.196    & 0.263    & 0.338    & 0.457    & 0.5      & 0.498    & 0.502    & 0.502 \\ \hline
		8                               & 0.06    & 0.234    & 0.358    & 0.523    & 0.808    & 0.925    & 0.937    & 0.945    & 0.946 \\ \hline
		16                              & 0.058   & 0.244    & 0.419    & 0.69     & 1.263    & 1.545    & 1.596    & 1.621    & 1.616 \\ \hline
		32                              & 0.055   & 0.241    & 0.46     & 0.802    & 1.586    & 2.026    & 2.118    & 2.162    & 2.168 \\ \hline
		64                              & 0.056   & 0.24     & 0.459    & 0.894    & 1.834    & 2.422    & 2.559    & 2.614    & 2.606 \\ \hline
		128                             & 0.058   & 0.239    & 0.477    & 0.889    & 1.988    & 2.628    & 2.608    & 2.8      & 2.867 \\ \hline
		256                             & 0.056   & 0.241    & 0.468    & 0.841    & 1.984    & 2.632    & 2.798    & 2.865    & 2.847 \\ \hline
		512                             & 0.059   & 0.225    & 0.448    & 0.848    & 2        & 2.619    & 2.786    & 2.856    & 2.866 \\ \hline
	\end{tabular}
	\caption{Speed Up realizado entre PyCUDA y C para la función de colisión con la GPU NVIDIA Geforce GTX 760 en simple precisión}
	\label{tab:s_py_c_760_test_simple_10}
\end{table}
\begin{table}[h!]
\centering
    \begin{tabular}{|c|c|c|c|c|c|c|c|c|c|}
    \hline
                   & \multicolumn{9}{c|}{\textbf{NUMERO DE ELEMENTOS DE LA MALLA}} \\ \hline
    \textbf{BLOCK} & $2^{8}$ & $2^{10}$& $2^{12}$& $2^{14}$& $2^{16}$& $2^{18}$& $2^{20}$& $2^{22}$& $2^{24}$\\ \hline
		1                               & 3.685   & 2.938    & 3.134    & 2.844    & 2.895    & 2.724    & 2.658    & 2.654    & 2.647 \\ \hline
		2                               & 3.853   & 3.525    & 3.299    & 2.74     & 2.668    & 2.558    & 2.603    & 2.596    & 2.584 \\ \hline
		4                               & 2.717   & 2.809    & 3.182    & 2.989    & 2.336    & 2.487    & 2.483    & 2.477    & 2.477 \\ \hline
		8                               & 2.842   & 3.601    & 3.131    & 2.904    & 2.329    & 2.306    & 2.285    & 2.277    & 2.28  \\ \hline
		16                              & 3.384   & 3.181    & 3.11     & 2.83     & 2.101    & 1.977    & 1.942    & 1.932    & 1.939 \\ \hline
		32                              & 4.297   & 3.014    & 2.717    & 2.189    & 1.707    & 1.436    & 1.492    & 1.476    & 1.473 \\ \hline
		64                              & 4.347   & 2.461    & 2.751    & 2.049    & 1.322    & 1.29     & 1.236    & 1.222    & 1.227 \\ \hline
		128                             & 2.525   & 2.646    & 2.881    & 2.066    & 1.202    & 1.18     & 1.211    & 1.141    & 1.114 \\ \hline
		256                             & 2.931   & 3.113    & 2.537    & 2.021    & 1.196    & 1.18     & 1.128    & 1.11     & 1.122 \\ \hline
		512                             & 2.792   & 2.696    & 2.652    & 1.953    & 1.181    & 1.1      & 1.132    & 1.117    & 1.114 \\ \hline
    \end{tabular}
    \caption{Speed Up realizado entre PyCUDA y CUDA para la función de colisión con la GPU NVIDIA Geforce GTX 760 en simple precisión}
    \label{tab:s_py_760_test_simple_10}
    \end{table}

\begin{table}[h!]
\centering
    \begin{tabular}{|c|c|c|c|c|c|c|c|c|c|}
    \hline
                   & \multicolumn{9}{c|}{\textbf{NUMERO DE ELEMENTOS DE LA MALLA}} \\ \hline
    \textbf{BLOCK} & $2^{8}$ & $2^{10}$& $2^{12}$& $2^{14}$& $2^{16}$& $2^{18}$& $2^{20}$& $2^{22}$& $2^{24}$\\ \hline
		1                               & 0.101   & 0.465    & 0.773    & 1.1      & 1.146    & 1.202    & 1.284    & 1.303    & 1.254 \\ \hline
		2                               & 0.106   & 0.52     & 1.047    & 1.837    & 2.115    & 2.605    & 2.885    & 2.569    & 2.474 \\ \hline
		4                               & 0.107   & 0.542    & 1.163    & 2.234    & 2.054    & 3.557    & 4.742    & 4.119    & 4.063 \\ \hline
		8                               & 0.107   & 0.547    & 1.269    & 2.39     & 3.529    & 4.331    & 4.679    & 4.616    & 4.307 \\ \hline
		16                              & 0.108   & 0.561    & 1.25     & 2.454    & 3.582    & 4.675    & 5.087    & 5.008    & 4.504 \\ \hline
		32                              & 0.105   & 0.553    & 1.245    & 2.468    & 3.645    & 4.716    & 4.944    & 4.658    & 4.67  \\ \hline
		64                              & 0.105   & 0.519    & 1.25     & 2.33     & 3.417    & 4.784    & 4.97     & 5.111    & 4.815 \\ \hline
		128                             & 0.102   & 0.56     & 1.249    & 2.578    & 3.395    & 4.77     & 5.37     & 4.753    & 4.578 \\ \hline
		256                             & 0.098   & 0.542    & 1.258    & 2.213    & 3.301    & 4.635    & 5.402    & 4.807    & 4.685 \\ \hline
		512                             & 0.099   & 0.496    & 1.287    & 2.479    & 3.46     & 4.74     & 5.397    & 5.052    & 4.592 \\ \hline
    \end{tabular}
    \caption{Speed Up realizado entre CUDA y C para la función de colisión con la GPU NVIDIA Geforce GTX 970 en simple precisión}
    \label{tab:s_cuda_970_test_simple_10}
    \end{table}

\begin{table}[h!]
	\centering
	\begin{tabular}{|c|c|c|c|c|c|c|c|c|c|}
		\hline
		& \multicolumn{9}{c|}{\textbf{NUMERO DE ELEMENTOS DE LA MALLA}} \\ \hline
		\textbf{BLOCK} & $2^{8}$ & $2^{10}$& $2^{12}$& $2^{14}$& $2^{16}$& $2^{18}$& $2^{20}$& $2^{22}$& $2^{24}$\\ \hline
		1                               & 0.025   & 0.093    & 0.131    & 0.162    & 0.181    & 0.187    & 0.217    & 0.213    & 0.200 \\ \hline
		2                               & 0.031   & 0.137    & 0.217    & 0.299    & 0.388    & 0.396    & 0.438    & 0.435    & 0.421 \\ \hline
		4                               & 0.033   & 0.162    & 0.31     & 0.496    & 0.722    & 0.719    & 0.847    & 0.849    & 0.771 \\ \hline
		8                               & 0.033   & 0.173    & 0.374    & 0.734    & 1.166    & 1.273    & 1.404    & 1.460    & 1.327 \\ \hline
		16                              & 0.033   & 0.188    & 0.451    & 0.995    & 1.723    & 1.929    & 2.188    & 2.277    & 2.064 \\ \hline
		32                              & 0.032   & 0.184    & 0.474    & 1.244    & 2.266    & 2.432    & 3.081    & 3.133    & 2.831 \\ \hline
		64                              & 0.033   & 0.178    & 0.453    & 1.386    & 2.74     & 3.013    & 3.972    & 3.532    & 3.507 \\ \hline
		128                             & 0.034   & 0.177    & 0.494    & 1.334    & 2.934    & 3.913    & 4.598    & 4.059    & 4.025 \\ \hline
		256                             & 0.03    & 0.171    & 0.489    & 1.335    & 2.966    & 3.307    & 4.701    & 4.683    & 4.242 \\ \hline
		512                             & 0.031   & 0.174    & 0.46     & 1.347    & 2.926    & 3.177    & 4.495    & 4.376    & 4.226 \\ \hline
	\end{tabular}
	\caption{Speed Up realizado entre PyCUDA y C para la función de colisión con la GPU NVIDIA Geforce GTX 970 en simple precisión}
	\label{tab:s_py_c_970_test_simple_10}
\end{table}

\begin{table}[t!]
\centering
    \begin{tabular}{|c|c|c|c|c|c|c|c|c|c|}
    \hline
                   & \multicolumn{9}{c|}{\textbf{NUMERO DE ELEMENTOS DE LA MALLA}} \\ \hline
    \textbf{BLOCK} & $2^{8}$ & $2^{10}$& $2^{12}$& $2^{14}$& $2^{16}$& $2^{18}$& $2^{20}$& $2^{22}$& $2^{24}$\\ \hline
		1                               & 4.016   & 5.016    & 5.886    & 6.799    & 6.342    & 6.412    & 5.912    & 6.12     & 6.266 \\ \hline
		2                               & 3.405   & 3.805    & 4.814    & 6.14     & 5.456    & 6.586    & 6.585    & 5.907    & 5.879 \\ \hline
		4                               & 3.265   & 3.353    & 3.752    & 4.503    & 2.846    & 4.948    & 5.597    & 4.85     & 5.271 \\ \hline
		8                               & 3.273   & 3.166    & 3.393    & 3.257    & 3.026    & 3.404    & 3.333    & 3.161    & 3.245 \\ \hline
		16                              & 3.253   & 2.983    & 2.77     & 2.467    & 2.079    & 2.423    & 2.325    & 2.2      & 2.182 \\ \hline
		32                              & 3.284   & 3.016    & 2.63     & 1.984    & 1.608    & 1.939    & 1.605    & 1.487    & 1.65  \\ \hline
		64                              & 3.199   & 2.911    & 2.756    & 1.68     & 1.247    & 1.588    & 1.251    & 1.447    & 1.373 \\ \hline
		128                             & 3.042   & 3.167    & 2.525    & 1.933    & 1.157    & 1.219    & 1.168    & 1.171    & 1.137 \\ \hline
		256                             & 3.237   & 3.165    & 2.574    & 1.658    & 1.113    & 1.402    & 1.149    & 1.026    & 1.104 \\ \hline
		512                             & 3.179   & 2.851    & 2.796    & 1.841    & 1.182    & 1.492    & 1.201    & 1.154    & 1.087 \\ \hline
    \end{tabular}
    \caption{Speed Up realizado entre PyCUDA y CUDA para la función de colisión con la GPU NVIDIA Geforce GTX 970 en simple precisión}
    \label{tab:s_py_970_test_simple_10}
    \end{table}

\begin{seccion}{Presentaciones en congresos asociadas a este proyecto integrador}
	
	\begin{enumerate}
		
		\item Argañaras P.E., Fogliatto E.O., Coronel T. Junio 2020. Modelo lattice Boltzmann para flujo multifásico con transferencia de calor en GPU. XXII Workshop de Investigadores en Ciencias de la Computación (WICC 2020, El Calafate, Argentina)
		
	\end{enumerate}
	
\end{seccion}


\begin{biblio}
\bibliography{biblio}
\end{biblio}


\begin{postliminary}

\listoffigures                  %Figuras

\listoftables                   %Tablas






\begin{seccion}{Agradecimientos}
\begin{footnotesize}	
	
A lo largo de todo el tiempo que transité el camino de este sueño, ahora cumplido, se me presentaron personas, momentos y lugares. El conjunto de cada uno de esos instantes me marcó y hacen que hoy sea quien soy.

Primero quiero agradecer a Ezequiel, mi director, que me guió, motivó, tuvo paciencia, motivó y creyó en mí para terminar este PI. En segundo lugar agradezco a Pablo, mi codirector, por haberme acompañado en el proceso. Sin ellos este trabajo no hubise sido posible, y más aún por la forma de trabajo que tuvimos que adoptar estos últimos meses.

Gracias al IB por permitirme estudiar en el lugar que siempre soñe de chico, a los profesores que a lo largo de los tres años que estuve me marcaron en la forma de aprender y de disfrutar Bariloche. Especialmente agradezco a Beto, Marcos, Tobias, Franco, Enzo, Fede, Horacio y Nancy. Después de muchas vivencias compartida a lo largo de los tres años en el IB les agradezco a Tomasito, Tincho, Santi, Lucas, Lucca y Belén, ya que los pabellones se hicieron más hogar gracias a ustedes. 

Muchas gracias Norbert por el apoyo y haber estado siempre, ya sea con mates, charlas o salidas, a partir de esos viajes hasta ahora. También te quiero agradecer Chipo por tus reflexiones, senderos y comidas compartidas. A Juli, Maru, Cristian y Fer que me brindaron todo su amor, cariño y fuerzas para que pueda seguir semana a semana. A Mariano, Ernesto, Roberto, Santiago y a toda la PM550 por darme otras herramientas para pensar y progresar. Fueron mi familia estos tres años.

Agradezco a mi mamá Norma por haberme brindado todo el amor, cariño, aliento, fuerzas, herramientas y el ejemplo, porque  hoy soy quien soy gracias a ella. A mi papá Carlos que siempre me motivó a seguir adelante e ir detrás de mis sueños. A mis hermanitos Sonia y Leandro, ya que sin sus consejos, charlas, llamadas y peleas, la vida sería muy aburrida, gracias porque sin importar la distancia me hacen sentir su calidez y apoyo incondicional para continuar. También agradezco a mis padrinos, tíos y primos por darme ese empujoncito que necite cada vez que volvia al hogar.

A mis abuelos Kiki, Chichi, y a . que son mis tres puntitos en el cielo, al día de hoy me acompañan y siguen marcando en la vida.

Un agradecimiento muy particular para Tito, Iván, Adrián, Ramiro y Gabriel  que siempre me esperaban con los brazos abiertos, y me hacían sentir que todavía estabamos con el overol puesto. A mis profes del IT Blasco, Dode y Ruiz que me incentivaron a ir más allá de la conformidad. A María, Nahuel, Mapi, Ferchu y toda la comunidad de LC. A Gonzalo y Juan Cruz que me permitieron ver otro lado de la vida en éste último semestre.

A Flor, Elisa y Julieta por transmitirme esa felicidad  y forma que poseen de  ver la vida. A Pablo y Julito que me hicieron llevadero los días en los anfiteatros de la Quinta,  a Matías, Flavio, Joaquín, Sabri, Leo, Blas, Franco por compartir esa etapa.

Muchas gracias a todos aquellos que de una forma u otra aparecieron en mi camino, porque durante el transcurso de este sueño, aprendí a ver lo maravillosa que es la vida.

\end{footnotesize}


\end{seccion}



\end{postliminary}



\end{document}
