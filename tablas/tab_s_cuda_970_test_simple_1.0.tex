\begin{table}[h!]
\centering
    \begin{tabular}{|c|c|c|c|c|c|c|c|c|c|}
    \hline
                   & \multicolumn{9}{c|}{\textbf{NUMERO DE ELEMENTOS DE LA MALLA}} \\ \hline
    \textbf{BLOCK} & $2^{8}$ & $2^{10}$& $2^{12}$& $2^{14}$& $2^{16}$& $2^{18}$& $2^{20}$& $2^{22}$& $2^{24}$\\ \hline
		1                               & 0.101   & 0.465    & 0.773    & 1.1      & 1.146    & 1.202    & 1.284    & 1.303    & 1.254 \\ \hline
		2                               & 0.106   & 0.52     & 1.047    & 1.837    & 2.115    & 2.605    & 2.885    & 2.569    & 2.474 \\ \hline
		4                               & 0.107   & 0.542    & 1.163    & 2.234    & 2.054    & 3.557    & 4.742    & 4.119    & 4.063 \\ \hline
		8                               & 0.107   & 0.547    & 1.269    & 2.39     & 3.529    & 4.331    & 4.679    & 4.616    & 4.307 \\ \hline
		16                              & 0.108   & 0.561    & 1.25     & 2.454    & 3.582    & 4.675    & 5.087    & 5.008    & 4.504 \\ \hline
		32                              & 0.105   & 0.553    & 1.245    & 2.468    & 3.645    & 4.716    & 4.944    & 4.658    & 4.67  \\ \hline
		64                              & 0.105   & 0.519    & 1.25     & 2.33     & 3.417    & 4.784    & 4.97     & 5.111    & 4.815 \\ \hline
		128                             & 0.102   & 0.56     & 1.249    & 2.578    & 3.395    & 4.77     & 5.37     & 4.753    & 4.578 \\ \hline
		256                             & 0.098   & 0.542    & 1.258    & 2.213    & 3.301    & 4.635    & 5.402    & 4.807    & 4.685 \\ \hline
		512                             & 0.099   & 0.496    & 1.287    & 2.479    & 3.46     & 4.74     & 5.397    & 5.052    & 4.592 \\ \hline
    \end{tabular}
    \caption{Speed Up realizado entre CUDA y C para la función de colisión con la GPU NVIDIA Geforce GTX 970 en simple precisión}
    \label{tab:s_cuda_970_test_simple_10}
    \end{table}
