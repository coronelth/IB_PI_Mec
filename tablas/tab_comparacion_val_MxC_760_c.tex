% Please add the following required packages to your document preamble:
% \usepackage{multirow}
\begin{table}[h!]
\centering
%\resizebox{17cm}{!}{
	\begin{tabular}{|c|c|c|c|c|c|c|}
	\hline
	& \multicolumn{3}{c|}{${\rho_{r}}_{\>gaseoso}$}      & \multicolumn{3}{c|}{${\rho_{r}}_{\>líquido}$} \\ \hline
	$\mathbf{T_r}$    & \textbf{Analítico}      & \textbf{Simple}       & \textbf{Doble}     & \textbf{Analítico}      & \textbf{Simple}     & \textbf{Doble}   \\ \hline
	0.600 & 0.0599097 & 0.0653772 & 0.0653724 & 1.32424  & 1.31956 & 1.31975 \\ \hline
	0.625 & 0.0733723 & 0.0848112 & 0.0847656 & 1.46149  & 1.46239 & 1.4624  \\ \hline
	0.650 & 0.0897449 & 0.0951336 & 0.095124  & 1.56762  & 1.56858 & 1.56859 \\ \hline
	0.675 & 0.107606  & 0.113153  & 0.113147  & 1.56762  & 1.65819 & 1.65821 \\ \hline
	0.700 & 0.128332  & 0.13353   & 0.133511  & 1.6572   & 1.73703 & 1.73704 \\ \hline
	0.725 & 0.150966  & 0.15182   & 0.151811  & 1.73595  & 1.80812 & 1.80813 \\ \hline
	0.750 & 0.177353  & 0.177323  & 0.177319  & 1.80706  & 1.87326 & 1.87328 \\ \hline
	0.775 & 0.206739  & 0.206086  & 0.206075  & 1.87233  & 1.93364 & 1.93364 \\ \hline
	0.800 & 0.23938   & 0.244229  & 0.244223  & 1.93243  & 1.98754 & 1.98759 \\ \hline
	0.825 & 0.277393  & 0.281299  & 0.281284  & 1.98899  & 2.04083 & 2.04085 \\ \hline
	0.850 & 0.319677  & 0.323471  & 0.323456  & 2.0423   & 2.09124 & 2.09124 \\ \hline
	0.875 & 0.368925  & 0.371915  & 0.371891  & 2.09234  & 2.14114 & 2.14117 \\ \hline
	0.900 & 0.425549  & 0.428416  & 0.428335  & 2.14012  & 2.18665 & 2.18665 \\ \hline
	0.925 & 0.493618  & 0.495973  & 0.495947  & 2.18563  & 2.23017 & 2.23019 \\ \hline
	0.950 & 0.493618  & 0.580402  & 0.580368  & 2.22933  & 2.27407 & 2.27411 \\ \hline
	0.975 & 0.578746  & 0.690126  & 0.690247  & 2.27117  & 2.312   & 2.312   \\ \hline
	\textbf{Distancia} & -         & 2.05233   & 2.05226   & -        & 0.14234 & 0.14241 \\ \hline
\end{tabular}%}
    \caption{Comparación de las precisiones con respecto a cuánto se acercan al valor analitico, tanto para doble, como simple precision, la norma utilizada para medir la distancia de los vectores es la norma  euclídea. Para el problema de la Construcción de Maxwell con la GPU NVIDIA Geforce GTX 760.}
    \label{tab:comp_MxC_precisiones_10}
    \end{table}

