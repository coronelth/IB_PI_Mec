\begin{table}[h!]
\centering
    \begin{tabular}{|c|c|c|c|c|c|c|c|c|c|}
    \hline
                   & \multicolumn{9}{c|}{\textbf{NUMERO DE ELEMENTOS DE LA MALLA}} \\ \hline
    \textbf{BLOCK} & $2^{8}$ & $2^{10}$& $2^{12}$& $2^{14}$& $2^{16}$& $2^{18}$& $2^{20}$& $2^{22}$& $2^{24}$\\ \hline
		1                               & 0.193   & 0.389    & 0.365    & 0.302    & 0.325    & 0.348    & 0.345    & 0.347    & 0.347 \\ \hline
		2                               & 0.212   & 0.588    & 0.603    & 0.537    & 0.638    & 0.664    & 0.662    & 0.667    & 0.667 \\ \hline
		4                               & 0.158   & 0.549    & 0.836    & 1.009    & 1.068    & 1.243    & 1.236    & 1.243    & 1.242 \\ \hline
		8                               & 0.17    & 0.842    & 1.121    & 1.519    & 1.882    & 2.133    & 2.14     & 2.153    & 2.157 \\ \hline
		16                              & 0.197   & 0.775    & 1.303    & 1.953    & 2.653    & 3.054    & 3.101    & 3.132    & 3.135 \\ \hline
		32                              & 0.236   & 0.727    & 1.25     & 1.755    & 2.707    & 2.91     & 3.159    & 3.191    & 3.193 \\ \hline
		64                              & 0.244   & 0.591    & 1.262    & 1.833    & 2.425    & 3.124    & 3.164    & 3.195    & 3.198 \\ \hline
		128                             & 0.146   & 0.631    & 1.374    & 1.837    & 2.389    & 3.1      & 3.159    & 3.196    & 3.194 \\ \hline
		256                             & 0.165   & 0.751    & 1.187    & 1.701    & 2.373    & 3.107    & 3.157    & 3.181    & 3.194 \\ \hline
		512                             & 0.166   & 0.607    & 1.187    & 1.656    & 2.362    & 2.881    & 3.152    & 3.189    & 3.193 \\ \hline

    \end{tabular}
    \caption{Speed Up realizado entre CUDA y C para la función de colisión con la GPU NVIDIA Geforce GTX 760 en simple precisión}
    \label{tab:s_cuda_760_test_simple_10}
    \end{table}
