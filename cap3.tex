\chapter{Modelo de lattice-Boltzmann}
\graphicspath{{figs/cap3/}}
\label{cap3}

\section{Introducción}

En éste capítulo se presentara cuál es el modelo de LB que se utilizará para resolver
los problemas de transferencia de calor con flujos multifásicos y cambio de fase.

Las ecuaciones de la mecánica de los fluidos son generalmente dificiles de resolver. 
Las soluciones analíticas de los problemas que pueden ser resueltos son escasos como el caso de flujos \textit{Couette} o \textit{Poisueuille}.
Problemas que contengan una geometría más compleja u otras situaciones de contorno; poseen 
una gran dificultad en resolver las ecuaciones de la mecánica de fluidos, si es que el problema tiene solución.
Debido a ello dichos tipos de problema son resueltos numéricamente. \cite{kruger2017lattice}(sección 3.1) 
Es de importancia el desarrollo de métodos numéricos que resuelvan los problemas 
de forma paralela para así reducir el tiempo de cálculo.

Debido a que los problemas que se plantean resolver en el presente trabajo es de transferencia de calor 
en flujos multifásicos con cambio de fase, la escala de del fluido que se decide adoptar es la mesoscópica.
Para dicha escala se opta por resolver numéricamente mediante LBM. 

La obtención de la ecuación de lattice-Boltzmann (LBE) a través de la ecuación de Boltzmann se encuentra descripta
en las secciones (3.4) y (3.5) de  \cite{kruger2017lattice}. Primero se explica la discretización en el espacio de velocidades, en el segundo la 
discretización en el espacio físico, tiempo e intrgración de las caracteísticas.

%los conocimienros básicos del modelo de lattice-Boltzmann, como se utiliza
%en la simulacion de fluidos, y a su vez también como es la implementacion del codigo. 
%La demostración de la obtención de la ecuación de lattice-Boltzmann, a travez de la ecuación de contínuidad
%de Boltzmann siendo discretizada en el espacio de velocidades a través de la expansión de las series de
%polinomios de Hermite, siendo discretizada en el espacio físico y a través del tiempo característico.
%Se encuentra en la sección 3.1 de \cite{Krüger:2017qoa}.

Para entender el modelo de lattice-Boltzmann, hay que saber que el mismo se deriva de discretizar la ecuación de Boltzmann.
La ecuación de Boltzmann es discretizada en el espacio de velocidades, espacio físico y espacio temporal.

Un LBM es el \textit{DdQq} en donde D significa dimensión y \textit{d} es la dimensión que posee el modelo; 
Q significa la cantidad de vecinos de un nodo que interactúan con él, siendo  \textit{q} la cantidad de vecinos.
En el presente trabajo el LBM es $D2Q9$. La figura \ref{fig:D2Q9esquema} muestra como es la interacción de un nodo con sus vecinos cercanos.

%% Insertar la figura con esquema de D2Q9 con los vecinos


El problema físico a resolver cuenta con una región, la cuál se le realizará un mallado para así discretizar el espacio. 
El nodo i - ésimo de la malla posee las coordenadas ${\bar{X}}_{i} = (x,y,z)$, a su vez densidad $\rho_{i}$ y temperatura $T_{i}$.
La velocidad en el nodo tiene las componentes ${\bar{U}}_{i} = ({U}_{ix},{U}_{iy},{U}_{iz})$.
A su vez encontramos el espacio de velocidades, el cuál indica que cada nodo posee una velocidad en el cual se propagan las propiedades en la grilla.
Dicha velocidad de grilla $\mathbf{e}$ posee $\alpha$ componentes donde $\alpha = q $. En la figura \ref{fig:D2Q9esquema} se muestra un esquema de
las velocidades de grilla $\alpha$ del nodo i - ésimo, y en la Ec (\ref{eq:velgrilla}) se encuentra el valor adoptado.


\begin{equation}
    {\mathbf{e}}_{i} =  
    \left( \begin{array}{c} 
                e_{i0} \\ e_{i1}\\ e_{i2}\\ e_{i3}\\ e_{i4}\\ e_{i5}\\
                e_{i6}\\ e_{i7}\\ e_{i8}\\
            \end{array}
    \right) =
    \left( \begin{array}{c} 
        (0,0,0) \\ (1,0,0) \\ (0,1,0) \\(-1,0,0) \\ (0,-1,0) \\ (1,1,0) \\
        (-1,1,0) \\ (-1,-1,0) \\ (1,-1,0)\\ 
    \end{array}
    \right) 
    \label{eq:velgrilla}
\end{equation}

A su vez cada uno de los nodos de la grilla posee una función de distribución
$f_{i}$ también con $\alpha$ componentes. Y con dicha función se pueden obtener las 
variables macroscópicas del problema.

Para el desarrollo de la solución de los problemas que se describieron en \ref{cap2}, 
se utiliza el modelo pseudopotencial de dos ecuaciones con operador MRT, siendo desarrollado en \cite{fogliatto2018modelado} 
y ...(citar el pdf que posee la ecuacion de energía) (no encontrado en google scholar)


\section{Modelo pseudopotencial de dos ecuaciones con operador MRT}

\subsection{Ecuación hidrodinámica}


La resolución de las ecuaciones hidrodinámicas puede analizarse mediante la evolución de
una función de distribución dada por (Li et al., 2013):on}

\begin{equation}
    \mathbf{f^*}(\mathbf{x},t) = \mathbf{M}^{-1} \left[ \mathbf{m} - \mathbf{\Lambda}(\mathbf{m} - \mathbf{m}^{(eq)}) + \delta_{t} \left( I - 0,5 \Lambda \right) \bar{S}  \right]_{(\mathbf{x},t)} 
    \label{eq:fieldmom}
\end{equation}

donde $\textit{f}_{\alpha}$ es la distribución de densidad en el espacio de poblaciones, t el tiempo, x la posición
espacial, \textit{\textbf{e}} las velocidades discretas a lo largo de la direcciones $\alpha$ y $\delta_{t}$ 
el paso de tiempo. En este caso, la notación usada en la Ec. (\ref{eq:fieldmom}) implica que la compontente $\alpha-ésima$ del miembro izquierdo
está dada por $f_{\alpha}(x + e_{\alpha} \delta_{t}  , t + \delta_{t} )$ . El miembro derecho de la Ec. (\ref{eq:fieldmom}) corresponde a la etapa de
post-colisión definida en el espacio de momentos, donde M es una matriz de transformación
ortogonal, $m = M \cdot f$ , $m_{eq} = M \cdot f_{eq}$ , I el tensor identidad y $\bar{S} = M S$ el término de fuente.
Para una grilla D2Q9, $ \Lambda$ es una matriz diagonal:
    
\begin{equation}
    \mathbf{\Lambda}  = diag ( {\tau_{\rho }}^{-1},{\tau_{e}}^{-1},{\tau_{\zeta }}^{-1},{\tau_{j}}^{-1},{\tau_{q}}^{-1},{\tau_{j}}^{-1},{\tau_{q}}^{-1},{\tau_{\nu }}^{-1},{\tau_{\nu}}^{-1},) 
\end{equation}
        mientras que la distribución de equilibrio está dada por:
\begin{equation}
    m_{eq} =  \rho ( 1, −2 + 3 {|u|}^{2} , 1 - 3{|u|}^{2} , u_{x} , - u_{x} , u_{y} , - u_{y} , {u_{x}}^{2} - {u_{y}}^{2} , u_{x} u_{y} ) 
\end{equation}


donde la densidad y velocidad macroscópica se obtienen mediante:

\begin{equation}
        \rho = \sum_{\alpha} f_{\alpha}
\end{equation}

\begin{equation}
    \rho \mathbf{u} = \sum_{\alpha} {\mathbf{e}}_{\alpha} f_{\alpha} + 0,5 {\delta}{t} \mathbf{F}
\end{equation}

En este caso, $ {\mathbf{F}} = (F_{x} , F_{y} ) = {\mathbf{F}}_{b} + {\mathbf{F}}_{int} $ es la fuerza total, ${\mathbf{F}}_{b}$ la fuerza volumétrica y ${\mathbf{F}}_{int}$
representa la fuerza de interacción que actúa sobre el sistema a través de un potencial $\psi(x)$:
    
\begin{equation}
    {\mathbf{F}}_{int} = - G \psi(\mathbf{x}) \sum_{\alpha=1}^{N} w({|{\mathbf{e}}_{\alpha}|}^{2}) \psi (\mathbf{x} + {\mathbf{e}}_{\alpha} \delta_{t}) {\mathbf{e}}_{\alpha} 
    \label{eq:fint}
\end{equation}

En la Ec. \ref{eq:fint}, G corresponde a la intensidad de interacción, $w({|{\mathbf{e}}_{\alpha}|}^{2})$ son los pesos corres-
pondientes a una grilla D2Q9 y $\psi$ está dado por:

\begin{equation} 
    \psi(\rho) = \sqrt{\frac{2 (p_{EOS} - \rho {c_{s}}^{2})}{G {c}^{2}}}
\end{equation}

En el presente trabajo se adopta una EOS de VdW:

\begin{equation}
    \rho_{EOS} = \frac{\rho r t}{1- \rho B} - A {\rho}^{2}
\end{equation}

donde \textit{a} y \textit{b} son parámetros que determinan los valores críticos de temperatura, presión y den-
sidad, y fueron fijados en $\textit{a} = 1$ y $\textit{b} = 4$. Finalmente, la fuerza de interacción se incorpora en la
etapa de colisión mediante un término de fuente apropiado:

\begin{equation}
    \bar{S} = 
    \left[ \begin{array}{c} 
        0\\
        6 \mathbf{u}\cdot \mathbf{F} + \frac{12 \sigma {|{\mathbf{F}_{int}|}}^{2} }{{\psi}^{2} \delta_{t} (\tau_{e} - 0,5)}\\
        6 \mathbf{u}\cdot \mathbf{F} - \frac{12 \sigma {|{\mathbf{F}_{int}|}}^{2} }{{\psi}^{2} \delta_{t} (\tau_{\zeta } - 0,5)}\\
        F_{x}\\
        -F_{x}\\
        F_{y}\\
        -F_{y}\\
        2(u_{x} F_{x} - u_{y} F_{y} )\\
        (u_{x} F_{x} + u_{y} F_{y} )\\              
    \end{array}
    \right]    
\end{equation}

donde $\sigma$ es un parámetro libre que es utilizado para ajustar el problema de inconsistencia termodinámica,
es decir, la diferencia entre las densidades de cada fase obtenidas en la simulación y aquellas determinadas por la EOS correspondiente.

\subsection{Ecuación energía}

Se debe acoplar una segunda ecuación LB para poder incorporar transferencia de calor al modelo de Li et al. (2013).
En particular, puede adoptarse una segunda distribución de poblaciones \textit{g} bajo un operador de colisión MRT:

\begin{equation}
    \mathbf{g^*}(\mathbf{x} + \mathbf{e} \delta_{t} ,t + \delta_{t}) = \mathbf{M}^{-1} \left[ \mathbf{n} - \mathbf{Q}(\mathbf{n} - \mathbf{n}^{(eq)}) + \delta_{t} \left( I - 0,5 Q \right) \hat{\Gamma}  \right]_{(\mathbf{x},t)}
    \label{eq:fieldenergy}
\end{equation}

donde $n = M g$ y $\hat{\Gamma}$es una fuente en el espacio de momentos. La temperatura macroscópica \textit{T}
puede recuperarse mediante:

\begin{equation}
    T = \sum_{\alpha} g_{\alpha} + \frac{1}{2} \delta_{t} {\hat{\Gamma}}_{0}
\end{equation}

La matriz de coeficientes de relajación \textbf{Q} está compuesta por una parte diagonal

\begin{equation}
    \textit{diag} (Q) = {( q_{0} , q_{1} , q_{2} , q_{3} , q_{4} , q_{5} , q_{6} , q_{7} , q_{8} )}^{T}
\end{equation}

pero, a diferencia de $\lambda$ , presenta elementos extra diagonales no nulos dados por:

\begin{equation}
    Q_{3,4} = q_{4} \left( \frac{q_{3}}{2} - 1 \right)
\end{equation}

\begin{equation}
    Q_{5,6} = q_{6} \left( \frac{q_{5}}{2} - 1 \right)
\end{equation}

Si se define una distribución de equilibrio \textit{n} como:

\begin{equation}
    {\mathbf{n}}_{eq} = T { \left( 1, \alpha_{1}, \alpha_{2}, u_{x}, -u_{x}, u_{y}, -u_{y}, 0, 0 \right) }^{T}
\end{equation}

y el término fuente mediante:

\begin{equation}
    \hat{\Gamma} = {( s, 0, 0, 0, 0, 0, 0, 0, 0 )}^{T}
\end{equation}

con 

\begin{equation}
    s = \frac{\chi}{\rho} \bigtriangledown T \cdot \bigtriangledown \rho + T \left[ 1 - \frac{1}{\rho c_{\nu}} {\left( \frac{\delta p_{EOS}}{\delta T} \right)}_{\rho} \right] \bigtriangledown \cdot \mathbf{u}
\end{equation}

entonces la Ec. \ref{eq:fieldenergy} puede recuperar adecuadamente la ecuación macroscópica derivada por
Márkus y Házi (2011):

\begin{equation}
    \delta_{t} T + \bigtriangledown \cdot ( \mathbf{u} T ) = \chi {\bigtriangledown }^{2} T + s
\end{equation}

Por simplicidad, en este trabajo se considera una difusifidad térmica $\chi$ constante, la cual
queda determinada mediante los factores de relajación de \textbf{Q} y los parámetros libres de ${\textbf{n}}_{eq}$ :

\begin{equation}
    \chi = \delta_{t} \left( \frac{1}{q_{3}} - \frac{1}{2} \right) \left( \frac{ 4 + 3 \alpha_{1} + 2 \alpha_{2}}{6} \right)
\end{equation}




%%% Local Variables: 
%%% mode: latex
%%% TeX-master: "template"
%%% End: 