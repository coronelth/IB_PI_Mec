\chapter{Descripción de los problemas a resolver y su validación}
\graphicspath{{figs/cap4/}}
\label{cap4}

El primero de los problemas que se utiliza para validar un método numérico de LB es aquel que resuelve de manera correccta la construcción de Maxwell.

En el caso de un dominio en el cuál la temperatura del fluido se mantenga constante y sea uniforme, al haber regiones en el dominio en donde se encuentren distintas fases con sus respectivas densidades. La construcción de Maxwell predice que al evolucionaren el tiempo desde dicho estado inicial, se concluirá con la separación de las fases de dicho fluido.



Descripción de los problemas en fluidos con transferencia de calor

Si se considera una cavidad sin gravedad y con temperatura uniforme, el modelo LB produce
una separación del fluido en regiones de diferente densidad. Esto permite determinar la densidad
de cada fase a ambos lados de la interface y así comparar ambos valores con los establecidos
por la construcción de Maxwell. En la Fig. 1 se muestra la curva de coexistencia de fases
calculada con diferentes valores de $\sigma$, es decir, los valores de concentración molar reducida
($c = \rho/\rho_c $) sobre la interface para distintos valores de temperatura reducida T r = T /T c . Puede
verse que el uso de $\sigma$ = 1/8 permite reproducir satisfactoriamente los valores teóricos dados
por la construcción de Maxwell en un amplio rango de temperaturas.




%%% Local Variables: 
%%% mode: latex
%%% TeX-master: "template"
%%% End: 
