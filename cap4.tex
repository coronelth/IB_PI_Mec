\chapter{Descripción de los problemas en fluidos con transferencia de calor }
\graphicspath{{figs/cap4/}}
\label{cap4}

En el presente capíulo se realiza la descripción de los problemas con transferencia de calor para fluidos multifásicos con cambio de fase llevados a cabo para validar los códigos numéricos desarrollados, siendo éstos la \textit{Construcción de Maxwell}, la \textit{Estratificación de un fluido Van Der Waals con temperatura no uniforme} y la \textit{Generación de burbujas en una superficie horizontal calefaccionada}.

\section{Construcción de Maxwell}

Como se desarrollo en el Cap. \ref{cap2} la Ec. (\ref{eq:VdW_P}) es una EOS que modela el comportamiento de un gas real.

\begin{align*}
	p = \frac{R T}{V_m - B} - A {\left(\frac{1}{V_m}\right)}^2
\end{align*}

La Ec. (\ref{eq:VdW_P}) puede ser representada gráficamente en un diagrama $P - V_m$. \\ La Figura \ref{fig:P_V_CO2} muestra el diagrama mencionado para el dióxido de carbono ($CO_2$) a distintas temperaturas: 373 (K), 304 (K) y 270 (K). Para $T = 270 \> K$ se vislumbra que en $p = 44,08 \> atm$ la gráfica se intersecta en tres valores de $V_m$, siendo dos de ellos estables; por lo que se observa que hay dos volúmenes molares de coexistencia, indicando las fases líquido y gaseosa.

La construcción de Maxwell, también llamada regla de igualdad de áreas, indicada en Ec. (\ref{eq:maxwell_Construction}), es un procedimiento analítico para encontrar las densidades de coexistencia del líquido y gas. Donde $P$ es la presión de la EOS y $p_0$ es una presión constante. Al realizar la integral propuesta surge que las áreas \textbf{A1} y \textbf{A2} de la Figura \ref{fig:P_V_CO2} deben ser iguales.

\begin{equation}
\int_{V_{m,l}}^{V_{m,g}} P d V_m = p_0 (V_{m,l} -  V_{m,g})
\label{eq:maxwell_Construction}
\end{equation}

\begin{figure}[h!]
	\centering
	\includegraphics[width=.8\textwidth]{figs/cap4/Diagrama_P_V_del_CO2_Multiphase_LBM}
	\caption{Diagrama $P - V_m$ de la EOS de VdW del $CO_2$ con las constantes $a = 3,592$ y $b = 0,04267$, representando par a $T = 270 \> K$ los volúmenes molares del líquido y gas. \cite{huang2015multiphase}}
	\label{fig:P_V_CO2}	
\end{figure}


La Ec. (\ref{eq:VdW_P}) se puede re-estructurar como Ec. (\ref{eq:rho_eos}) puesto que $\rho = \frac{1}{v}$, siendo $v$ el volúmen másico y relacionando $V_m$ con $v$ según cada fluido. Donde para un dado valor de temperatura tendremos la coexistencia de fases con su densidad $\rho_l$ para la fase líquida y $\rho_g$ para la gaseosa.

\begin{equation*}
p_{EOS} = \frac{\rho R T}{1- \rho b} - a {\rho}^{2} \nonumber 
%\label{eq:VdW_rho}
\end{equation*}

Para un dado valor de temperatura, llamado temperatura crítica (\textit{$T_c$}) comienzan a coexistir las dos fases. En el ejemplo mostrado de la Figura \ref{fig:P_V_CO2} $T_c = 304 \> K$. Analíticamente $T_c$ surge de aplicar el criterio de la primera y segunda derivada a la Ec.(\ref{eq:rho}) como se indica en Ec.(\ref{eq:criterio_1_2_deriv}) y se deben conocer los parámetros \textit{a} y \textit{b}.

\begin{equation}
	\frac{\partial\> p}{\partial\> V_{m}} = 0 \qquad \qquad \frac{\partial^{2} \> p}{\partial\> {V_{m}}^{2}} = 0
	\label{eq:criterio_1_2_deriv}
\end{equation}

Realizando adecuadamente la adimensionalización  de la Ec.(\ref{eq:rho_eos}) se puede graficar una curva de coexistencia $T_r - \rho_r$  como se observa en la Figura \ref{fig:T_r_rho_r_analitico}, siendo $T_r = \frac{T}{T_c}$ y $\rho_r = \frac{\rho}{\rho_c}$.

\begin{figure}[h!]
	\centering
	\includegraphics[width=.8\textwidth]{figs/cap4/Diagrama_T_r_vs_rho_r_analitico}
	\caption{Curva de coexistencia de fases para un fluido de VdW con los parámetros $a = 0,5 $ y $b = 4,0 $.}
	\label{fig:T_r_rho_r_analitico}	
\end{figure}

\newpage
Es de importancia destacar que las curvas de coexistencia dependen de la EOS que se esté utilizando para describir el comportamiento del fluido, en éste caso de estudio, la EOS es de VdW y posee  cómo parámetros \textit{a} y \textit{b} para describir los distintos fluidos .

\subsection{Validación}

La validación de éste problema se hizo utilizando los parámetros $a =0,5$ y $b = 4,0$; para un tamaño de malla de 201 x 201 nodos y $T_r$ variando con un paso de $0,025$ en el rango de $[0,6 - 0,975]$.  Los valores que se utilizaron de $\Lambda$ y $\mathbf{M}$ para el modelo de LBM realizado se encuentran en el Apéndice (xxxx).
%La descripción de cómo se realizó la curva de coexistencia analítica se encuentra en el Apéndice (xxxx).

La Figura(\ref{fig:v_760_MxC_c_simple}) muestra la validación del código realizado en \textbf{C} para simple precisión en una GPU NVIDIA Geforce GTX 760; con distintos parámetros $\sigma$ del modelo MRT, donde se observa que el valor de $\sigma = 0,125$ es el que mejor ajusta a la curva de coexistencia. La Figura (\ref{fig:v_760_MxC_cuda_simple}) muestra el resultados obtenidos del código realizado en \textbf{CUDA C} en simple precisión en la misma GPU.

\begin{figure}[h!]
	\centering
	\includegraphics[width=\textwidth]{figs/cap4/v_760_MxC_c_simple}
	\caption{Curva de coexistencia de fases para un fluido de VdW con los parámetros $a = 0,5 $ y $b = 4,0 $, obtenida en simple precisión en la GPU NVIDIA Geforce GTX 760 en el código desarrollado en \textbf{C}. $\sigma = 0.075[\bigtriangleup]$	 $\sigma = 0.125[\bigcirc]$ y $\sigma = 0.200[\diamondsuit]$ }
 	\label{fig:v_760_MxC_c_simple}	
\end{figure}

\begin{figure}[h!]
	\centering
	\includegraphics[width=\textwidth]{figs/cap4/v_760_MxC_cuda_simple}
	\caption{Curva de coexistencia de fases para un fluido de VdW con los parámetros $a = 0,5 $ y $b = 4,0 $, obtenida en simple precisión en la GPU NVIDIA Geforce GTX 760 en el código desarrollado en \textbf{CUDA C}. $\sigma = 0.075[\bigtriangleup]$	 $\sigma = 0.125[\bigcirc]$ y $\sigma = 0.200[\diamondsuit]$ }
	\label{fig:v_760_MxC_cuda_simple}	
\end{figure}

Ambos resultados mostrados en las Figuras (\ref{fig:v_760_MxC_c_simple}) y (\ref{fig:v_760_MxC_cuda_simple}) fueron realizados para 50000 pasos de tiempo en cada uno de los valores de $T_r$. El valor obtenido de las densidades de coexistencia de fases entre los códigos de \textbf{C} y \textbf{CUDA C} resultan exactamente iguales. 

El resultado que se obtuvo de realizar la validación en doble precisión es que los dos códigos desarrollados obtienen el mismo valor en las densidades de fase.

\newpage
\subsection{Comparación de precisiones}

En el análisis de la comparación de las precisiones, sólo se presentan los resultados obtenidos mediante el código de \textbf{C}; debido a que los resultados de \textbf{CUDA C} son idénticos.

La Figura (\ref{fig:v_760_MxC_c_comparacion}) muestra los resultados de las densidades de coexistencia de fases para simple y doble precisión, no siendo apreciable la diferencia.

\begin{figure}[h!]
	\centering
	\includegraphics[width=\textwidth]{figs/cap4/v_760_MxC_c_comparacion}
	\caption{Curva de coexistencia de fases para un fluido de VdW con los parámetros $a = 0,5 $ y $b = 4,0 $, obtenida en simple precisión[$\bigcirc$] y doble precisión [$\diamondsuit$] en la GPU NVIDIA Geforce GTX 760 en el código desarrollado en \textbf{C}.} 
	\label{fig:v_760_MxC_c_comparacion}	
\end{figure}


La Tabla (\ref{tab:comp_MxC_precisiones_10}) contiene los valores de las densidades de coexistencia: analítico, y los obtenidos mediante el código en simple precisión y doble precisión. De la Figura (\ref{fig:v_760_MxC_c_comparacion}) se observa que se ajusta mejor al resultado analítico la fase líquida que la fase gaseosa, por ello la Tabla (\ref{tab:comp_MxC_precisiones_10}) presenta esta diferenciación.

Para dar una idea de proximidad a la solución analítica se adoptó como parámetro la distancia entre los vectores de densidad de fase obtenido y el analítico. La distancia se calculó por medio de la Norma Euclídea, siendo calculada la distancia entre dos vectores \textbf{\textit{A}} y \textbf{\textit{B}} con \textit{i} elementos como indica la Ec.(\ref{eq:norma_euclidea}):

\begin{align}
dist(\mathbf{A},\mathbf{B}) = \sqrt{\sum_i {\left( a_i - b_i \right)}^2  }
\label{eq:norma_euclidea}
\end{align}

Los resultados que muestra la Tabla (\ref{tab:comp_MxC_precisiones_10}) en cuanto a la distancia de los vectores se observa que en doble precisión los resultados se aproximan mejor en la densidad de coexistencia de la fases, en la fase gaseosa; mientras que para la densidad de coexistencia de fase, de la fase líquida, se aproxima mejor mediante simple precisión. 

Porcentualmente para la fase gaseosa, la distancia calculada en simple precisión es 0,0034 \% mayor que en doble precisión; para la fase líquida la distancia en doble precisión es 0,0491 \% mayor que en simple precisión.


% Please add the following required packages to your document preamble:
% \usepackage{multirow}
\begin{table}[h!]
\centering
%\resizebox{17cm}{!}{
	\begin{tabular}{|c|c|c|c|c|c|c|}
	\hline
	& \multicolumn{3}{c|}{${\rho_{r}}_{\>gaseoso}$}      & \multicolumn{3}{c|}{${\rho_{r}}_{\>líquido}$} \\ \hline
	$\mathbf{T_r}$    & \textbf{Analítico}      & \textbf{Simple}       & \textbf{Doble}     & \textbf{Analítico}      & \textbf{Simple}     & \textbf{Doble}   \\ \hline
	0.600 & 0.0599097 & 0.0653772 & 0.0653724 & 1.32424  & 1.31956 & 1.31975 \\ \hline
	0.625 & 0.0733723 & 0.0848112 & 0.0847656 & 1.46149  & 1.46239 & 1.4624  \\ \hline
	0.650 & 0.0897449 & 0.0951336 & 0.095124  & 1.56762  & 1.56858 & 1.56859 \\ \hline
	0.675 & 0.107606  & 0.113153  & 0.113147  & 1.56762  & 1.65819 & 1.65821 \\ \hline
	0.700 & 0.128332  & 0.13353   & 0.133511  & 1.6572   & 1.73703 & 1.73704 \\ \hline
	0.725 & 0.150966  & 0.15182   & 0.151811  & 1.73595  & 1.80812 & 1.80813 \\ \hline
	0.750 & 0.177353  & 0.177323  & 0.177319  & 1.80706  & 1.87326 & 1.87328 \\ \hline
	0.775 & 0.206739  & 0.206086  & 0.206075  & 1.87233  & 1.93364 & 1.93364 \\ \hline
	0.800 & 0.23938   & 0.244229  & 0.244223  & 1.93243  & 1.98754 & 1.98759 \\ \hline
	0.825 & 0.277393  & 0.281299  & 0.281284  & 1.98899  & 2.04083 & 2.04085 \\ \hline
	0.850 & 0.319677  & 0.323471  & 0.323456  & 2.0423   & 2.09124 & 2.09124 \\ \hline
	0.875 & 0.368925  & 0.371915  & 0.371891  & 2.09234  & 2.14114 & 2.14117 \\ \hline
	0.900 & 0.425549  & 0.428416  & 0.428335  & 2.14012  & 2.18665 & 2.18665 \\ \hline
	0.925 & 0.493618  & 0.495973  & 0.495947  & 2.18563  & 2.23017 & 2.23019 \\ \hline
	0.950 & 0.493618  & 0.580402  & 0.580368  & 2.22933  & 2.27407 & 2.27411 \\ \hline
	0.975 & 0.578746  & 0.690126  & 0.690247  & 2.27117  & 2.312   & 2.312   \\ \hline
	\textbf{Distancia} & -         & 2.05233   & 2.05226   & -        & 0.14234 & 0.14241 \\ \hline
\end{tabular}%}
    \caption{Comparación de las precisiones con respecto a cuánto se acercan al valor analitico, tanto para doble, como simple precision, la norma utilizada para medir la distancia de los vectores es la norma  euclídea. Para el problema de la Construcción de Maxwell con la GPU NVIDIA Geforce GTX 760.}
    \label{tab:comp_MxC_precisiones_10}
    \end{table}



\newpage

\subsection{Speed Up}

En la presente sección se muestran las mejora en el tiempo de cálculo realizados para el código de \textbf{C} y \textbf{CUDA C}. La comparación se realizó en simple y doble precisión; en dos GPU, siendo las mismas NVIDIA Geforce GTX 760 y NVIDIA Geforce GTX 970. Se tomó una $T_r$ fija y se varió el tamaño de la grilla, de manera que ésta siempre fuese cuadrada, respetando un número de nodos de potencia de 2 en los lados del cuadrado. La cantidad de \textit{thread blocks} que se utilizó para realizar la comnparación en el código de \textbf{CUDA} fueron de potencia de 2.

\subsubsection{NVIDIA Geforce GTX 760}

Los tamaños de grilla que se utilizaron para realizar las pruebas de tiempo de ésta placa, tienen el rango de grilla de 16x16 nodos hasta 2048x2048 nodos. La cantidad de \textit{thread blocks} que se utilizó fueron de 1 a 512.

Las Figuras (\ref{fig:s_760_MxC_simple_1.0}) y (\ref{fig:s_760_MxC_double_1.0}) muestran el \textit{Speed Up} obtenido comparando los códigos de \textbf{C} y \textbf{CUDA C}, donde la Figura (\ref{fig:s_760_MxC_simple_1.0}) está obtenida con simple precisión y la Figura (\ref{fig:s_760_MxC_double_1.0}) en doble precisión. El mejor resultado en ambos casos se obtuvo para un número de \textit{thread block} igual a 64, donde la mejora fue de 18.67 y 11.40 en simple y doble precisión respectivamente, para el mayor número de elementos de malla.


\begin{figure}[h!]
	\centering
	\includegraphics[width=\textwidth]{figs/cap4/s_760_MxC_simple_10}
	\caption{Speed Up realizado para el problema de la Construcción de Maxwell con la GPU NVIDIA Geforce GTX 760 en simple precisión, comparando los códigos de \textbf{C} y \textbf{CUDA C}.} 
	\label{fig:s_760_MxC_simple_1.0}	
\end{figure}

\begin{figure}[h!]
	\centering
	\includegraphics[width=\textwidth]{figs/cap4/s_760_MxC_double_10}
	\caption{Speed Up realizado para el problema de la Construcción de Maxwell con la GPU NVIDIA Geforce GTX 760 en doble precisión, comparando los códigos de \textbf{C} y \textbf{CUDA C}.} 
	\label{fig:s_760_MxC_double_1.0}	
\end{figure}

\newpage

Las Figuras (\ref{fig:c_760_MxC_c_10}) y (\ref{fig:c_760_MxC_cuda_10}) muestran el \textit{Speed Up} obtenido comparando simple precisión y doble precisión, donde la Figura (\ref{fig:c_760_MxC_c_10}) está obtenida el código de \textbf{C} y la Figura (\ref{fig:c_760_MxC_cuda_10}) en el código de \textbf{CUDA C}. 

En el código de \textbf{C} para el mayor número de elementos de la malla, el resultado de tiempos de cálculo en doble precisión es apenas 1,026 veces mayor que en  simple precisión. En contraste, al fijarse el resultado para un número de \textit{thread block} igual a 64 (el que mayor ganancia obtuvo), el tiempo de cálculo en doble precisión es 1,68 veces mayor que en simple precisión; para el mayor número de elementos de malla calculado.

\begin{figure}[h!]
	\centering
	\includegraphics[width=\textwidth]{figs/cap4/c_760_MxC_c_10}
	\caption{Speed Up realizado para el problema de la Construcción de Maxwell con la GPU NVIDIA Geforce GTX 760 en en el código de \textbf{C}, comparando simple precisión y doble precisión.} 
	\label{fig:c_760_MxC_c_10}	
\end{figure}

\begin{figure}[h!]
	\centering
	\includegraphics[width=\textwidth]{figs/cap4/c_760_MxC_cuda_10}
	\caption{Speed Up realizado para el problema de la Construcción de Maxwell con la GPU NVIDIA Geforce GTX 760 en en el código de \textbf{CUDA C}, comparando simple precisión y doble precisión.} 
	\label{fig:c_760_MxC_cuda_10}	
\end{figure}

Los valores que se obtuvieron en las Figuras (\ref{fig:s_760_MxC_simple_1.0}), (\ref{fig:s_760_MxC_double_1.0}), (\ref{fig:c_760_MxC_c_10}) y (\ref{fig:c_760_MxC_cuda_10}) se encuentran en el Apéndice \ref{apend_MxC_760}

\newpage

\subsubsection{NVIDIA Geforce GTX 970}

Los tamaños de grilla que se utilizaron para realizar las pruebas de tiempo de ésta placa, tienen el rango de grilla de 16x16 nodos hasta 4096x4096 nodos en simple precisión y de 16x16 nodos hasta 2048x2048 nodos en doble precisión . La cantidad de \textit{thread blocks} que se utilizó fueron de 1 a 512.

Las Figuras (\ref{fig:s_970_MxC_simple_10}) y (\ref{fig:s_970_MxC_double_10}) muestran el \textit{Speed Up} obtenido comparando los códigos de \textbf{C} y \textbf{CUDA C}, donde la Figura (\ref{fig:s_970_MxC_simple_10}) está obtenida con simple precisión y la Figura (\ref{fig:s_970_MxC_double_10}) en doble precisión. El mejor resultado en ambos casos se obtuvo para un número de \textit{thread block} igual a 32, donde la mejora fue de 23.39 y 10.96 en simple y doble precisión respectivamente, para el mayor número de elementos de malla.

Las Figuras (\ref{fig:c_970_MxC_c_10}) y (\ref{fig:c_970_MxC_cuda_10}) muestran el \textit{Speed Up} obtenido comparando simple precisión y doble precisión, donde la Figura (\ref{fig:c_970_MxC_c_10}) está obtenida el código de \textbf{C} y la Figura (\ref{fig:c_970_MxC_cuda_10}) en el código de \textbf{CUDA C}.

En el código de \textbf{C} para el mayor número de elementos de la malla, el resultado de tiempos de cálculo en doble precisión es apenas 1,035 veces mayor que en  simple precisión. En contraste, al fijarse el resultado para un número de \textit{thread block} igual a 32 (el que mayor ganancia obtuvo), el tiempo de cálculo en doble precisión es 1,29 veces mayor que en simple precisión; para el mayor número de elementos de malla calculado.

\begin{figure}[h!]
	\centering
	\includegraphics[width=\textwidth]{figs/cap4/s_970_MxC_simple_10}
	\caption{Speed Up realizado para el problema de la Construcción de Maxwell con la GPU NVIDIA Geforce GTX 970 en simple precisión, comparando los códigos de \textbf{C} y \textbf{CUDA C}.} 
	\label{fig:s_970_MxC_simple_10}	
\end{figure}

\begin{figure}[h!]
	\centering
	\includegraphics[width=\textwidth]{figs/cap4/s_970_MxC_double_10}
	\caption{Speed Up realizado para el problema de la Construcción de Maxwell con la GPU NVIDIA Geforce GTX 970 en doble precisión, comparando los códigos de \textbf{C} y \textbf{CUDA C}.} 
	\label{fig:s_970_MxC_double_10}	
\end{figure}


\begin{figure}[h!]
	\centering
	\includegraphics[width=\textwidth]{figs/cap4/c_970_MxC_c_10}
	\caption{Speed Up realizado para el problema de la Construcción de Maxwell con la GPU NVIDIA Geforce GTX 970 en en el código de \textbf{C}, comparando simple precisión y doble precisión.} 
	\label{fig:c_970_MxC_c_10}	
\end{figure}

\begin{figure}[h!]
	\centering
	\includegraphics[width=\textwidth]{figs/cap4/c_970_MxC_cuda_10}
	\caption{Speed Up realizado para el problema de la Construcción de Maxwell con la GPU NVIDIA Geforce GTX 970 en en el código de \textbf{CUDA C}, comparando simple precisión y doble precisión.} 
	\label{fig:c_970_MxC_cuda_10}	
\end{figure}

Los valores que se obtuvieron en las Figuras (\ref{fig:s_970_MxC_simple_10}), (\ref{fig:s_970_MxC_double_10}), (\ref{fig:c_970_MxC_c_10}) y (\ref{fig:c_970_MxC_cuda_10}) se encuentran en el Apéndice \ref{apend_MxC_970}


\subsection{Análisis}




\section{Estratificación de un fluido VdW con temperatura no uniforme}

Se quiere resolver el problema de, tener una cavidad unidimensional en presencia de un fluido cuya EOS es la de Van der Waals; con temperatura  no uniforme y fuerza de gravedad no nula. Éste problema fue desarrollado por Berberan-Santos \cite{berberan2002liquid} y que Fogliatto \cite{fogliatto2019simulation} extendió. 

Se toma como coordenada del problema \textit{y}, teniéndose una temperatura fija $T_{0}$ en $y = 0$ y $T_{1}$ en $y = H$ como se observa en la Figura (xxxx) que se muestra en presencia de la gravedad.





Se toma el problema de tener una cavidad unidimensional ocupada por un fluido cuya EOS es de Van der Waals, para temperatura no uniforme y fuerza de gravedad no nula.



\section{Generación de burbujas sobre una superficie horizontal calefaccionada}
%%% Local Variables: 
%%% mode: latex
%%% TeX-master: "template"
%%% End: 
